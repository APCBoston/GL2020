\documentclass[char]{GL2020}
\parindent=0pt
\begin{document}
\name{\cAssistantScientist{}}

You are \cAssistantScientist{}, (\cAssistantScientist{\they}/\cAssistantScientist{\them}), and 25 years old. 6 years ago you were \cKidScientist{}, a student at \pSc{} -- and the only one out of the group of 12 voting students to make it out alive. Despite your new name and appearance, you've still spent the past 6 years looking over your shoulder, wondering if someone is out there looking for you to finish the job. Being back at \pSc{} is the best chance you'll have to figure out what happened 6 years ago -- assuming nobody recognizes you while you're at it.

Your family is from a small fiefdom in \pFarm{} that borders \pTech{}, but you have lived in \pTech{} most of your life. Born of a midnight romance, your mother emigrated to \pTech{} when you were still just a promise in her womb. Unlike many families crossing nations, your family encouraged you to maintain close relationships with both sides of your lineage, and you spent many a happy summer break in \pFarm{}, playing in the vineyards, helping with the harvest, and learning about wine -- knowledge that would soon save your life after you enrolled in the \pSc{} and participated in the Ritual.

The storms were supposed to go to \pTech{} that year, not the \pShip{}. You, and every student present, knew that they wouldn’t have prepared. All the discussion you were privy to suggested students were voting for the \pTech{}. And yet, somehow, the votes came in, and the \pShip{} were the ones targeted, leaving everyone present stunned. Your thoughts swirled. How many would die because of this betrayal? How had such a conspiracy gone undetected? What would happen to you, as a member of the class that so visibly voted to upset the balance between the three nations? But as you looked around and counted, your suspicion grew.  You had, of course, followed with tradition and voted for the \pTech{}, and you’re sure the \pFarm{} student with the most votes had too. Assuming that the students from the \pShip{} hadn’t betrayed their own country, it seemed mathematically impossible for this outcome to have occurred.

The mulled wine they brought out interrupted your racing thoughts with the automatic and bone-deep indignation your \pFarm{} relatives had imparted to you. This wine was far too fine to be heated and tainted with spices -- this vintage was one of the more expensive and delicate vintages exported by \pFarm{}! Anyone who could afford to buy such a vintage should know how to properly consume it! Grimacing at the strong odor coming from the carafes, you watched with distaste as your fellow students began to partake.

As you were about to hold your nose and join your friends in drinking from the blasphemously tampered wine, one of the students from \pFarm{} (who had to be on at \emph{least} their third glass by now) coughed, moved as if to put his half-empty glass upon a table, and collapsed to the floor. He began to seize, and white foam built and spilled from the edges of his mouth as he fought for air.  

Within moments, another student joined him, then another. One by one, your classmates fell.  The door to the outside hall was locked, and nobody answered the screams for help.  Nothing you did made any difference -- and within what seemed like minutes, everyone around you was dead. The sheer horror of the sight overwhelmed you, and you blacked out and knew no more.

After the Betrayal, as you have come to call that fateful night, you woke up in a haze. You were somewhere dark, but at least you were outside - no longer trapped in that room. In your confused and distraught state, you ran, and ran, and ran, and knew not where. Eventually, when you recovered your senses, you realized that you were standing within the Storm itself -- and that the Storm-ravaged \pShip{} was exactly where you needed to be. After all, you didn’t know if anyone was pursuing you, and a country ravaged by the Storm was the last place anyone would come looking. The ferocity of the Storms shook you to your core -- but it didn’t kill you. So you fled -- into thunderstorms, where the rain fell so heavily it felt like you were being battered, into windstorms where the winds howled and near lifted you off your feet. With the winds howling in your ears like the screams of the dying, all you could hear was your classmate’s last moments. And the silences, in those brief moments of respite within the tempest, where everything paused for several minutes before the wind and rain and hail and thunder set back in -- those silences were filled with their whispered prayers for salvation.

\cSaviourFleet{} found you -- rambling, muttering incoherently to yourself, delirious with pneumonia and exposure. They nursed you back to health. And when you were well enough, you hesitantly told them who you were and what had happened. Where you expected to find anger, you found acceptance. You had feared that they would hate you for what your people had done to them. But instead, they cared for you as if you were one of their own lost children.

It pained you to leave your rescuers, but you knew that the story didn’t end this way. You had to go back. You had to work your way back into the \pTech{} and find out what had really happened during the Betrayal. \cSavFlet{} was supportive the whole way. They helped you change your identity, obtain forged documents from the burgeoning black market, and remake you into \cAssistantScientist{} -- the identity you still go by to this day. You can never repay them for their kindness to you, but you still do what you can. As Assistant Scientist to \cHeadScientist{}, you have access not only to your own research pertaining to magic and the ending of the Storms, but to all kinds of magical tech destined for the war efforts. A little magical nudge here or there, nothing detectable of course, is really all it takes to ensure lengthy and costly repairs once the sabotaged tech gets to its final destination.

As for the magic to end the Storms - well, it doesn’t work. In fact, your research has concluded that it is impossible to end the Storms. You and \cHeadScientist{} have known that it doesn’t work for six full months, which certainly has the Council -- and thus, \cHeadScientist{} -- in a pickle.  After all, one can’t just admit that one has broken a centuries-long treaty and sentenced who knows how many innocents to their deaths if one doesn’t have a really nice carrot on the end of that stick. You and \cHeadScientist{} are supposed to present your solution at the \pSc{} this weekend. \cHeadScientist{} is really, really worried about it. And while you don’t exactly feel like you owe any loyalty to the country, you know enough to save your own skin. If the two of you work together, there is a very slim chance you can keep a lid on this secret and prevent anyone finding out. It is far more likely that people will find out, and you’ll be stuck doing damage control. If you and \cHeadScientist{} stick together, you and your careers \emph{might} survive. But if \cHeadScientist{\they} manage to pin this on you, you’re screwed. So maybe you should try to beat \cHeadScientist{\them} to the punch and throw \cHeadScientist{\them} under the cart first? It’s not so unbelievable that  \cHeadScientist{} wouldn’t like to share knowledge with her ``lowly assistant.’’ It is a hard decision; your best chance is if you both work together. But can you trust \cHeadScientist{} when \cHeadScientist{\their} precious career is on the line?

There are also other, probably more important things to work on this weekend. Namely, you aim to figure out what really happened 6 years ago -- without blowing your cover -- and make sure nothing like it happens again. This is another politically charged cycle -- and similar to 6 years ago, there are important treaties under discussion.  The \pSc{} is like a powder keg about to explode, you can feel it -- and you NEED to make sure history doesn’t repeat itself.

Advisors from \pFarm{} or \pTech{} who were present at the school six years ago, namely \cDiplomat{} and \cEvil{}, are to be treated with extreme suspicion. After all, you don’t know who was behind the Betrayal, but both nations have profited greatly from it. It’s always possible, of course, that whoever masterminded the conspiracy either was in deep cover at the school before the Betrayal, or wasn’t here for the actual execution -- so EVERY \pFarm{} or \pTech{} teacher and advisor should be considered suspicious. You don’t KNOW they’ll be here this weekend, of course\ldots but if there was ever a time for a repeat performance for 6 years ago, it would be today. As for people who might help, \cHeir{} is the younger \cHeir{\sibling} of one of your former classmates, \cHeirSibling{}, who was your best friend before \cHeirSibling{\their} death. You originally met \cHeir{} at a function held by the Faledon family, where you approached \cHeir{\them} purely to see if \cHeir{\they} were like \cHeir{\their} \cHeir{\sibling}, but the two of you hit it off even though you didn’t reveal your connection to \cHeirSibling{}, and you’ve kept in touch since and become friends. It’s likely that if you ask, \cHeir{\they} could be of help with your investigations. 

You didn’t pay much attention to the preparations for the Ritual during your last year at the school, but you feel compelled to be involved now. Something during the Ritual went catastrophically wrong before, and without your watchful eye on it, something could happen again.  Similarly, you need to keep an eye on the Bunkers -- they’re due for maintenance this cycle and if someone was planning something nefarious, this would easily be one way to ensure that no one makes it off the island alive. And as if you weren’t already busy enough, the ritual to maintain the ley lines also happens on the 3 year cycle -- you want to try to be involved with that as well. You're not sure that ritual six years ago contributed to the storm somehow being sent to /pShip{}, but you don't want to risk overlooking it.

You also are still working to find ways to help \cSavFlet{}{} and \pShip{}.  Recently, you learned through the grapevine that a student of \pTech{} named \cTechStar{} engineered a vidcom device that allows easy communication across large distances. Apparently, \cTechStar{\They} guard\cTechStar{\plural} the design for the devices carefully -- but you want to get your hands on either a vidcom device or the plans for one so that you can send it to \pShip{} through \cSavFlet{}. Either way, once you have the info you need, you need to ensure that no further vidcom devices are created for use in \pTech{} or \pFarm{} - it might be the key to \pShip{} winning the war. If all else fails, you could\ldots eliminate\ldots \cTechStar{} -- but there are probably cleaner, more humane, and less costly ways to do it. But you won’t hesitate if that’s what’s necessary. After all, what is the life of this one student when weighed against the livelihood of all those who you will save by removing this incredibly dangerous weapon from the hands of those responsible for the Betrayal?

Hopefully this weekend you can finally lay your burdens to rest and avenge your fallen classmates, while preventing anything from happening to this year's voting class. Somehow, you will bring their murders to justice -- even if you have to tear apart the current order of the world in order to do it.

\begin{itemz}[Goals]
	\item Find out what happened 6 years ago, acquire proof, and make it off the island alive with it, to bring it to the proper authorities. \cHeir{} can probably be made into an ally in this.
	\item Make sure that the Ritual to control the Storm goes as planned.
	\item Make sure nothing happens to the Bunkers.
	\item Make sure that the ritual to renew the leylines goes according to plan.
	\item Secure the vidcom devices for \pShip{} and ensure that \pTech{} and \pFarm{} can’t get any more.
	\item Figure out what to do about the fact that there is no way to end the storms.
	\item Perform your advisor and assistant duties well enough to not be suspected of having other priorities.
\end{itemz}

\begin{itemz}[Notes]
	\item 
\end{itemz}

\begin{contacts}
	\contact{\cHeadScientist{}} Your boss. You’ll have to decide if you trust \cHeadScientist{}, or if you are going to try to pin the inability to end the storms on \cHeadScientist{\them} somehow.
	\contact{\cDiplomat{}}  \pTech{} advisor and diplomat. One of your prime suspects for the murders.
	\contact{\cEvil{}} \pFarm{} advisor and bigwig. One of your prime suspects for the murders.
	\contact{\cHeir{}} Younger \cHeir{\sibling} to \cHeirSibling{}, one of the students murdered 6 years ago, who had been your best friend. And as \cHeir{\formal}, \cHeir{} is a big deal in \pTech{}, as well as being a friend to you since you met recently.  
\end{contacts}

\end{document}



