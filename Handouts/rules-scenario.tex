\documentclass[sheet]{GL2020}

\usepackage{graphicx}
\graphicspath{ {./images/} }
\usepackage{xcolor}
\usepackage{hyperref}
\usepackage{multicol}
\usepackage{ltablex}
\usepackage{tabularx}
\usepackage{indentfirst}
\renewcommand{\tabularxcolumn}[1]{m{#1}}
\setlength{\columnsep}{1cm}

%% document-wide tweaks
\interlinepenalty10000
\setstretch{1}
\def\mytype{Rules and Scenario}
\lfoot{}\rfoot{}

\begin{document}

%% layout for cover page
\thispagestyle{empty}
\parskip0pt

%% title box
\begin{center}\LARGE\bf\begin{tabular}{|c|}
  \hline \gamename\\ \gamedate\\ Rules and Scenario\\ \hline
\end{tabular}\end{center}

\vfill\vfill

This game and all materials thereof are copyright 2021 by Acata Felton, Aaron Sunshine, Eric Fritz, Jeremy Cole, Kelsey Miranda, and Koh Henderson.\\

\vfill\vfill

\begin{center}\bf
  BROUGHT TO YOU BY THE LUMINARY ROLEPLAY SOCIETY
\end{center}

\vfill

\clearpage

%% layout for Table of Contents page
\thispagestyle{empty}
\tableofcontents

\clearpage

%% layout for main body of rules
\setcounter{page}{1}
\parskip5pt
\vfill
\section{Introduction}

The following are the rules for {\em\gamename}, a real-time, real-space, roleplaying game sponsored by the Luminary Roleplay Society. \textcolor{red}{This version of the rules has been updated as of 10Oct2022. Changes, additions, and clarifications are presented in red text, just like this.}

\subsection{Expectations for Players}
You are responsible for knowing and following these rules. It is through the constraints created by these rules and other mechanics that interesting character to character interactions are made possible. Many of these rules are nigh-impossible for GMs to enforce, and rely upon the honor system. Do not cheat. Do not abuse loopholes. Play fair. To do otherwise will deprive both yourself and your fellow players out of the experience you signed up for.

We estimate the number of pages of reading for this game is between 40 and 50 pages for most characters, but this may change a little as we finalize game material. This page estimate is across all of the documents, including this rules document. We will do our best to ensure that you have at least 2 months with your character material (20-30 pages total) before game. Please make sure you set aside enough time to prepare for game by familiarizing yourself with all of the documents. You do not need to memorize anything (except maybe your CR stat), but you should read everything in detail at least once, and know where to look up information. Print copies of everything will be provided at registration.

\subsection{GM Commitment}
The \textbf{gamemasters} (\textbf{GMs}) run the game. If you have any problems or questions concerning the game, contact a GM. Rulings the GMs make are final.  They may violate the letter of the rules to preserve the spirit.  The GMs promise to be as fair and reasonable as possible. Neither they nor these rules are perfect, so we ask for your understanding and flexibility.

\subsection{Disclaimers and Acknowledgments}
This game is a work of fiction. The attitude of the characters and the social milieu of the game world do not necessarily represent the opinions of the GMs, or of the player playing that character. While any work of fiction must draw some inspiration from real world situations, it is not intended to be a replication or direct allegory of any real world experience. We understand that the player experience is necessarily grounded in their own lived experiences in life, and have done our best to be mindful of the ways that game content can interact with that. 

\subsection{Game Style}
This game is Secrets and Powers (S\&P) style aka ``Lit Form''. Game content is created by the GMs, and hooks are given to players to start various plots. Character goals may be mutually exclusive, leading to character vs. character conflict. Players will get the most out of the game by embracing their character, playing to discover the world created by the game writers, and reacting to new revelations from other PCs in character. Players will likely not come out of this game having accomplished everything their character wanted.

While characters are often ``playing to win'', players should look to the best stories. Embrace the most narratively impactful moment to have your character back down, admit a failure, spill a secret, or change their mind about something. Also keep in mind that you do not have complete narrative control over your character's story. Your secrets may come out for reasons that are out of your control. Other characters may succeed in stymieing your goals, or sneak their opposing goals in under your nose. We encourage you to embrace these additions to your character's story. You can use a ``yes, and'' approach to respond to the actions and revelations of other characters that impact your character, as you would in a more free-form game. This approach to storytelling allows players to lean into characters who want to keep a secret hidden while still enjoying the full breadth of their character arc.

This game also contains significant interactions between PCs and the environment. These interactions are regulated through predefined mechanics that allow players to interact with elements of the environment without waiting for a GM to narrate something or make a ruling. This generally leads to less sitting around waiting out of character, and more time playing and interacting in character. 

This is not a ``what you see is what you get'' (WYSIWYG) game. You will need to explain to each other out  of character what characters perceive in situations where it is not obvious out of game; e.g.\ ``My character's hands are covered in blood.'' {\bf Metagaming} is inferring in-game knowledge that is inappropriate for your character from out-of-game information. While some players may be more familiar with games with higher levels of transparency, not every player enjoys having to firewall a lot of information, or having their character's secrets revealed out of game before they are revealed in game. Please do not volunteer information to another player that their character might not have, or solicit information from them that your character might not have. However, if necessary for safety, players may discuss secret character information ahead of time. We generally find that this is necessary less frequently than you might think in this style of game. Trust your fellow players to enjoy unexpected twists in the story as new information is revealed to players and characters simultaneously.

\subsubsection{Game-Wide Secrets}
Secrets and Powers style games often have game-wide secrets that are revealed part way through the event. These can change the way characters view and interact with the world, and therefore character priorities. This game has the potential for several \textbf{world shifting} secrets to come to light, and for characters to take \textbf{world changing} actions. The GMs will do our best to make sure that any such secrets are interesting and satisfying, rather than leaving players feeling like the reveal invalidates their character and character goals. 

\subsubsection{Player Experience}
Players should expect to spend time before the weekend of game familiarizing themselves with game material, and preparing any costuming they choose. There is no need to connect with other players before the game -- there will be time for that during Friday workshops (onsite). During the game, players should plan to spend extensive amounts of game time in character, interacting with each other and the world. Players should not expect to need to spend extensive periods of time out of character negotiating things. Scenes can typically go wherever the characters take it, unless someone invokes a safety mechanic (see below).

\subsubsection{Costuming} 
\textbf{Costumes are always admired but never required.} Costume shaming of any kind will not be tolerated. While it can be fun to have a costume, making something from scratch, thrifting and altering, or finding the prefect premade item can be time consuming and expensive. Not everyone has time and money for that. If you choose to costume, keep in mind that much of game will be happening outside, and that therefore good footwear will be important. We encourage layers, as it can be quite cool at night in November. \textbf{Please ensure that you are able to wear a mask (for covid protection) with your costume.} We won't be making the call on whether masks are necessary until the last minute (approximately 1 week before game), so you should plan for it just in case.


\section{Logistics}
\subsection{Basic Information}
\begin{itemize}
  \item \textbf{Dates and Times:} Players will be expected to be on site at the event location from 1 pm  on Fri, Nov 11th - 5 pm on Sun, Nov 13th. Arriving late or leaving early requires special dispensation from the GMs. Please request any such arrangements well in advance.
  \item \textcolor{red}{\textbf{Location:} Community of the Great Commission; 30303 Chicken Hawk Rd, Foresthill, CA 95631. (~3 hr 15 min drive from SJC Airport)}
\end{itemize}

\subsubsection{Sleeping Arrangements}
All sleeping spaces at the camp are twin bunk beds, but we have secured enough sleeping space that everyone should have their pick of top or bottom bunks. Each sleeping space sleep up to 10 people, although we will be spreading people out to a lower density among the spaces.

\section{Game Content}
As this game is a S\&P, the GM team has prepared extensive information about the premise and direction of the game, as well as the types of personal plots and stories that are likely to show up.

\subsection{The Premise}
\emph{Welcome to the world of \pEarth{}. The Gods have gifted the people with magic, but at a cost -- the ravages of magical devastation roaming across the land once every 3 years. Only a chosen few students can manipulate the path of these magical Storms. Until recently, a treaty ensured a tenuous peace. The three nations agreed that the Storm should hit each nation in turn, sharing the devastating burden. Although the Storms always took their toll, no one nation suffered too greatly. That all changed when the treaty was broken, and a nation betrayed. Although their architects and diplomats struggle to repair what was lost, the work won't be complete by the next Time of Deciding.}

\emph{Our story begins at the center of the known world. Here at the \pSchool{}, students are trained in the rituals necessary to change the path of the Storm. Instructors maintain the veneer of impartiality -- but in reality, each fights for their own nation's interests to varying degrees, all vying for control of the magical disaster. But despite their efforts, no nation has the complete picture -- and new interests conspire to change the fate of the world. Where the magic will ravage next -- and what consequences it will have -- are for you to decide.}

\subsection{Themes and Topics}
This game is set in a fantasy world with strong magical and religious elements. Not all is well however, and the people in this world are facing many difficult choices. We ask that players prepare to engage with these themes, as they constitute much of the fabric of the world.

While having fun is a primary goal of most players it is important to recognize that some of the themes and topics addressed in the game are quite serious. All of the characters are facing very significant moral and ethical choices that have real world analogies in the abstract, if not the specific. It is possible to both have fun at this game, and not dismiss the gravity of the decisions being made, and the consequences they will have for this fantastic world.

\subsubsection{Magic and its Price}
Magic in this world is the source of almost everything. Food can be grown without magic, but not nearly enough to feed the population. Ships cannot be built well enough to withstand a sea serpent attack. Technology runs on magical energy, and cannot be made to function without it. There is no corner of life on \pEarth{} that is not dependent on magic, and the loss of magic would be as unimaginable as you the player living in a world where electricity stopped working.

The Storms are also magical in origin and nature. Every three years, a magical Storm brews over the lake where the \pSc{} resides, and then whirls off to cut a swath of destruction across the continent. To date, no one has been able to figure out how to stop this from happening. The best anyone could do was use the Ritual to direct the Storm toward a particular country. This is not a trivial problem to be solved overnight -- it has shaped the land, the lives, and the politics of \pEarth{} since the storms began.

\subsubsection{Religion}
Religion is a big part of this world, and the lives of all of the characters. To help bring the world of \pEarth{} to life, we ask players to engage with the trappings of the religion, at least casually. As part of workshops, the players from each nation will create several common practices they share. The clerics of each religion will also pick and announce a time for a short religious ceremony of some kind (recommend no more than 15 minutes) during the weekend. If you miss it, it is reasonable to expect social consequences, even if you have a very good explanation.

This serves a few purposes. Firstly, this \textbf{is} a world with Gods. The Gods exist. They gave people magic. They appear to people, or speak through their Avatars frequently. It would take almost total isolation and ignorance to make someone a true atheist, or even agnostic in this world. The most lively areas of study all involve innovative magic use, and there is no way to separate the intellectual from the magical from the religious in this world. This is an intentional part of the fiction of the world. 

Secondly, participating in shared rituals helps create a deep sense of connection and community between the characters, which is an experience we hope to foster for the players. 

Lastly, for those few characters that do reject part or all of the religion, the experience of trying to navigate a religious world can only happen if the rest of the players help create that world for them to navigate. Unless your character sheet specifically describes your character as questioning or rejecting the religion, you should assume that your character is moderately to highly religious, and play them accordingly. 

\subsubsection{Morality and Ethics}
On a national scale, morality is dictated by religion, and governments endeavor to set rules and laws in place that reinforce this. On a personal level, many characters have opinions, not all of which align with the national and religious agenda. This game is absolutely meant to be a space where characters can engage in discussions about morality and ethics. However, we ask players to set aside some time to think seriously about the fact that your characters have grown up in this world, with very few dissenting voices. You should be prepared to argue for and support the status quo unless you have a specific reason outlined in your character sheet to question it.

We know that some of the morals that the fictional societies uphold are likely to be orthogonal, or even contradictory, to what some of our players believe in real life. We ask you to resist the urge to modify your character's opinions to match your own too early in the weekend. This phenomenon is common in games (sometimes called the ``21st century morality problem'’) and can very quickly cause tension and conflict to collapse. Many of these characters have hills they are willing to die on (sometimes literally), and the game will be more interesting if you are willing to stick to that until another character successfully challenges that worldview. 

\subsection{Content Warnings}
This game deals with a number of serious and potentially triggering or upsetting topics. We have done our best to provide a list below of the most widely reaching content. Almost every character will bump up against these. Some characters will have large swaths of their play built around them. Players are asked to treat these topics with the gravity they deserve.

\subsubsection{War}
There is a war going on in the world of this game. It has been going on for almost 6 years. Many characters have been directly or indirectly involved, and everyone has been impacted in some way. War is not a conflict we introduce lightly. A lot of people end up dead, or mentally or physically disabled as a result of war, even in a fictionalized world. But much like in real life, many people are convinced that it is necessary, or that it is just, or that there is no way out that saves face (and that saving face is important.) While there is an opportunity in game for the characters present to try to forge a cease-fire or a peace treaty, the ultimate decision to ratify a treaty does not rest with the players. Instead, characters can submit correspondence to be sent to their respective governments (managed by the GMs), who will ultimately approve or reject proposals. This structure exists, not to take agency away from players, but to avoid trivializing the solution. 

\subsubsection{Nationalism and Jingoism}
Aside from the immediate effects of war, nations have done everything they can to amp up national pride, and a sense of righteousness of their own cause. Leaders in every country are happy to villainize those from elsewhere, to create a boogeyman that keeps the fear and anger that started the war burning hot. The students are somewhat insulated from these effects inside the \pSchool{}, but the advisors are bringing an extra-heavy dose of this into game.

\subsubsection{Religion and Spirituality}
We understand that not all of our players are religious or spiritual in their own lives, and that some folks may have trauma around organized religion (one of your GMs does). Being a part of this highly religious world may feel strange or uncomfortable, especially at first. We ask players to take particular care to avoid sharing a cavalier attitude toward religion and spirituality during and around this game, or to assume that ``everyone'' believes something or other out of character. \textbf{Players} should avoid proselytizing while out of character, whether about religion, spirituality, atheism, or other. There isn't much proselytizing in character either; each nation is happy to have new people join their sect, but generally respect each character's choice of religion. Let people have their own lives, do not assume that their character attitudes reflect their player attitude, and do not offer unsolicited advice or opinions about religion or spirituality.

\subsubsection{Murder}
In the world of \pEarth{}, the Gods declared murder the ultimate crime, and enforce punishment themselves. The person who took another life loses their memories. Pretty much all memories, usually immediately, regardless of the circumstances. Some characters are interacting closely with plots that more directly address this, others are more removed, but it is a fact of this world. There are mechanical consequences for killing someone in game. See the section on combat, subsection ``Killing Blows and Character Death'’ for more information.

\subsubsection{Memory Loss}
Even though character memory loss is a possibility, ableism is \textbf{not} part of this game. Characters can be suspicious of another character claiming not to remember things, but you should be careful about jumping to the conclusion that it is divine amnesia; plenty of people lie all the time about not remembering one thing or another for a variety of reasons. It is also important to distinguish that some \emph{players} may have difficulty remembering things. Be kind and patient with each other if someone asks for you to clarify or repeat something you said previously, or needs to refer to a game document.

\subsubsection{Potential Apocalypse}
It may be possible, through certain character actions or inaction, to cause an apocalypse level event that will drastically change the course of \pEarth{}n history. No change so drastic can come without significant fallout among many inhabitants, and the worst parts of it are likely to fall on the most marginalized communities even if they are specifically prepared (for example: if they are instigating the change.) Any decision you make that drives toward something like this should be made bearing this weight in mind.

\textcolor{red}{An Apocalypse scenario would include imminent risk of death for the player characters. While not every character would die in such a situation, many characters could, and an individual character's survival may be outside of the full control of the player of that character. We encourage you to consider this content warning in the context of the game style, which includes mutually exclusive goals and character vs. character conflict that does not have pre-negotiated resolutions.}

\subsubsection{Classism}
The world of \pEarth{} is riddled with direct and indirect classism. Some characters are more cognizant of this than others, and some have significant levels of privilege. Classism figures heavily in the backstory and motivations of several characters, and some of the plots and stories in game center around trying to address this in either an individual or a systemic way. Addressing classism is not a trivial thing, and players should be cautious about trying to ``white-knight'’ a solution by trying to fix a perceived problem without any input from the marginalized people they are attempting to help.

\subsubsection{Content Warnings impacting PART of game}
\begin{itemize}
  	\item Family issues. Some characters have difficult relationships with biological or non-biological family members, including estrangement.
	\item Established Romance -- some characters are in established romantic relationships; at the discretion of the players involved, these can be re-imagined as platonic life partnerships.
	\item Potential Relationships -- some characters are interested in exploring new romantic or platonic relationships with other characters. None of the feelings would necessarily be unwelcome, but some characters may not realize yet that someone else is interested in a relationship with them.
	\item Polyamory -- Some of the existing or potential relationships involve more than two consenting adults.
	\item Cheating or infidelity -- some characters may not be honest in their relationships.
	\item Harm to animals -- the Avatars of the Gods typically take the form of sentient animals. Some characters may wish to do harm to those specific animals due to their avatar nature.
	\item Immortality and Death -- characters grappling with what it means to be immortal.
	\item Tunnel Vision -- characters grappling with a single goal that overrides all else.
	\item \textcolor{red}{Character to character competition. Specifically, there is a mechanic in which the students are being ranked in something akin to a popularity contest by the teachers and advisors, (not by their fellow students).}

\end{itemize}

\subsubsection{Content Warnings NOT in game}
Due to the game design, we can provide the following list of topics and tropes that will \textbf{not} be part of the game material, and shouldn't come up during play. This is not a complete list, and if you have questions or want to check about a particular other topic, please reach out and ask.

\begin{multicols}{2}
\begin{itemize}
  	\item LGBTQIA+ Discrimination.
	\item Fatphobia / fat shaming.
	\item Sexism.
\item Unwanted sexual attention (harassment, assault, etc.)
	\item Ableism (other than suspicion toward individuals with significant memory loss.)
	\item No characters are soldiers or police, but some characters have positions of authority over other characters.
	\item Medical procedures. All healing is abstracted, and is not meant to support extended graphic roleplay scenes around healing or other medical procedures.
	\item Pregnancy or pregnancy loss.
	\item Substance abuse.
	\item Partner or child abuse.
	\item Pandemics
\end{itemize}
\end{multicols}

\textbf{A note about sexual content:} This game has no pre-written sexual content, and no sex mechanic. Players are required to obtain explicit, enthusiastic consent from everyone present in a space before engaging in any such play (not just from the players about to engage with the content). Players are allowed to withdraw their consent at any time, and you are required to comply with any request to relocate such play elsewhere.

Once you get your character, please stay within the bounds outlined by that sheet and the game content in general. Do not invent edgelord backstory elements, or introduce new plot points. This is important for the integrity of the game. If you invent something new and abandon content written for the character, you leave the other characters involved in that content out in the cold. This is also crucial for the safety of your fellow players since they didn't get a chance to opt into or out of the new content. If something isn't working for you, get in touch with the GM team as soon as possible, and we'll work together to find a solution.

\section{Safety}
This is only a game.  Everyone involved should act with courtesy, sportsmanship, patience, and taste.  The GMs may expel anyone they believe to be violating the spirit of the rules or the game.  Emotions may run high. If you think things are crossing the line from game to reality too much, or if you are just getting too stressed, take a break. Always, play safely, then play to have fun. 

Real violence is unacceptable. Game action should cause no real-world damage, either to people or property. If something dangerous is happening, call a halt. Stay in control, use common sense, and do not endanger yourself or others.

Safety is a shared responsibility in the community, between all of the participants, players and GM alike. Only you can determine if you need to step away from a scene, plot, player, etc. Only you can say whether you need a temporary or a permanent solution. But we can all be aware of the ways that we interact with others, and be willing to adjust our behavior based on feedback for the safety and comfort of our fellow participants.

\subsection{Covid-19 Safety}
The Covid-19 safety policy for this game is described in this google doc \url{https://docs.google.com/document/d/1x9ZlmUnXveMVTHniRa3M_B84zEZrIFyuW3c05wYeBOY/edit?usp=sharing}

\subsection{Calibration}
\textbf{Players are more important than the game.} In this game, this adage must be applied preemptively, not just when someone gets triggered or upset in the middle of game. Secrets and Powers games are fragile. While they can survive a character not being present, the game is significantly the poorer for it. This is due to the design concept that any character the game could run without should either be cut before casting, or rewritten to be more integral. This makes it way less likely that a player will be bored or feel left out, but it does pose important considerations for safety.

Calibration is handled primarily through casting. Please think carefully about your responses to the casting survey as we will use those to cast you to a pre-written character. Once you receive your character sheet, we ask you to read over it and let us know \textbf{as soon as possible} if there are aspects we need to change. Please don't just change things on your own without talking to the GMs; we need to re-balance game around the changes.

\textcolor{red}{\subsubsection{Expectation Management}}
\textcolor{red}{You may find it a valuable excersise to take a few minutes \textbf{before} arriving to the venue to ask yourself: What do you as a player hope to get out of this experience? What would consititue a sucessful or worthwhile event to you? Are there any experiences you wish to avoid, or patterns from previous games you wish to challenge? What actions can you take to help with these things?}

\textcolor{red}{\subsubsection{Player to Player calibration}}
\textcolor{red}{This game style generally relies on little to no player to player calibration. Characters are written to fall within the comfort levels indicated on casting surveys given most likely interactions with other characters. In situations where calibration is necessary, it should be accomplished in character if possible. If not possible, then the OOC calibration should be kept as short as possible. We want to allow as much as possible to unfold organically in game.}

\textcolor{red}{If you wish to do any pre-play calibration at dinner on Friday night, first ask ``are you open to pre-game calibration with me?'' without further detail.  Remember that other players may not be aware that they have a plot with your character, may not believe calibration is necessary (IC or OOC), or may prefer to go in without expectations. In such a case, respect your fellow players' choices and contact a GM if you want support for yourself.}

\textcolor{red}{\subsubsection{Negotiating Outcomes and Pre-Planning}}
\textbf{red}{This game style \textbf{does not support} negotiating outcomes or pre-planning. These games are very much ``play to discover'' and scheduling reveals, scenes, or activities limits that spontaneity and creates closed circles that other players cannot broach. In this, we very much ask players to ``trust the process'' and act like your character would act. Trust that your fellow players will respond in interesting ways that will enrich your story without you needing to know what that reaction is ahead of time.}

\subsubsection{Self Care}
\textbf{Players are more important than the game.} Take some time in the days or weeks leading up to game to ask yourself what you need to take care of yourself. We want you to have the best chance of having an enjoyable experience at game, and contingencies in case that doesn't happen. Reflect on the game scenario and the listed content warnings (see above). Reflect on your mental, physical, and emotional reserves, and what actions you can take to prepare care for yourself before game to boost your resiliency. Review the safety mechanics (see below) for this game. It is also a good idea to be prepared in case of drop or bleed afterward. See the ``Deroll and Debrief'’ section below for more information about these terms.

\subsection{Safety Mechanics}
When in doubt, use the universal Out of Game symbol of placing a fist on top of your head. Use this to go out of character and talk, or do whatever you need to as players to re-establish a safe play space, or address an out of character situation. 

Your GMs will be available during game to assist and facilitate as well. During game on hours, we will either have someone available at GM Headquarters, or leave a note with where to find us. We are not trained mediators or therapists, but we care about you and want you to have a good time. We can help you come up with plans for self comforting, and when you are ready, help you decide if and how you want to re-engage with game.

The following safety mechanics will be used in ``\gamename{}''. We will practice with these mechanics during workshops before game. These are for player safety. Deliberately ignoring another player's use of them will have disciplinary consequences.

\begin{tabularx}{\textwidth}{|>{\centering\arraybackslash} m{1.5cm} | >{\centering\arraybackslash} m{2.5cm} | >{\centering\arraybackslash}X |}
\hline 
 \textbf{Mechanic} & \textbf{How to Use} & \textbf{What happens next} \\
\hline
Out of Game Hand Signal & Make a fist with one hand. Place it on top of your head (visible from in front of and behind you) & You are now out of character (OOC). You may talk and act as your player. This is useful for many things including talking player to player about something that just happened that you need to adjust, stepping out to use the restroom, or because a mechanic told you to. If someone uses this around you, your character no longer sees theirs. You should ignore OOC players unless they address you directly and ask you to go OOC too to discuss something. If you need to have an extended conversation OOC, we encourage you to step away from ongoing game play so as not to disrupt it. \\
    \hline
``Okay Check In'’ & Make eye contact with the person you want to check with, and flash them the ``ok'’ symbol with your hand. They should respond with a ``thumbs up,'’ ``thumbs down,'’' or ``thumbs sideways'’ & Use this mechanic to check in with someone who appears to be in distress (e.g.: crying), who's demeanor has suddenly changed (e.g.: stopped talking and has a 50-yard stare), or you otherwise want to check in on whether the \textbf{player} under the character is doing okay. If you get a ``thumbs down'’ or ``thumbs sideways'’ response, pause game play and take a moment out of character to check in with the player. Encourage them to check in with themselves if they need to change anything about current play or the current environment. If you don't feel able to check in with them at that time, find a GM or send them to GM HQ.\\
    \hline
``Brake'’ (from ``Cut and Brake'’, sometimes used as ``Largo'’) & Say ``Brake,'’ and put both hands out flat, at shoulder height, palms away from you. Push out into the air once or twice, like pushing the brakes of a car. & Use this to de-escalate the intensity of an interaction for a player's safety or comfort. (e.g.: Ask a player to speak more quietly, even though their character is still yelling, or take a few steps back, even if their character is still physically blocking yours in.) This is not a negotiation - the person being asked to ``brake'' should immediately comply for the comfort and safety of their fellow players. \\
    \hline
``Game Halt'’ & Say ``Game Halt'’ in a loud enough voice for people right around the corner to hear you, but not so loud as to shout. End the halt once the issue is resolved by saying “3, 2, 1, Game On” &  Used to halt game play in a whole area, either for safety or a mechanic. This call should be used to halt game play to sort out a safety issue such as a player being unable to continue at the moment, or in need of a longer player to player discussion for which ``brake'’ is insufficient, and everyone in the scene needs to be involved. This call should also be used for physical danger, for example: repositioning players so no one is at risk of falling down a cliff. This call is also used by some mechanics to pause the game to fetch a GM for something. \\
    \hline
``Badge Off'’ & At a time when you are not in the middle of a game action, simply take your badge off & Use this to indicate that you are exiting game for the night, or need to take an extended break from game (longer that would be comfortable to maintain the Out of Game Hand Signal). If you are going``Badge Off'’ for a game related reason, feel free to come find a GM if we can be of any help. Players should assume that another player without their badge is out of character for an extended period. You may take no game actions toward someone who isn't wearing their character badge. If you are looking for their character, assume they cannot be found. \\
    \hline
``Open Door'’ & Physically leave the game space. & If you need to leave the game space, please do so. If you do so, PLEASE, let a GM know so we don't start a search and rescue operation assuming you are lost in the mountains somewhere. \\
    \hline
\end{tabularx}

\section{Game Schedule}
The current game	schedule is available in the ``\gWeekendSchedule{\MYname}'' greensheet. \url{https://drive.google.com/file/d/1Si4lgvwn6X5ha8Mp7DAVX5Vz6JOHKnuu/view?usp=sharing} \textbf{red}{This schedule is subject to change regarding what is scheduled when, especially regarding the order of pre-game workshops.}

\textcolor{red}{\paragraph{Storm Surges:} During the weekend there will be 3 Storm surges, during which characters will want to seek protection in one of the bunkers (the 3 lounges). Greensheets will be available in each game space to explain what happens if your character does not.}

\paragraph{Pre-Game Workshops:} There will be three \textbf{mandatory} pre-game workshops on Friday afternoon to help with building player rapport and establishing the shared contract of game.

\paragraph{Game Re-balancing:} There will be an \textbf{optional} OOC activity on Saturday Afternoon to recalibrate and redirect to improve your play experience.

\paragraph{Post Game Workshops:} Formal deroll and debrief are optional, but highly encouraged activities after the game ends. These activities help with \textbf{bleed}: character feelings affecting player experiences, in this case after the game (i.e.: a player is short with someone at work after an intense weekend game). These activities also help with \textbf{drop}: players feeling low emotions after interacting closely with people for several days and then going their separate ways.

\section{Getting Started}
\textcolor{red}{\subsection{Unless You Know Otherwise}
You will see the term ``unless you know otherwise'' scattered throughout this document and many other game documents. This phrase is a common short hand that means the general rules as written preclude something, but that does not mean that this thing is impossible. For example, some characters may have a greensheet, ability, or a character stat that allows them to do something that normal people can't. Ask a GM if you are not sure whehter something you have qualifies to allow you to bypass an ``unless you know otherwise.''}

\subsection{Character Packets}

Your character packet is a big manila envelope.  It contains your role: who you are, what you're trying to accomplish; everything about your part as a {\bf player-character} ({\bf PC}) in the game. Read all the contents and generally keep them with you during the game.If you are missing something or find something which doesn't seem to belong to you, tell one of the GMs.  Character packets are confidential. Game materials which cannot be given to other players are marked ``Not Transferable,'' whereas things which can be given to others are marked ``Freely Transferable'' or ``Game Item.''  \textcolor{red}{Do not show these items/documents to other players; you are always free to share the information through in-character roleplay once the game begins.}

Your Character Packet would normally contain:
\paragraph{Name-Badge:} A name-badge with your character name, player name, character description, and {\bf badge number} on it. Wearing this shows that you are in the game; wear it visibly.  It also represents your character's body in-game.  Badge numbers are not in-game information. See the \emph{Character Bodies} and \emph{Badge Numbers} sections for more details.

\paragraph{Character Sheet:} Your character sheet describes who you are and what you are up to.  It contains a list of everything else that should be in your character packet. Do not show or read your character sheet to other players.

\paragraph{Bluesheets:} A bluesheet describes information common to members of a group.  When in conflict, character sheet information overrides bluesheet information. Do not show or read a bluesheet to other players.

\paragraph{Greensheets:} A greensheet describes and expands abilities, mechanics, or in-game knowledge. Do not show or read a greensheet to other players.

\paragraph{Stat Card:} Your stat card lists your statistics, and is printed on pink or salmon colored paper. You might not know what all of your stats mean. Do not show your stats to others. The reverse side is a {\bf death report}; fill it out and give it to the GMs if your character dies.

\paragraph{Ability Cards:} An ability card explains a special ability your character has, and is printed on yellow paper. The front side (says ``Ability effect'’) describes the effects; show it to players when you use the ability.  The reverse is the rules of use and must not be shown to other players.

\paragraph{Memory/Event Packets:} A memory packet is an envelope or stapled piece of paper with a {\bf trigger} which describes when to open and read it. If the trigger is a number, open the packet when you see something with that number. If it's a quoted phrase, open when you hear or read it in-game.  If it's a symbol, open when instructed. Do not take game action based on an unopened trigger. Do not show or read a memory packet to other players.

\paragraph{Research Notebooks:} A research notebook is a little booklet of pages folded over and stapled shut. These packets guide you through creating or researching something taking many steps, in which the full process that will be required is not known at the start. \textbf{You should open the first page when game starts}. Each page will describe information you learn, and tell you what action to take next. When you complete the requested action, you may open the next page. Do not show or read a research notebook to other players. 

\emph{Exception:} Some research notebooks will list other characters as having the same notebook. In these cases, if the \textbf{characters} elect to share information, players may open pages to match.. I.e.: If player 1 has opened page 3, but player 2 has not, after conferring in character and agreeing to share information, player 2 may open pages up to and including page 3.

\paragraph{Items:} In-game items may be transferred from character to character, and should be marked as such.  See the \emph{Items Etc.} section for more details.

\paragraph{Whitesheets:} These are in-game items that are the size of a full piece of paper and represent things like documents and letters with specific content. Treated exactly like any other item.

%% Some Assassin Game fundamentals
\subsection{Game Concepts}

\paragraph{Not-Here:} Some mechanics may instruct you to go ``Not-Here.'’ You go not-here by turning your name-badge around so the ``I'm Not Here'' side is showing. Putting a hand on your head, visible from a distance, helps if you're near other players. When you are not-here, your character is not there.  Your character cannot see, hear, or remember any game actions or information you (the player) happen to encounter.  Avoid other characters, common game areas, game signs, or any sort of game interaction. Going ``Not-Here'' is a game mechanic. Doing so in front of other characters represents something like suddenly becoming invisible, and is distinct from the safety mechanics described above.

\paragraph{Non-Players:} Use tact and common sense when dealing with non-players ({\bf NPs}). Camp staff have their own lives, and other groups at the camp have their own events to attend. NPs may not knowingly affect the game. They and their rooms may not be used to hold items or information.

Avoid conspicuous or threatening game actions in front of NPs. Please keep conversations about murdering another character and other potentially triggering or upsetting topics out of earshot of NPs. Do \textbf{NOT} shout ``fire'' unless there is a real life, actual fire. If there is an in-game fire, you may use stage whispers. If, despite your most valiant efforts, some NPs do get upset, call the GMs who will help calm them down.

If you are about to take an action that would likely upset a nearby NP, you may call a game-halt and relocate the scene. This is considered an out-of-game issue.

\paragraph{Observers:} An observer is someone not playing the game who has agreed to watch.  They generally wear an observer headband or an observer name-badge.  Observers have traditionally been called ``ghosts.''  They should stay out of the way; you can always ask an observer to leave.  If a friend who is not playing wants to observe game, have them contact the GMs as soon as possible so we can arrange an observer ticket.

\paragraph{Mechanics:} Many actions your character can take, such as walking, talking, and general interaction with other characters, are represented by you doing them.  Others, like combat, are performed via abstract mechanics, which are described in ability cards, greensheets, and rules.  The abstract information for mechanics (like badge numbers) may not be discussed in-game.  If you want to do something special for which there is no mechanic, ask a GM.

Become familiar with your mechanics before game starts, especially those which occur under time-pressure (like combat), to save time in such situations.

Do not invent items or solutions to a problem that already has a mechanic. This is a kludge for game balance. A \textbf{kludge} is something impervious to logic and cleverness, usually for game-balance.  You can't affect a kludge without a specified mechanic, even if common sense tells you that in real life you could find a work-around. If a door says it needs a key to open, you cannot declare that you beat the door down, nor can you claim to have the key if you don't actually have the key item (use item numbers to verify if you have the correct item. More about item numbers below).  If there is no mechanic, plots should be resolved via \textbf{in character} discussion (aka: roleplaying). If you aren't sure, you can always check with a GM. 

{\bf Zone of Control} ({\bf ZoC}) is a rough distance measurement. You are within ZoC of someone if your outstretched fingers can touch their outstretched fingers.  Double-ZoC is twice this distance, triple-ZoC is three times, etc. You should not run or otherwise force your way into or through someone else's ZoC, and you should not make physical contact with another player without permission.

{\bf Headbands} represent obvious visual effects; wear them visibly on your head. If you see a headband and don't know what it represents, ask. If you are wearing a headband, tell people what their characters see.

An {\bf interruptible} mechanic has some duration, and may involve continuous roleplaying.  It is stopped if you are attacked or if someone within ZoC says {\bf ``I stop you''} or an equivalent phrase. A {\bf n-count} is an interruptible mechanic with a repeated, counted incant (``I pour a drink one, I pour a drink two, I pour a drink three'').  Speak clearly; each count must take at least a full second. Each n-count will specify the number, e.g. ``a 3-count''. Unless you know otherwise, another character can interrupt you simply by saying ``I stop you.'' or initiating an attack against you.

\textcolor{red}{Unless you know otherwise, any mechanic described as a ``Ritual'' is interruptible. If you aren't sure if something is interruptible, but you want interrupt it if you can, try it! Say ``I stop you.'' If the mechanic cannot be interrupted, the player(s) will briefly go OOC and tell you so.}

\textcolor{red}{\textbf{Mechanics with a timer:} Some mechanics will take a certain duration to complete. They may say to ``stand here for 30 seconds then do X'' or they may say ``roleplay doing this thing for 5 minutes, then proceed to the next step.'' These mechanics do not preclude simple actions like talking with someone else. Anything that meaningfully takes your attention away like walking away from the specified location or attacking someone (or being attacked) would disrupt you and require you to start the timer over.}

\textcolor{red}{A few mechanics will \textbf{specify} that talking disrupts them. Characters may still make a simple hand gesture (like holding a hand/finger up, or making a shoo-ing motion) without disrupting their timer, to try to encourage someone to go away.}

\textcolor{red}{Some mechanics will instead include a ``cool-down'' timer. In this case, instead of having to wait before a task is considered accomplished, you accomplish the task once immediately, but then must wait the specified duration before you can complete it again. }

\textcolor{red}{If any mechanic is inaccessible to you for any reason, see a GM and we will find a way to redesign it for you.}

\section{Items Etc.}

Many in-game items are represented by little white cards with a number and description.  Item cards may be shown to others, passed around, stolen, etc.  The {\bf item number} on the card is not in-game information and may not be discussed. Some mechanics in game may involve making item cards or writing things down on paper (essentially creating a ``whitesheet''). Such items should be clearly marked as ``in game'' and treated as such.

Use common sense. You can't carry a hundred rocks in your pocket, fold a sword in half, or hide a life-sized statue in a fire hose. You can't stop a bullet with a set of blueprints or rip apart a metal safe with your bare hands.  Even if your bag can carry a shovel in it, the shovel noticeably sticks out (``you see a shovel sticking out of my bag'').

\paragraph{Written Information:} If you write in-game information down on a piece of paper, that paper is now an in-game item and must be clearly marked as such. Don't write in-game information on out-of-game documents (character sheet, etc.). Don't write out-of-game information (like memory packet triggers) on in-game documents. Paper and writing implements will be available at GM headquarters.

\paragraph{Envelopes:} Some items and locations may have an attached envelope (or just be a labeled packet or folded paper). The envelope may include directions for when to open these (``open packet if you press the big red button'' or ``open packet if you eat this''); otherwise you may only open them if instructed.  Close them when you are done.  Open and close packets gently.

\paragraph{Signs:} Some locations and other game materials are represented by signs or packets posted throughout game area. You may read any signs and must follow any rules printed on them. If a sign or packet doesn't have some sort of in-game description (it only has out-of-game mechanics information, like a number or just a colored dot), then your character doesn't even see it or know that anything unusual is there.

\textcolor{red}{Some signs may have items or game documents associated with them. They will have envelopes attached to them. The sign will have instructions on how to access what is in the envelope. Most will have either a wait time or a randomization mechanic to represent time searching for something useful, or will have a cool-down time before you can return and take another item from the envelope. Some signs will allow you to search the envelope for a specific item. Others will require you take at random, read the instructions carefully.}

\textcolor{red}{By default you cannot store items in a location represented by a sign. Only some signs represent locations where you can store things. Most of these will have a maximum bulkiness that can be stored there; you will need to check the items already present to determine if there is room to store the additional item.}

\textcolor{red}{If you just drew something from an envelope and don't want it, you can put it back (but you may not pick a new item to replace it - you are just rejecting what you found).}

\paragraph{Bulkiness:} A bulky item is too big or heavy to be carried or concealed freely.  Bulkiness is measured in {\bf hands} (how many hands it takes to carry it).  \textcolor{red}{\textbf{If you are carrying a bulky item, hold the item-card and any phys-rep in your hands/arms)}}.  A hand carrying a bulky object may do nothing else.  With one hand less than required (e.g. you have 2 hands, but wish to carry 3-hands bulky worth of items), you may drag the item(s) at a slow pace (traditionally by walking ``heel-toe'’).

\paragraph{Phys Reps:} Short for ``Physical Representations,'' aka: props. Some items may have props associated with them.  The card and the prop should be kept together. \textbf{If they are separated, the card is the real item}. If you bring your own prop to represent an item, you \textbf{must} attach the item card and display it prominently. The bulkiness of the item is defined by the card, not by the prop.

\paragraph{Unstashable Items:} Unstashable items can't be hidden or left behind. They look too important, valuable, or interesting; NPCs will not let them stay there. These include any item that has a prop. This is a kludge. If you're not leaving an unstashable item in another PC's care, and you want to leave it behind, give it to a GM or observer. You may leave it in plain sight in a public area if there are other PCs around.

\paragraph{Character Bodies:} A body is {\bf three hands bulky} and usually represented by a name-badge.  It must be willing or unable to resist for you to carry it.  Carry the badge conspicuously. Onlookers can't tell if it's dead without close examination, unless it would be obvious (like headless). If you are carrying a PC, and that player is available, they should walk along with you. Please do not actually attempt to physically carry another PC.

\textcolor{red}{\textbf{Destroying Items:} Unless you know otherwise, items, whitesheets, etc \textbf{cannot be destroyed}. This is a kludge. Some mechanics will involve ``consuming'' items. Please discard these items to the nearest stock vessel (or give them to the nearest GM). This will save time and paper for items that are unlimited for the purposes of game without us having to print a thousand copies just in case.}

\subsection{Searching, Stashing, and Stealing}

\paragraph{Places:} To search a place, search it.  Normal items can be stashed in any reasonable, legal place. Don't put items behind locked doors, inside ceilings, in construction sites, or in other dangerous places; consequently, don't go rummaging through such places for game items.  Don't stash or search in places that are not in-game; see the \emph{Game Areas} section for more information.

\paragraph{Doors and Locks:} Some doors or items in game are \emph{locked}. You may not open them or get past them unless you fit the requirements listed, or have some other method of opening locks. Closing such an item or door locks it again.

\paragraph{People:} All searches of characters or their belongings are conducted via player dialogue. There is to be no actual physical contact during searches. A character must be willing or unable to resist for you to search them.  You need at least one free hand to search someone. Searching is interruptible (see above).
A search reveals all in-game items, and takes as long as your victim spends handing over possessions. If you're the victim, hand over items at a reasonable pace.

\paragraph{Bags:} To search a bag in someone's possession, say ``I search your bag.''  This proceeds just as a search of a character. To search an unattended bag, search the prop bag. Don't look through someone's personal or out of game items. If the bag has an attached, displayed item card with an envelope, the bag is a prop; search the envelope and not the bag.

If you want to leave in-game items in an unattended bag (e.g.\ to hide a bomb), keep items in reasonable places that could be found with a quick search of the bag. Don't hide in-game materials mixed together with out-of-game materials. You can attach an item card and envelope to segregate in-game items from out-of-game materials.

\section{Violence, Damage, and Death}
For those familiar with the ``Darkwater Combat'' system from other games, there is no martial combat in this game. Instead, all combat is fought with magic. In \pEarth{}, magic is so ubiquitous, and so powerful compared to physically attacking someone, that no adult would ever consider throwing a punch. Young children are very quickly taught that physically fighting is unacceptable in any and all circumstances.

\subsection{Health States}

Characters have four possible states, concerning health and damage. When you are {\bf fine}, you may act freely. When you are {\bf restrained}, you are helpless and may do nothing but talk.  When you are {\bf knocked out}, you will wake up in five minutes. When you are {\bf dead}, you are dead.

When knocked out, your character falls down and drops anything they are holding. As a player, you may sit or lie on the ground, or just tell people OOC that you are knocked out. You won't be doing much of anything until you wake up. Do not listen to conversations going on. If you are restrained however, you remain conscious and can talk and listen, but you cannot move on your own. An unrestrained character can move you.

Dead men tell no tales. If dead, do not give out any information about your character or death to any players. You may remain on the scene to play the part of your corpse; describe obvious information to onlookers (``I have a fireball burn on my back''). When you leave, place the front of your name-badge with a description of the body's obvious state. Take the ``I'm Not Here'' side to wear. Stack your items with your body. Fill out your Death Report. Make sure the GMs know about your death. If your death becomes generally known to the other characters, you may be able to become an observer. Until the game is over, you may not convey game information to any player.

\subsection{Weapons}
Every character is assumed to have a small arcane focus of their choice on their person. This is purely for flavor - you will not have an item card for it, and while it could theoretically be confiscated by someone, your characters can cast magic just as well without it (so for simplicity, they cannot be taken from you). You may choose to bring a prop for your arcane focus as part of your costume if you so desire.

There may be some items in game that can serve as weapons. These can be confiscated like any other item.

\subsection{Magical Combat}
All characters have a {\bf Combat Rating} ({\bf CR}) stat. This represents your basic skill in magical combat. Someone with a CR of 1 can't fight with magic very well. Someone with a CR of 3 is magically powerful or an experienced combatant. Someone with a CR of 5 is magically powerful AND an experienced combatant. 

When using this stat, you may hold back by using a lower number. Some abilities may modify your CR temporarily or permanently for attacking, resisting, or both.

\paragraph{Attack Intention} All characters may have one of three intentions when initiating an attack. 
\begin{itemize}
	\item They may wish to ``knock out'’ their target, which if they succeed will knock the target unconscious for 5 minutes. 
	\item They may wish to ``upstage'’ their target, which if they succeed simply means that they have demonstrated your superior magical abilities in some way (i.e.: by embarrassing them), but with no additional mechanical effects. Use ``upstage'’ for demonstrations where no one wants to cause harm, sparring, attempting practical jokes, to issue threats, etc. 
	\item \textcolor{red}{They may wish to ``restrain'' the target, which if they succeed means that the target can no longer move on their own. \textbf{However,} due to everyone's inherent magical abilities, no rope will hold them. People only remain restained as long as someone is actively holding them. To continue to restrain someone occupies \textbf{2} hands, either 1 from each of 2 people, or both hands from a single person. This mechanic is the same as carrying ``bulky'' items. You have only 2 hands for such tasks. If you wish to move a restrained person, they are considered a ``body'' and are thus 3-hands bulky. (see the section above on ``bulkiness'' for more details.)}
\end{itemize}

\paragraph{Initiating an Attack:} To attack someone, clearly state your attack and CR (``\aKnockOut{} 2'') from within 1 ZoC unless you know otherwise (the ability will state an increased range of attack). Your attack must resolve before you make another; otherwise, you may act freely. 

\paragraph{Assisting:} If an ally directs {\bf \aAssist{}} at you after you attack, you may, within 2 seconds, restate your attack with the \aAssist{}'s CR added (``\aKnockOut{} 3'', ``\aAssist{} 2'', ``\aKnockOut{} 5''). You may not use \aAssist{} to assist someone's defense. The best you can do is try to assist them in any counter attack they choose to make, or make your own attack on their assailant. You may ignore an \aAssist{} if you so choose.

\paragraph{Defending:} When attacked, resolve by comparing the attack against your CR.  If your CR is lower, take the effects; else, say ``{\bf resist}''. Mechanically the attack has no effect. You cannot ignore an attack; if your CR is too low to resist the attack, you must take the effect, else say ``resist'’ in a timely manner so that combat may continue. You can defend no matter how far away the attack comes from.

The attack begins when the player beings speaking; all of your actions other than counterattacks are interrupted. Serial attacks don't prevent simple actions (talking, weapon-drawing, etc.) in-between. Resolve all attacks alone, in the order they occur; agree on the order if it is unclear.

\paragraph{Cinematic Combat:} Once the Attacker's CR (plus any assists they choose to accept) has been compared to the Defender's CR, the Attacker should briefly narrate (in no more than a few words) what the attack looks like (i.e.: I throw a fireball at you.) If the defender is able to ``resist'’ they should briefly describe how the attack was dodged, parried, blocked, or otherwise nullified. If the defender cannot, or chooses not to resist, they should take the effect of the attack.

\subsection{Stealth}

Stealth abilities represent sneaking up on a victim with obvious intent to invade their personal space, probably to attack them by surprise or to pick their pocket. Everyone has the ability to attempt a waylay attack (see below), but only some characters have additional stealth abilities.

To use a stealth ability, you must be within 1 ZoC of your victim. Form a ``llama'' with your hand (index and pinky fingers extended, thumb, middle and ring fingers touching) and extend it along the direct, unobstructed line from your shoulder to the victim's head. Hold this position for the time specified by your ability. Before this time is up, the ability is thwarted if anyone attacks you or if the victim notices the symbol. If they react in any way to the symbol, they have noticed; you (the attacker) make the call. \textcolor{red}{Any mechanic that requires this action is a suspicious activity, and characters who see you doing this may not think highly of it, or you.}

If you notice someone using a stealth ability on you, make it obvious. ``I notice you'' is unambiguous; use it if you can. Once a stealth ability is finished, you may not retroactively have noticed.

\paragraph{Waylay:} Anyone can attempt to attack by surprise as a stealth ability. Even if you want to use a magical attack that allows for attacking from further away, waylay must always happen from within 1 ZoC. You must hold the symbol for \textbf{five seconds} (count silently). If you succeed, you may replace your CR with ``waylay'' for a single immediate attack on your victim. If you are hit with a ``waylay attack'’, you do not have the opportunity to resist; you are surprised and the attack just works.

\subsection{Killing Blows and Character Death}

This weekend is a particular focus for the attention of the Gods. They will pay attention for as long as they can, until the growing power of the storm drives them away. \textbf{From Game Start to 6:00 pm on Saturday Night,} the attention of the Gods will make it impossible to kill anyone present at the \pSc{}. You may attempt to do so using the ``Killing Blow'' mechanic, but it will fail dramatically, even if you complete it successfully. You should feel free to roleplay this as impressively as you like. After 6:00 pm on Saturday night, the strength of the Storm will be too strong, and will drive the influence of the Gods far enough away for characters to be able to kill each other if they willing to take the consequences therein.

The actual mechanic of a ``Killing Blow'' is an ``n-count'' of ten. To perform a killing blow: point at your target and incant ``killing blow 1,'’ ``killing blow 2,'’ etc. up to ``killing blow 10.'’ Speak loudly enough that anyone in the room with you could hear you, and slowly enough that each phrase takes at least 1 second. Anyone conscious and able to act within 1 ZoC of you may at any point before you reach ``10'’ say ``I stop you,'’ including your target. You are also interrupted if you are attacked. At which point, your character has been interrupted, and you must start over.

You expect that if you succed in killing another character, your character will immediately lose all memory of who they are and what they were trying to accomplish. Resolve any immediate roleplay in the scene, then ASAP, go to GM headquarters and take a copy of the appropriate information from the sign ``\sMurderConsequences{}.''

If you character is killed during game, \textbf{you MUST play the body for at least 10 minutes.} There may be some mechanics that utilize dead bodies. After the 10 minutes has elapsed, leave your name badge on the ground to represent your body, and go OOC (fist on head). Go to the GM HQ and take a greensheet from the sign ``\sMurdered{}.''

\section{Miscellaneous}

\paragraph{Badge Numbers:} The first digit of your badge number is your character's apparent age in decades.

\subsection{Stickers}
\textcolor{red}{No mechanics require stickers at this time.}

\section{Important Rules Reminders for \gamename}
This section attempts to summarize the changes from the ``standard'' rules that are in effect for this game, but we cannot guarantee that it is comprehensive. Even veteran players should read the whole rules document carefully to make sure you understand the context for these changes.

\subsection{Safety Mechanics}
The player is more important than the game. If all else fails, drop out of character and take a deep breath. You can step away from the situation and process on your own, or with a GM. When you are ready, you can come back and talk it out player to player, with or without a GM for support.

The safety mechanics we are using for this game are: The Out of Game hand signal, Okay Check-in, Brake, Game Halt, Badge off, and Open Door.

\textcolor{red}{\subsection{Expectation Management}}
\textcolor{red}{You may find it a valuable excersise to take a few minutes \textbf{before} arriving to the venue to ask yourself: What do you as a player hope to get out of this experience? What would consititue a sucessful or worthwhile event to you? Are there any experiences you wish to avoid, or patterns from previous games you wish to challenge? What actions can you take to help with these things?}

\subsection{Pre and Post Game Workshops}
The pre game workshops are a required part of this game; post game workshops are optional. Players are expected to be on site from \textbf{12:00 pm on Friday to 5 pm on Sunday.} You must make arrangements with the GMs well in advance if that will not be possible.

\subsection{Combat}
There are no wound attacks in game. To keep someone restrained requires 2 hands. Some abilities will allow you to attack from beyond 1 ZoC, but you can always attempt to defend, no matter how far away the attack comes from.

Killing blows are disallowed until 6:00 pm on Saturday. After that point, while killing characters is allowed, there are mechanical consequences for doing so.

\textcolor{red}{\subsection{Magic:}}
\textcolor{red}{Magic is everywhere in this game. All combat is magic. Curses are magic. Building technology that works is magic. Growing food is magic. the avatars of the Deities are magic. Most of your abilities are powered by magic either overtly or subtly. The voting stones are magic. The relics are magic. And all of the rituals in game represent groups casting magic together.}

\textcolor{red}{\subsubsection{Curses:}}
\textcolor{red}{Some characters have the ability to create curses (and cures for those curses, but ``curse'' is the umbrella term). These are represented by item cards or item-envelopes that describe their effects. Some curses will have associated abilities or greensheets. Curses begin as latent magical effects, but can be activated in one of several ways. Once activated, write ``expended'' on the card and the person upon whom the curse is active should carry it around with them to remind themselves of the effect until it ends. ``Expended'' curses no longer represent physical items, they are just OOC reminders.}

\textcolor{red}{\subsubsection{Activating a Curse:}}
\textcolor{red}{Unless you know otherwise, you may:}
\begin{itemize}
	\textcolor{red}
	\item You may chose to activate a curse on yourself, as long as you are not unconcious or restrained. Write ``expended'' on the item and hold onto it as a reminder.
	\item You may activate a curse on a willing target. Write ``expended'' on it and hand them the item as a reminder.
	\item You may activate a curse on an unconscious, restrained, or upstaged target. Write ``expended'' on it and hand them the item as a reminder.
	\item You may use a waylay with a 5-count to secretly activate a curse on someone. (hand the item to a GM. we will pass it on to your target as quickly as we can while avoiding casting suspicion on you. You may even hand the curse to a GM pre-emptively if you are about to attempt the waylay.)
	\item Some mechanics may specify other ways to activate a specific curse on someone.
\end{itemize}

\textcolor{red}{Some curses may specify more stringent requirements for activation. Such items will say so. e.g. ``you cannot activate this curse unless you know otherwise.''}

\textcolor{red}{\subsubsubsection{Maker's Marks:}}
\textcolor{red}{Every magic user has a unique signature to their magic. For the purposes of game, this is most relevant regarding the construction of curses. Every curse will have a maker's mark on it, and the mark represents in-game information. You do not automatically know who's mark it is, but you could compare two curses to see if they share the same mark. Highly skilled curse makers have the ability to disguise their makers mark sometimes, making it look like a random other maker's mark. They cannot normally copy someone else's mark, unless they have an example of that mark in hand at the time of constructing a new curse.}

\textcolor{red}{\subsection{Randomization Mechanics:}}
\textcolor{red}{Several mechanics in this game use D20 dice for a randomization mechanic. The GMs will have D20s available for your use during game (and to keep afterwards.) You will want to keep this die handy as it may be called for at unexpected times.}



\section{Closing Notes}

These rules are imperfect.  The GMs may violate the letter of the rules to preserve the spirit.  We hope these rules are reasonably clear, but if you have any doubts about your interpretation, talk it over with us in advance.  We GMs are human too: when all of our carefully laid plans are going haywire, we may lose our cool.  The best way to deal with people is remaining calm and friendly, especially if everyone is tired and hungry.

We hope you have lots of fun.  Good luck.

\end{document}


