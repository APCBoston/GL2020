\documentclass[blue]{GL2020}
\parindent=0pt
\begin{document}
\name{\bAdvisorBlue{}}

Being an Advisor sent to the \pSchool{} during the Time of Deciding is a tremendous honor and a tremendous responsibility. You are here to protect the life and livelihoods of your nation, and maybe forge a better future for all of \pEarth{}. Many of you have other tasks as well, not so grand or sweeping, but equally important in their niches. The roster of advisors turns over quite regularly. It is not common for an advisor to attend more than a few Times of Deciding in a row; \cDiplomat{\full} and \cHeadDiplomat{\full} are the notable exceptions here.

Advisors arrive at the \pSc{} just before the brewing Storm cuts off travel, some time in the mid to late morning on Friday. During a normal Time of Deciding, your one and only task is to ensure that the Storm is sent to the proper place. For the years of the Time of Peace treaty between all three nations, this was a relatively simple task of reminding the students of their duty, and then doing your best to relax. But that lax attitude paved the way for the breaking of the treaty, and now you face a much more difficult task.

There is no clear answer where the Storm should be sent this time. Even if you adhere to the basic principle, ``not to my home nation,'' you still have two choices. And then there is the matter that you don't \emph{technically} get to decide. It's the students themselves who will vote for where to send the Storm. All you can do is try to persuade, advise, cajole, and, if necessary, threaten and blackmail.

You have a second role in this Time of Deciding as well. This is the first time in 3 years that advisors from all three nations are gathered in the same place. This could be an opportunity to forge a new peace. If both advisors and the governments they represent can be convinced, that is. Impassioned speeches to your fellow advisors here will not be enough -- you will have to find ways to persuade the ruling bodies of the nations that your solution has merit as well. They will not let you wave away things your nation believes is owed to it. A new treaty will be hard, but not impossible. After all, you have the very best advisors and diplomats in all of \pEarth{} here.

Some of you have personal business with each other, and the teachers and students, for good or for ill. Some have pet projects that require resources and knowledge found only at the school. And you all have personal and national reputations to uphold -- proving that you are better than your uncouth neighbors.

While the immediate future of \pEarth{} rests with the students, the greater future, spanning years or decades, is yours to shape. What are you willing to trade away to get what you want? Is there a price too high to protect your people? What will your legacy be?

\begin{itemz}[Goals]
	\item Assess the potential of the students from your nation for which one should have the greatest voting authority. Many advisors may find this synonymous with ``the one most likely to vote the way I want.'' Submit your ranking by 9 pm on Saturday night.
	\item Negotiate with the other advisors and see if you can hammer out a new treaty that your home governments are willing to ratify -- or have your nation go it alone if there is nothing to be gained from a treaty.
	\item Protect the interests of your nation, and uphold its dignity.
\end{itemz}

\end{document}
