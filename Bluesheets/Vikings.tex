\documentclass[blue]{GL2020}
\parindent=0pt
\begin{document}
\name{\bVikings{}}

\section*{Basic History}


\subsection*{The Storms}
In other nations, the people can hunker down and wait out the storm. In \pShip{}, that isn't really possible. Ships are as much at risk of damage in port as they are at sea. Coastal towns can be torn from the cliffside by the massive waves, or washed away in the floods from the rain. The storms stir up mud from the bottom that drive fish from their normal migrating patterns, making food much harder to come by. The storms also seem to attract sea serpents, with the largest (and most dangerous) specimens entering shallow waters in the wake of the Storm.

Despite the \pShip{} \cEbb{\God}es being connected with rain and the sea, the Storms are still devastating. \cEbb{} and \cFlow{} are powerless to substantially impact the storm. It's all part of the great balance. Magic necessitates Storms. Fighting off the serpents and uniting the \pShip{} nation requires magic. Therefore, in order to preserve their way of life\ldots Somewhere around here, the logic of the argument starts to break down somewhat. More so since the treaty was broken and the \pShippies{} way of life was shipwrecked. %%Publically known that this is the price of magic?

Six years ago, it was \pTech{}'s turn to have the storm directed at them. By some means however, the storm was directed at the \pShip{} instead. %%How publically known is it why storm was sent at the L'eau?
The damage was tremendous. Unprepared for the storm to hit them, the \pShippies{} were about their regular business. More than half the fleet was destroyed in the initial days.  %%How long does the storm last? How long does it take to travel from place to place?
With fishing patterns disrupted, their fleet in tatters, and the sea serpents agitated, the \pShip{} struggled to feed themselves those first few months after.

\subsection*{Politics/International Relations}
Up until 6 years ago, the \pShip{} were on good terms with both the \pFarm{} and \pTech{}. The \pShippies{} were available for hire to transport goods, or to chase off sea serpents. For the most part they kept to their own and more or less lived and let lived. That all changed with the breaking of the treaty. Calls for isolationism reached the Council of Stormwatchers from almost every Fleet within 4 months of the Storm. One voice in particular, \cLoud{}, enjoyed a metoric rise to prominence on unifying these cries. Very few in the nation doubt that ultimately \pFarm{} and \pTech{} do not value \pShip{} lives as they should, and that the only recourse is to protect themselves by withdrawing from the global stage. The scant few voices that argued otherwise fell silent 3 years ago, when the storm was sent to \pShip{} territory again.

\section*{Geography}
The \pShip{} are mostly a waterfaring folk. They travel Cengea's oceans, at least as far out as the continental shelf extends. hHe waters beyond that harbor sea serpents beyond the ken of even the most experienced \pShippies{} crew. No ship that has ventured out beyond the shelf has returned to tell the tale. \emph{Exploration beyond the edge of the continental shelf is beyond the scope of this game.} The \pShippies{} also float Cengea's rivers and lakes, using them as highways for travel, moving goods and people from the other two nations as well as their own.

Most \pShip{} towns are ports, and cling to the coastline, no matter how rugged. Only on the eastern side of Cengea do the \pShippies{} claim any land in the interior, and most of it is a massive flood-plain. Around the rest of the continent, the \pShip{} control small stretches of coastline, and almost every island large enough to support a human settlement. The only area of ocean that \pShippies{} fear to sail is \pWod{}. \pWod{} is due west of Cengea. It is an area of particularly turbulent waters, with unpredictable currents and jagged rock pillars looming out of the ever-present mist. The source of the unsettled waters is a massive whirlpool at the heart of \pWod{}. A few of the bravest - or most desperate - crews will work \pWod{}, scavenging from the shipwrecks. Ships are driven into the \pWod{} by sea serpents on a regular basis, leading to the prevailing theory among the \pShippies{} that it is a place fundamentally opposed to their dieties.

\section*{Economics and Industry}
The economy of \pShip{} is primarily one of trade. Within the country, trade is always good between ships and port, and ship to ship, as folks trade food, equipment, and luxuries from all around the continent. Economic interactions with the other two countries were primarily the charging and collecting of delivery fees as the \pShippies{} transported goods and materials. In the years since the isolationism began, the economy has suffered some, as there is no longer an influx of outside wealth, but compared to the damage done by two storms in a row, the loss is negligible.

By leaps and bounds, the \pShippies{} can build ships like no other. Whether by lack of necessity on the part of other countries, or some influence from \cEbbFull{} a ship built by the \pShippies{} will always be higher quality, no matter it's intended purpose. Ropes, nets, and other sea-faring gear is much the same. \pShippies{} are also the only nation meaningfully engaged in fishing. Hunting the nomadic shoals of fishes, and dodging or fighting off the sea serpents requires strength, skill, and luck that none beyond the \pShippies{} possess.

Both \pFarm{} and \pTech{} control the majority of the coastline that borders their interior holdings, and \pFarm{} have many rivers and lakes. All are infested with sea serpents. The other two countries regularly hire \pShippies{} to clear out lakes, defend a port, or to escort travelers. A talented crew can make good money doing this - as long as they don't get killed on the job.

\section*{Form of Government} The \pShippies{} have a consensus based representational democracy. The government consists of four levels:

\begin{enumerate}	
	\item The {\bf``Ships'':} The ships are the smallest unit within the \pShip{} government. Each ``Ship'' is comprised, literally, of the crew of each ship. The term also applies to the residents of a house in a town or city (most \pShip{} houses are communal living complexes, with strong internal communities and shared crafts - e.g.: a house made up of a dozen or more weavers). Each ship is in a way the law unto itself. They may make their own rules as long as they do not violate edicts from the larger governing bodies (e.g.: The ``Raft''). A ship may also elevate an issue that goes beyond their own rails to the ``Raft'' and call a meeting of those representatives to discuss it. When the Storm-Watchers declare something in need of discussion, they send messages (via dolphin or gull in modern times.) All members of the crew, 14 years and older are allowed to participate in the discussion, which continues until consensus has been reached. Once it is reached, a representative (traditionally the first mate) is sent to the ``Raft'' to carry on the next round of discussion and consensus building.
		\item The {\bf ``Rafts'':} A ``Raft'' is composed of all of the ``Ships'' that call the same port home. Houses belong to their nearest port that is in the same ``Fleet''; this is usually the city the house is built in, as the \pShippies{} have very few inland towns. A representative from each ``Ship'' will be sent to a predetermined meeting point for the ``Raft'' when it is necessary to discuss matters of policy, either brought up from a ``Ship'', or handed down from the ``Council of Storm-Watchers''.
		\item The {\bf``Fleets:'':} A ``Fleet'' is composed of all of the ``Rafts'' from a geographic area. Each geographic area is identified by a 1/12th arc of a clock, centered in the middle of the continent. The sections are numbered 1-12, starting N to NE Whichever slice the home port is in, the ``Raft'' belongs to that ``Fleet'', regardless of where the ``Ship'' is currently. A representative from each ``Raft'' will be sent to a predetermined meeting point for the ``Fleet'' when it is necessary to discuss matters of policy from either above or below. When necessary, a representative from the ``Fleet'' will be sent to the ``Council of Storm-Watchers'' when consensus is reached.
		\item The {\bf``Council of Storm-Watchers'':} The standing high council of the \pShip{} nation. This group of 12 elected officials (one for each ``Fleet'') holds office for 3 years. Council-members may not hold consecutive terms, and may not hold more than 3 terms in their life. One position is up for election each season of the year, meaning that in any 3 years, the whole council will have turned over. Storm-Watchers are tasked with identifying topics that must be discussed by the nation as a whole so that policy may be developed. When such policy is identified and disseminated, the expanded council is eventually convened. It is made up of the 12 ``Storm-Watchers'' and the 12 representatives from the ``Fleets'', and convenes only to discuss matters of policy. The most recent decision made by the extended Council of Storm-Watchers was the vote for isolationism, \pIsolation{} ago.
\end{enumerate}

When not discussing matters of policy, most ships run under mild hierarchy. The exception is when they are hunting. During an active hunt for Sea Serpents, a strict militaristic hierarchy is observed, and most captains won't hesitate to throw a mutineer overboard.

\section*{Family Structure}
Chosen family is far more important to \pShippies{} than blood family. Children are raised communally on the ship or in the port, and children as young as 7 are regularly allowed to change ships if they so desire, as long as the other ship will accept them.

\section*{Religion and Morality}
The \pShippies{} primarily serve the dual \cEbb{\God}, \cEbb{} and \cFlow{}. \cEbbFull{\MYname}, is also primary keeper of ships, and the caretaker of things lost or taken away. \cFlowFull{\MYname}, is also primary keeper of the rain, and the caretaker of things created or given. They preside jointly over the sea and their people.

\subsection*{Seeking a Goddess' Approval}
As \cEbb{} and \cFlow{} are Patrons of \pShip{}, they are called up on regularly by their people to judge actions and consult in decisions. A simple invocation of the \cEbb{\God}' name, followed by an out loud statement of the problem is generally considered sufficient. A small offering, such as the burning of a length of cord, or tossing a bit of fish overboard is included for especially important matters. A breeze stirring to blow upon your face is a sign of approval. A cloud over the sun or the moon is considered disapproval. Anything else, including apparently nothing changing, means you are on your own. Still, you've done your due diligence in the matter, and you should proceed according to your best judgement.

\subsection*{Creation Myth}
The most pervasive creation myth in \pShip{} is as follows:
\emph{When the world was complete, \cEbb{} and \cFlow{} wanted beings in their own image on Cengea. They came down to the earth on the Western shore of Cengea. They walked out across the waves, each in a different direction - \cEbb{} to the north, and \cFlow{} to the south. Only one sliver of the ocean around Cengea, \pWod{} did not feel the trod of the \cEbb{\God}es' feet. In each place the \cEbb{\God}es stepped, up rose an island - land that was to be sacred to their people. They walked for three days and three nights, and finally met on the Eastern coast of Cengea. There they sat to rest on an Island that grew up from the sea.}

\emph{\cFlow{} began to build with the sand, but the dry sand would simply collapse - refusing to hold it's shape, even for a \cFlow{\God}. \cEbb{} observed this, and drew the waves up the beach until the sand \cFlow{} was working with was soaked. The two of them knelt and built statues of a small group of humans together. \cFlow{} raised the Sun, and it dried the sand, creating firm bodies that would not be washed away by the water. \cEbb{} raised the Moon, and it's soft light brought peace and wisdom into the statues. Both \cEbb{\God}es called the wind, and it blew wanderlust into the heart of the \pShippies{}. Though the \cEbb{\God}es were tired they looked upon their creation with joy.}
%%what do we call the place of the gods?
\emph{The \cEbb{\God}es took each other's hands and prepared to return to the place of the Gods, but the Human's they had brought to life cried out, and begged the \cEbb{\God}es not to abandon them. The \pShippies{}'s plea moved the \cEbb{\God}es, and their soft hearts ached. It seemed that nothing could be done, as the \cEbb{\God}es could not stay on on Cengea, and the humans could not return to the place of the Gods en masse. Very little stops a \cEbb{\God} for long though, and \cEbb{} and \cFlow{} soon had a solution. They took up a fistful of sand, and threw it to the night sky. There the grains lodged as the first stars. If a \pShippies{} lives their life right by the \cEbb{\God}es, and by each other, then they will have a place among the stars, alongside the moon and sun, and \cEbb{} and \cFlow{}.}

\subsection*{Magic}
The magic of \pShip{} is a subtle one. It lives in the hands of the shipwright, the fisherfolk, the harpoon-maker, and the cooks. Their magic is tied to the sea, and tied to the intentions and outcomes of the items made. So subtle is their magic that there are some in \pFarm{} and \pTech{} that look down their nose at the \pShippies{}, and claim their magic of no consequence. But for those with the skill to see it, the magic of the \pShippies{} makes daily life possible.

\section*{Culture}
The \pShip{} culture is one of wanderlust. Very few \pShippies{} stay in the same place, or on the same ship, where they were born past their teens. Adventurous spirits pervade, and detailed planning is eschewed in favor of following where the wind takes you. Camaraderie among a crew is paramount, with fellowship within a chosen family holding much stronger sway than any blood tie.

\pShippies{} on the whole make very little visual art. They tend toward song, dance, and storytelling as forms of expression.

\subsection*{Taboos and Cultural Norms}
Naming among the \pShip{} is a simple enough affair. Each person is given a single name by their biological parents, which they may change at any time, assuming the \cEbb{\God}es approve the change.

\pShippies{} do not do well with being confined indoors. Most buildings made by \pShip{} hands have huge windows, patios, and courtyards. They use light and airy building material wherever the weather allows, including walls that fold away, opening entire rooms to the weather outside.

\section*{Important Figures}
The Current Council of Storm-Watchers:
\begin{itemize}
	\item \cLoud{}
\end{itemize}

\emph{Players are encouraged to invent new NPCs as needed, for example, very few of the Storm-Watchers are pre-defined.}

\subsection*{School attendees}

Kids




\section*{Unusual Biological Features (if any)}
Most \pShippies{} can hold their breath for upwards of 15 minutes.

\section{Educational system:}

\section{Technology:}

\section{Costuming Related Items:}
Nautical themed costuming.

%Separatists - doesn’t trade with anyone, self sustaining
%Little bit of everything - water, fertile land, mountains and valleys for mining, etc
%Would always have less than those who trade, but would be less impacted when other nations are hit, hit harder when they are hit
%Separatists nations tend to be more warfaring
%Other nations would have more alliances, may send storms toward separatist country several years in a row → causing separatist country to suffer and start sacrificing goats to get a leg up
%Rise of the Chupacabra religion could be from here!!! No alliances
%Ocean-faring, somewhat agrarian -- This is the separatist country
%Industrial equivalence to Victorian-era innovation or sailpunk
%Live by the water / ocean
%Vikings?  Hunting leviathans, taking trophies, hanging them on their ships, etc
%Religion tends to be warlike, chaos, childbirth, women, sea serpents, death, sex
%God of Balance/duality
%Balancing good and evil acts
%God of chaos - God of Change
%Leadership
%Counsels / republic
%Disparate groups that come together for counsels
%Chiefs get together to vote / etc
%Chiefs are elected democratically (21st century morality, arguably the “villains” but voted for it)
%What type of democracy?  Representative or true?
%Borrow from Silent Conversation - consensus building from ground up.  Within each community, people decide whether isolationist or not.  People get together in small group (ie: by house / boat / etc).  Discuss until agree.  One representative from each discussion go to discussion at village and talk until agree.  Propagate upwards / hierarchical process
%Could be using this process to elect who is going to the school, rather than “son / daughter / child of the leader”
%What caused them to shift to isolationism?
%Fear mongering?
%Have a leader / charismatic public speaker convince people that they’re the bad guys and they’re out to get you
%Could play on balance
%For years and years and years, we have been cooperating / trading / etc.  Now need to reset balance and withdraw for a while
%Get hit 2x in a row → spur for them to become isolationist.  Chain of events that leads to them withdrawing.  3rd storm also goes against them
%2nd storm - balance is off, need to be isolationist
%3rd storm - they’re going to be isolationism, we’ll send the storm only against them
%Some of the Vikings believe it was a mistake
%Some of them believe in order to restore balance, need to restore the school
%Why did the 2nd storm go against them?
%Plot - uncovering history
%Bad blood between people at school?
%Why did this happen?
%Could be something that some characters are interested in
%
%
%Why are the Vikings isolationist?  Why would the storms ever be sent toward the allies?
%Originally every three year / equal sharing
%Recent change in Viking culture, no longer trading, pulled out of alliances
%Alliance started hitting only separatists
%Separatists changed to Chupacabra religion - want to take out school, stop targeting nations.  People at lower levels / low socio-economic in other countries have joined them because they’re tired of being hit also because they get hit the hardest when storms hit
%Some Vikings may disagree with this
%Why not just wipe out Vikings?
%Could be that military of Vikings is too strong for them to be taken out?  Using storms instead?
%Leviathans - unspoken truth that they would attack people on shore if Vikings do not hunt them, Vikings hunting them is good for all nations - another reason to not wipe out Vikings
%All countries should have coast - school is landlocked, countries are not → one continent
%Maybe Vikings also have islands as well around most of it
%
%
%Vikings
%Overall Culture:
%Worship God of balance, hit by last two storms, go around and fight sea monsters and that was important to culture, Democracy
%Heroically defend balance of world by fighting epic sea creatures and utilize democratic element.  
%Currently isolationist but weren’t in the past.  Voted on and implemented to reestablish balance
%
%Vikings - monsters stronger or use magic to fight them, without magic might not be able to - entire set of cultures might lose ability to be seafaring
%Vikings - separate from Chubacabra religion but more likely to be part of Chubacabra religion because seen as option in part of restoring the balance.  Not endorsed by Vikings, just growing more easily there because of terrible conditions there
%Some Vikings - don’t want to destroy magic
%
%
%Vikings
%Before seafaring / able to take on serpents
%Decreased ability to go fishing, not able to go fishing
%Cultural consequences and losses
%Potentially preparing for this outcome?  Didn’t think this would work in the first place
%
%Vikings do not believe there is a technological solution but may not believe it should be ended immediately so not all Chubacabras
%Storm is the Balance - maintains balance, should be shared out equally
%
%Why shouldn’t they just continue to hammer the Vikings / wipe them off the earth?
%Sea serpents - make them very important
%Nothing to control them, they are ability to walk on land?
%Impassable mountain between Agrarian and Mageocracy, have to use boats to go between or they’re on two islands
%Archipelago of islands?  School just on another island
%Vikings patrol islands and hunt sea serpents
%What happens when mageocracy makes airships?
%Sea serpents that can fly for short periods of time - enough to take down an airship
%Tell story to various characters - in history, so-and-so invented an airship that would attempt to get from one place to another but eaten by the flying sea serpent!
%Bulk materials - airships can’t carry much.  Perhaps just one path but it was lost?
%Communication - magical communication devices
%Vikings 
%Before isolationism - were the sole fleet of ships
%Afterwards - the other nations had to build their own ships, but weren’t very good at it (or lent some of the fleet?)
%Were they being paid a lot for these services?  Why would they go isolationist?  Maybe a movement but not complete thing.  Vote was isolationist but dissenters have become pirates and are conducting illicit trade between two nations
%How has that impacted ability of two nations to trade or are they building their own boats?  Short term impacts, probably nations building their own boats.  Dissenters, defect and pass on ship building knowledge, ships
%Dissenters may still keep old Gods / old religion since still “maintaining the balance”
%Chubacabraists - perhaps the more “liberal” of the religion (out of rebellion against poverty).  Perhaps more likely to be young, except for the students of the school, as they are hand picked
%Thoroughly indoctrinated - less likely to convert mid game than those who just haven’t been exposed
%
%
%Vikings
%Magic to deal with monsters, possibly to build / keep ships (why nations can’t just build ships of their own)
%Using more subtle magic, may not be aware of it
%Faction may start as “we don’t need magic” but one or two people start researching, find out might not be so OK
%Chubacabra - why don’t all of them jump on
%Has to be more than traditionalist
%Family that is in charge because have lots of magic is invested in magic not going away
%Perhaps one of them has theory that ships go faster because magic - will go do research to prove it
%A Viking kid who wants to go to school might run to go to school to prove / disprove thesis
%Vikings initially sympathetic to Chubacabras but then backpedal
%Probably looking to elect kids who will try to break up deal between Agrarian and Technocracy
%How often do elections happen?  Annually as turn 18 (17-year-olds get to campaign for this)
%
%
%Vikings have fortifications on opposite side to protect the opposite side of the map
%
%
%Oracles in other religions?  How to communicate with the Gods
%Casting runes - Vikings / Technocracy?
%Perhaps each one has large creatures as oracles, Chubacabra religion is attempting to get an oracle to communicate with
%Chubacabra or horrifying creature
%Giant 3-headed goat - probably not
%Vikings
%Sea serpent?  Kept reincarnating but eventually stopped because got mad
%Maybe sea serpents are so bad because kept killing the messengers
%Now rely on runes until they are able to get their oracle back
%
%
%Immigration: Viking - 6 months on our ships, if can’t hack it then we dump you wherever we are and you go your own way
%
%Wave Riders
%Chosen family is very important
%Transitory - very uncommon to marry one person and stay with them for 50 years and stay with them for X years
%More common - stay on a ship, one family and group of kids, move to different ship and have different family and group of kids
%If folks on that ship love the child, maybe leave on the ship or maybe take with you
%Kids maybe follow moms around getting experience on a bunch of boats getting experience before ready to make cultural decision or maybe stay with their “cousins’
%Also leaves door open for deadbeat mothers



\end{document}
