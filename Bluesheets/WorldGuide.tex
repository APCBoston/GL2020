\documentclass[blue]{GL2020}
\parindent=0pt
\begin{document}
\name{\bWorld{}}

\section*{World History}

\subsection*{The Creation Myth}
In the beginning, all of the Deities were equal in the nothingness. \cFarmGod{}, \cTechGod{}, and the twin \cEbb{\God}es \cEbb{} and \cFlow{} got together and decided, in their infinite wisdom, that what was needed was a world. And so they went to the other gods, and through persuasion and logic, trickery, bribery, and unexpected kindnesses, convinced the entire pantheon that they were correct. And so the Deities brought their power together and created \pEarth{}. Upon it they placed the single continent by the same name. This single land mass represents the cooperation of all the Gods, and the wonders that they could create when united. When the primary work was done, a council was called among all the gods to determine who would create the race of beings most like the Deities themselves. The vote was unanimous, and perhaps not unexpected. Those who had originally brought the idea should create its crowning glory. And so \cFarmGod{}, \cTechGod{}, and the twin \cEbb{\God}es \cEbb{} and \cFlow{} set about creating humanity, each in their own way. And so upon \pEarth{} there walk 3 peoples, and their Patron Deities rule the Pantheon.

%In the beginning, all of the Deities were equal in the nothingness. A Deity who's name is now lost to time conceived of something new: a world - tangible and real, created by the Gods as a distraction from the endless existence. The other Deities were slow to agree at first, but as the project progressed, more and more saw the potential of this creation and sought to dominate it. Conflicts became more, and more violent. Eventually the struggle culminated with one God striking a mortal wound upon the God who had originally conceived the idea of a world. Never before had the Pantheon needed to face the loss of one of their number. In anger and fear, they rose up together and destroyed the God who had so wounded the creator. The dying Diety, for their part, was pained as much by the fighting among the Gods as by their wound. With their dying breath, they begged the Gods to set aside their quarrels and be content in each other.

%For a time, there was peace, but it was not to last. Eventually, \cFarmGod{}, \cTechGod{}, and the twin \cEbb{\God}es \cEbb{} and \cFlow{} coalessed into a powerful faction that began overpowering other Deities, one by one. On \pEarth{}, the humans who served these Deities waged their own wars, spilling the blood of those who worshiped other gods, or subsuming them. Those who bowed to the new Gods were spared, and brought into the fold. Those who did not were killed. Eventually, victory was achieved both in the place of the Gods,and on \pEarth{}. The four Deities pooled their power together for a gift to humanity that would cement their worshiper's control of the world: Magic. And so \pEarth{} has been a world of three nations, and their three Patron Gods for as long as history remembers.

\section*{Geography}
\subsection*{Cengea}
\subsection*{The Children of the Sun}
\subsection*{The Free People's Republic}
\subsection*{L'eau}
The \pShippies{} of \pShip{} live on their ships, in coastal towns on the mainland and many of the small islands surrounding the whole continent of Cengea.
\subsection*{The College of the Gods}
There is a lake in the middle of the continent of Cengea. It is not a particularly large lake, but it has many unusual properties due to the ley lines. Ley lines crisscross the ground of Cengea, providing sources of energy and natural conduits for those who know how to use them. Every major ley line in the world connects here. In the center of the lake, the lines rise from the ground, through the water. They twine about each other like vines, and together they reach toward the sky. Up and up until they reach the clouds, where they spread out again, blanketing the world.

The lake is made of something very like water. It tastes like water, quenches thirst like water, and cleans like water, but in other ways, it behaves like a much more viscous liquid. The surface of the lake is normally mirror smooth. No matter how strong the mundane wind blows, the water remains as smooth as glass. Human interference can cause ripples for a short time - i.e. throwing a stone will cause ripples, but they dissipate far more quickly than they should. <INSERT WHETHER CERTAIN CULTURES FORBID BATHING OR DRINKING THE WATER>. It is only in the winds of the Storm that waves begin to form on the surface of the lake.

The lake is also unusual in that instead of having an island on the water, it has one above it. Floating over a thousand feet in the air is \pSchool{}, set on an island suspended in the twining ley lines of the world. The \pSchool{} is part teaching academy, part library, part archive, and part temple complex. The best and strongest students of magic from all three countries are sent to the \pSc{} to finish their studies and ultimately to participate in the ritual that controls the Storm.

\subsection*{The Oceans and the Sea Serpents}
The continent of Cengea has a fairly large continental shelf, with shallow waters extending a good 60 miles out from the coastline. This is lucky as the largest sea serpents won't venture onto the shelf, except during the spawning season, when the breeding serpents enter the brackish water of the deltas and birth live young. Alas the term ``sea serpent'' is a bit of a misnomer as the young live in fresh water, and even the largest have wings that can carry them short distances. They seem to be particularly attracted to any flying contraptions.  The smallest serpents are no more than 5 feet long, and live in fresh water. They swim upstream, settling in rivers and lakes across the continent, and even stretching their wings to reach terminal lakes. The only exception is the Lake under the \pSchool{}. No serpent has ever been recorded to take up residence there, and aerial travel is the only practical way to access the school anyway.

\section*{The Gods}
Cengea has a pantheon of Gods. The most important are the Patron Gods: <INSERT GODS for Children and Free People>, and the twin \cEbb{\God}es \cEbbFull{}, and \cFlowFull{}. The Patrons set the rules of Cengea, granting power to those who follow their teachings, and punishing those who disrespect them. The Patrons are matched in power, and each shields their own faithful from the others. The Pantheon also contains many minor gods, who can grants small blessings, but lack the authority to issue punishment, or the power to protect worshipers from the Patron Gods.

\subsection*{Magic}
Magic has been a part of Cengea since the beginning. The Patron Gods themselves granted humanity magic, breathing life into the ley lines. Each spoke to their most loyal followers and ordered them to build a school on top of the floating island in the middle of the continent. After a few decades spent mastering enough magic to figure out how to get up there, the people of Cengea did just that. It was not for several more decades that people came to understand why the \pSc{} mattered. About a century after the Gods granted humans magic, the Storms began. It was another 30 years before humanity learned to control the storms well enough to direct them in any particular direction and therfore mitigate the damage somewhat.

\subsection*{The Storm}
Every \pCycle{} years, a Storm brews on the lake, blocking off access to the \pSc{} for several days, before spinning off to wreak havoc on some part of Cengea. The students at \pSchool{} can work together, following an ancient ritual, and direct the storm toward one of three areas: The nation of \pFarm{}, \pTech{}, or to the coastlines and outlying islands where the \pShippies{} live and work. Every effort in the past to mitigate the damages done by the Storm has failed. No matter how people try to avoid the damage, the same failure occurs - the storm escapes control of the students and wreaks havoc over the entire continent before dissipating. In the end, more damage is done to all three countries than would have been done to a single country if the attempt had not been made. 

Thirty four years ago, the treaty was established that yielded the rotating pattern of who suffered the storm. Prior to that, the Time of Deciding was a chaotic power grab that left a brutal body count in it's wake, both at the \pSchool{} and in whichever country ended up losing out and getting hit that year. For 25 years, the treaty was honored, and each nation took the storm in turn. Forewarned, the nation targeted was able to devote resources to protecting their most valuable and vulnerable assets, and the others could concentrate on growth. When the treaty was broken 6 years ago, and \pShip{} was struck out of turn, the consequences were devastating. The \pShippies{} were completely unprepared, and lost nearly half of their fleet in the first day. When they were hit a second time, three years ago, they were slightly better prepared, but there wasn't much left to protect.

Some folks advocate for a return to the treaty - although opinions are split on who's turn it would be. Others point out how vulnerable relying on such things made the \pShip{}. \emph{Important note for PCs: Forging a new treaty is beyond the scope of this game. None of the folks necessary to write or ratify such a treaty are available at the \pSchool{}. The tension in game is intended to be elsewhere.}

Timeline of the Storms:
Storms - hit every 3 years
Present year - happening, don’t know where
T-3: \pShip{}//
T-6: \pShip{} (was \pTech{} turn)//
T-9: \pFarm{}//
T-12: \pShip{}//
T-15: \pTech{}//
Etc.



\end{document}
