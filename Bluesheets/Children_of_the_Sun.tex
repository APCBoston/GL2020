\documentclass[blue]{GL2020}
\begin{document}
\name{\bAgrarians{}}

%Vikings are at war
%Add things from world doc that I was supposed to port over

%% Add a bit about how the war is bad, we're already getting pounded, but the Pirates are insult to injury.

\section*{Form of government:}

\pFarm{} are ruled by a monarchy and are currently under the governance of \cQueen{}, considered by most to be a wise and benevolent leader.  With \cQueen{\they} Consort at \cQueen{\their} side and advised by a panel of Councilors, she presides over the noble Lords and Ladies of the country - wealthy land owners and mages.  She passes down the laws of the land and ensures that the nobility uphold their obligations and responsibilities toward their people and their lands.

The system of government is such that the eldest child of the reigning monarch inherits.  Whoever they marry doesn't automatically take on the title of ruler, but instead becomes a Consort of the current ruler.  Upon the reigning monarch's death, the next closest kin inherits.  Should the partnership be dissolved, the Consort does not retain any titles of King or Queen, and any future children they have are not considered eligible for the throne.  For the purposes of inheritance, gender is irrelevant.

The Nobility preside over their lands and the people who live on and work them, none of whom own any land themselves.  For the purposes of day to day management, the Nobility typically designate one or more trusted individuals to oversee their lands during their absences.  On their own lands, the Nobility's words are law - so long as they don't contradict the queen's edicts.

%GM Note: The Queen is beloved by many, but she works hard to keep it that way.  Those who oppose the Queen have a habit of quietly disappearing or dying of seemingly natural causes or unfortunate accidents.

\section*{Names of major players:}

Youngest daughter of the Queen who everyone likes - plot to get her on the throne?
-Agrarian farm worker who was "`adopted"' into a noble family - internal angst between their indoctrinated responsibility as a mage and their now-severed ties to their family
Child of noble who went to talk to Queen on behalf of Vikings - Cursed with misfortune, plot to uncurse and find out who cursed them and why
Minor Diplomat - attempted to immigrate to a different land in their youth, failed, made up a story about being from a different country to avoid condemnation of their family and landspeople - plot???  
Person who cares that the hummingbirds went away and want to work to get them back / bring the balance back / stop hitting the Vikings

\section*{School attendees:}

The majority of the students sent to the \pSchool{} are the children of the reigning monarch or those of the Nobility. Occasionally, however, a child with magic will be born to a working class family.  When these children are discovered, they are often taken in by the nobility for tutelage.  On the rare occasion that they show especially promising talents, they sometimes will be sponsored to go to the \pSchool{}.

\section*{Geography:}

\pFarmers{} are a people gifted with plenty.  Bounded by mountains to the North and ocean on every other side, they live on a large watershed ripe with farmlands and teeming with birds and wildlife.  The climate is Mediterranean.

\section*{Major industries, Economy, and Trade:}

The major industry of \pFarm{} is agriculture - with their vast, rich farmlands, enhanced every spring by the mages amongst the nobility and aided by the technology produced by \pTech{}, the \pFarmers{} produce enough food to feed not only themselves but also \pTech{} to the North.  In return for the traded food, they receive technology from \pTech{} that allow them to continue their mass production of food.  Until the recent breakdown in trade with the \pShippies{}, they produced enough surplus to supply to their nation as well.  In return for food, they were provided with fish for fertilizers and protection from the sea serpents.  It is not a great loss.  The farming technology from \pTech{} works more effectively than the fish-based fertilizers from \pShip{}, and the \pShippies{} still keep the sea serpents far enough away.

It is a time of plenty for \pFarmers{}, but the farmlands which were once fertile have become weary from decades of constant, unsustainably exuberant growth.  Without the technology of \pTech{} and the flood of magics from the nobility of \pFarm{} every year, the land would collapse into dustbowl and famine.

\section*{Politics, International relations, and Immigration:}

The alliance between the \pFarm{} and \pTech{} has lead to an unprecedented period of growth for the \pFarmers{} - it has been nine years since they were last hit by the storms, and the guaranteed safety from the storms has meant that they have been able to dedicate themselves to agricultural expansion and improvements, rather than focusing on preparations and fortifications for the next coming storm.  Should things continue as the leaders of \pTech{} suggest, with the magic and technology of \pTech{}, soon the storms will hopefully be rectified altogether.  It is unfortunate for the \pShip{} that they will have to bear the brunt of the storms until then, but they don't have the responsibility of providing food for multiple nations or producing lifesaving technology, and aren't as dependent on magic in the first place.

Relations with the \pShip{} have been rocky since the Alliance was severed.  The leaders of the \pShip{} are threatening to pull out from the small settlements and stop protecting them from the sea dragons if not appeased.  A few of the Noble families whose lands border the coast and the \pShip{} settlements have attempted to speak for them to \cQueen{}, but so far \cQueen{\they} has decided that the benefits of the continued alliance with \pTechies{} and continued safety of the \pFarm{} from the storms outweighs the cost to the \pShip{} and potential loss of protection.

%Something unpleasant should befall one of the Noble families who spoke out against the Queen's treatment of the Vikings - starting with that family, those who went to see her and speak out against their treatment begin to suffer unpleasant accidents and unfortunate circumstances.  

Immigration to \pFarm{} is uncommon, but welcome.  Unlike the stringent requirements of the other two nations, the \pFarm{} will take you as you are, so long as you are willing to work and work hard.  Unless the person migrating is possessed of a great skill or talent that is considered of cultural use, they will typically find themselves working the lands.  Immigration away from \pFarm{} is more rare.  The \pFarmers{} strongly discourage members of their society from immigrating away.  It is considered borderline blasphemous to travel to \pTech{} for the regular auditions - gifts of magecraft are needed to ensure the seeds sprout every spring, and gifts of music are sacred and contained within the priesthood to.  As for immigration to \pShip{} - it is a fantasy worthy of children to want to run away to a \pShippies{} ship, but the realities of \pShip{} life are hard and dangerous.  Who would trade a peaceful life on solid ground under the gifts of the Sun for the unpredictability and danger of the sea?  Those who fail at both endeavors must return home to the lost standing in the eyes of their families and neighbors.

%GM Note: Often those who return are not welcomed back into their old Fiefdoms.  It is generally easier to travel to a different Fiefdom and start a new life than to face the ostracization from your peers and disownment of your family.  There are those who will pretend they are immigrating from a different country rather than be condemned as failures.

\section*{Brief History of the country:}

%Current events and recent history: 
\begin{itemize}
	\item The oldest child of the Queen is widely disliked and appear more interested in entertainment than the working of the government or the state of their people.  The youngest daughter is widely beloved, but is too far away in the succession path to inherit.  There are whispers amongst the people of a plot to put the youngest daughter on the throne.  While it is fairly common for second children to inherit (occasionally under the pressure of the royalty and Councilors) never have five elder siblings abdicated in favor of the youngest sibling.
	\item The largest of the hummingbirds, avatars of the God, have gone missing and have not been seen for six years, since the first year that the \pShip{} were hit by the storms out of turn.  There are whispers in the Monastery that the God is signaling disapproval for the new pattern of the storms.  There are ancient writings that speak of the last time the avatars went missing and the great anger which had to be assuaged before their return. %GM Note - I super don't know what this was - it just seemed really ominous and fun?
	\item Ten years ago, the farmworkers rose up and overthrew the nobility at one of the fiefdoms along the coast.  Fed up after many years of neglect and mistreatment, the farm workers sought the help of a powerful magic user born of common blood.  She created cursed blankets which were then snuck onto the beds of every member of the nobility during the annual Creation Day celebration, and which strangled them to death in their sleep.  Instead of granting the lands to a new noble family, as was the precedent when tragedy should befall a fiefdom, the Queen ruled that the nobility's mistreatment had not warranted murder.  As the farm workers had appeared to be accessory to the murders, they would witness the consequences of their actions.  The next spring, the lands were not imbued with magic and the once-fertile lands collapsed into dust bowl.  The land still lies dead as a reminder of what happens when the finely tuned balance is disrupted.
\end{itemize}

\section*{Educational system:}

The wealthy offspring of the Nobility are educated by private tutors, hired to give them a thorough grounding in a variety of subjects, including magic, religion, reading, writing, geography, history, mathematics, and governance.  Their tutors include members of the priesthood and the younger or children of nobility without strong magical talents of their own.  The children of the wealthiest families are also tutored by members of the Federation.

Many members of the nobility see themselves as the caretakers of the people who live and work their lands.  As the priests say, the working class are the closest to the earth and are therefore closest to the gods, even if their work is full of hardship.  The farmworkers are educated by members of the priesthood.  Lessons are given to children until the age of twelve, when the majority join their parents in learning their trades.  A select few join the priesthood and continue their lessons - whether because they feel called to the work, are scholastically inclined, don't want to work in manual labor all their lives, have a disability that prevents such work, or to fulfill a love of art or music by channeling it through service to the Gods.

\section*{Cultural foods:}

The food of the \pFarm{} is a food rich in herbs and spices.  While they may preserve some herbs and spices for sale and shipment to the wealthier people in \pTech{}, the goods do not maintain their full flavor after the preservation process.  Generally one has to travel to \pFarm{} to experience the full depth of flavor that can be found in the dishes of even the most humble of homes.  The food of the \pFarmers{} is also mostly vegetarian, due to the prohibition of their God against the killing or harming of other nonhuman beings.  The people who live along the coasts may trade locally for fish with the \pShip{} settlements, but otherwise none of the \pFarm{} consume meat.

%GM Note - As imbedded into their culture as their rich and flavorful foods are the many varied poisons cultivated, brewed, and distilled by \pFarmers{}.  Publicly this is seen through the varied pesticides and deterrents sprayed on crops and warding dwellings.  Silently, this is seen as an every varying and mutating range of poisons to disable or destroy, deployed through ingestion, inhalation, or simply through skin contact.  

\section*{Gender, Sexuality, Relationships, and Family Structure:}

For \pFarm{}, gender is largely irrelevant when it comes to social standing or societal roles.  For the noble classes, children are expected to pair with at least one partner who is compatible for procreation, said partner also expected to be gifted in magecraft.  Though this partly stems from desires to continue on the family lines, it also stems from a need to provide for the futures of their lands and their peoples - without mages, the seeds will not sprout in the spring and the land will collapse into famine.  Unfortunately, strong magical talents are not common in children even through careful selection of partners, so the nobility tend to have large family structures.  Older strong magical talents inherit, younger strong magical talents are married to nobility in other lands in a complex diplomatic dance.  Children without magical talents of their own or without strong enough talents to attract a strong-magical partner either find a place for themselves as a caretaker of the estate or educator, or join the priesthood.

For the working classes, family structures are more flexible.  Children are considered a communal responsibility and resource - adults are free to partner with whomever they wish, but it is frowned upon to choose to be childless.  With the majority of adults working on the farm during the daylight hours, caregiving responsibilities are shared throughout the community.  Single, double, and multiple adult households are all common.       

\section*{Cultural taboos / social norms:}

The cultural norms of the \pFarm{} are:
\begin{itemize}
	\item Vegetarianism
	\item Children are cherished and it is expected that all those who can have them will
	\item Adoption
\end{itemize}

The cultural taboos of the \pFarm{} are:
\begin{itemize}
	\item Cruelty and physical harm to animals and sentient living beings
	\item Choosing not to have a family
	\item Choosing to immigrate away and abandon your responsibilities
	\item Music and art created outside of the Priesthood that isn't dedicated to the God
\end{itemize}

\section*{Costuming Related Items:}

The \pFarm{} amongst all social strata are a people in love with color - all members of society, from the poorest farm worker to the wealthiest among the nobility, dress in a multitude of gem-like hues.  Dye-making is almost an art amongst the working class, and special washes are brewed from local plants to make colorful garments difficult to stain.  Embroidery and needlework are pastimes for winter months, and garments are often embellished.  Common motifs include flowers, leaves, and the sacred hummingbird.

\section*{Magic:}

In \pFarm{}, magic is the gift and responsibility of the nobility. Generally only members of the noble classes have magic, and care is taken by each family to ensure that this gift continues amongst their children.  It is common for courtship rituals amongst the nobility to include demonstrations and displays of magical talent.  On the occasion, a child with magical abilities will be found amongst the working class - perhaps from an extraordinary ability to coax life into stubborn seeds or from a demonstration of healing when one of the farm animals or other farm workers becomes injured or sick.  Should the child attract the attention of the nobility, they are removed from their family and taken into the household of the overlord or local nobility for education.  If they are in very good standing with the landholder, the child may be sponsored to go to \pSchool{}.  Should the child be discovered early enough, they are formally adopted into the household as another son or daughter and may not even be told the truth of their heritage.  It is for the good of all to fully remove the child from their family as soon as possible - after all, unfortunate or malevolent things are known to happen when magical talents crop up where they do not belong.

The gift of magic is one that comes with a heavy responsibility - that of ensuring that all people in multiple countries remain fed.  Every spring, the mages of each fiefdom go down to the farmlands to push their magic into the lands and seeds prepared by the farm workers.  Each seed is imbued with magic, allowing them to sprout and causing them to grow two to three times as quickly and as bountifully as they would otherwise do.  Every plant is pest free, every fruit and vegetable free from fungi or rot that would blight the harvest.  In times of severe injury or illness, the overlord and mages in the keep will also be sent for to provide healing.   

Before the fertilizers produced by \pTech{}, the lands were allowed to fallow and rest between seasons to rebuild the nutritional value of the land.  Since the trade relationship was struck, however, that has no longer been necessary - the fertilizers replenish what nutrients are removed from the soil each year, allowing an increase in farming capacity that has caused \pFarm{} to comfortably be able to provide for the agricultural needs of all three nations.

However, their farmlands are now dependent on this magic - the land is exhausted, and the magic continuing to flow into the land returns it's strength to it each year.  Were the magic to cease to flow into the land, it is theorized that all plant life would die as the dirt itself turned dust.
 
\section*{Religion:}

Hummingbirds are considered sacred to the \pFarmers{}, for they are the oracles of the many-faceted God of \pFarm{}.  Ranging in size from the nail of one's thumb to the size of small houses, their song contains the messages of the God.  Monasteries plant special flowers and sweet-smelling herbs known to be especially pleasing to hummingbirds, and many in the Priesthood study music their entire lives to commune with the oracles and please the God with their songs.

All living creatures are under the care and protection of the God, from the lowliest pill bug to the most majestic stag.  None are permitted to be harmed or killed by \pFarm{} without facing severe punishment from the God.  Because of this, the people of \pFarm{} work diligently to maintain the ecological balance of their farmlands, ensuring that aphids and other living pests are kept under control by their own predators such as ladybugs and birds and discouraged through the use of curses and selective poisons.  After all, how can the harm be traced if the deed is done by magic?  

\section*{Crime and Punishment:}

While small crimes and misdemeanors are punished by society, there are some crimes which are considered so blasphemous by the God of \pFarm{} that, if they are committed, the world itself will enact punishment regardless of whether the crime was observed or not.

Although all Gods punish murder of humans with the removal of one's conscious memories, the God of \pFarm{} considers all living creatures to be worthy of such protection.  Birds, fish, mammals, bugs - all creatures which walk, crawl, fly, or swim are precious to the God.  Those \pFarm{} who live near the settlements of the \pShip{} and eat fish are careful not to trade for live fish, but to only accept fish post-death.

Abuse to animals is also strongly punished.  Should they intentionally starve their animals, kick their pet, beat their livestock, they will receive the same as they laid upon their animals but tenfold.  Whispers are told of people who intentionally squashed a beetle on their plants, only to have a giant bolder hurtle from the sky and squash them where they stood.  While families work hard to ensure that their children learn this lesson early, it is known that the God is lenient towards youth.  Should a six year old accidentally crush an earthworm in their enthusiastic grasp, they are likely to get a peck and a scold from a nearby hummingbird rather than instant karma. 

\end{document}
