
%%%%
%%
%% This file sets up the Sign and Label datatypes and creates Sign and
%% Label macros.
%%
%% Signs generally represent interesting parts of game area, usually
%% as things posted on walls.  Labels represent other things, often on
%% or inside envelopes, that are part of complex mechanics.
%%
%% The default value for \MYloc will inherit location from the Place
%% or Sign most immediately up the ownership tree.  Override this by
%% setting \MYloc to anything (even blank).
%%
%% Sign is for full-sized signs that would cover most of a large
%% manila envelope; SignMedium is for signs sized to half-sized manila
%% envelopes; SignSmall is for signs sized for small manila envelopes
%% (the same size as item cards).  Label, LabelMedium, and LabelSmall
%% are analagous, but they don't have a \takedownby note at the
%% bottom.  You can always use a sign or label without an envelope or
%% with a differently-sized envelope.  Choose which based on
%% visibility and content.
%%
%% SignTiny is for signs you want to be hard to find; it is small and
%% does not have a \takedownby note.  SignDot is for a very small
%% "dot" which only has a title.
%%
%% SignStrip produces a strip of paper (without a \takedownby note)
%% with labels on the outside that show on both sides if you fold it
%% in half.  These are a convenient alternative to sub-envelopes. They
%% can also be used for "s-packets" taped to walls (see
%% Extras/README-s-packets).
%%
%% LabelCover produces a label similar to the cover to a research
%% notebook.  LabelPage, likewise, produces a page.
%%
%% EOG is for full-sized end-of-game signs.
%%
%%%%%

\DECLARESUBTYPE{Sign}{Element}
\PRESETS{Sign}{
  \FD\MYloc	{\mylocation} %% real-space location
  \FD\MYtext	{} %% text of sign
  }
\POSTSETS{Sign}{
  \edef\mylocation{\MYloc}
  \protected@edef\@ownerstring{%
    \MYname%
    \ifx\mylocation\empty\else\ (\mylocation)\fi%
    }
  }
\def\mylocation{}

\def\loc#1{\rs\MYloc{#1}}

\DECLARESUBTYPE{SignMedium}{Sign}
\DECLARESUBTYPE{SignSmall}{Sign}
\DECLARESUBTYPE{SignTiny}{Sign}
\DECLARESUBTYPE{SignDot}{Sign}
\PRESETS{SignDot}{\s\MYtext{}}

\DECLARESUBTYPE{Label}{Sign}
\PRESETS{Label}{\s\MYloc{}}
\DECLARESUBTYPE{LabelMedium}{Label}
\DECLARESUBTYPE{LabelSmall}{Label}

\DECLARESUBTYPE{SignStrip}{Sign}
\DECLARESUBTYPE{LabelCover}{Label}
\DECLARESUBTYPE{LabelPage}{Label}

\DECLARESUBTYPE{EOG}{Sign}
\PRESETS{EOG}{%
  \s\MYname	{End Of Game}
  \s\MYtext	{{\bf\Huge You may not pass through here.}}
  }


%%%%%
%% \signbig[<location>]{<name>}{<text>}
%% \eog[<location>]
%%
%% \signmdeium[<location>]{<name>}{<text>}
%% \signsmall[<location>]{<name>}{<text>}
%% \signtiny[<location>]{<name>}{<text>}
%% \signdot[<location>]{<name>}
%%
%% \labelbig{<name>}{<text>}
%% \labelmedium{<name>}{<text>}
%% \labelsmall{<name>}{<text>}
%%
%% \signstrip[<location>]{<name>}{<text>}
%% \labelcover{<name>}{<text>}
%% \labelpage{<name>}{<text>}
\newinstance{Sign}{\signbig[3][\mylocation]}{
  \s\MYloc{#1}\s\MYname{#2}\s\MYtext{#3}}
\newinstance{EOG}{\eog[1][\mylocation]}{\s\MYloc{#1}}

\newinstance{SignMedium}{\signmedium[3][\mylocation]}{
  \s\MYloc{#1}\s\MYname{#2}\s\MYtext{#3}}
\newinstance{SignSmall}{\signsmall[3][\mylocation]}{
  \s\MYloc{#1}\s\MYname{#2}\s\MYtext{#3}}
\newinstance{SignTiny}{\signtiny[3][\mylocation]}{
  \s\MYloc{#1}\s\MYname{#2}\s\MYtext{#3}}
\newinstance{SignDot}{\signdot[2][\mylocation]}{
  \s\MYloc{#1}\s\MYname{#2}}

\newinstance{Label}{\labelbig[2]}{
  \s\MYname{#1}\s\MYtext{#2}}
\newinstance{LabelMedium}{\labelmedium[2]}{
  \s\MYname{#1}\s\MYtext{#2}}
\newinstance{LabelSmall}{\labelsmall[2]}{
  \s\MYname{#1}\s\MYtext{#2}}

\newinstance{SignStrip}{\signstrip[3][\mylocation]}{
  \s\MYloc{#1}\s\MYname{#2}\s\MYtext{#3}}
\newinstance{LabelCover}{\labelcover[2]}{
  \s\MYname{#1}\s\MYtext{#2}}
\newinstance{LabelPage}{\labelpage[2]}{
  \s\MYname{#1}\s\MYtext{#2}}


%%%%%
%% \sEOG{}
%% use \sEOg[\loc{<location>}]{} for EOG sign at a specific place
\NEW{EOG}{\sEOG}{
  }


%%%%%%%%%%%%%%%%%%%%%%%%%%%%%%%%%%%%%%%%%%%%%%%%%%%%%%%%%%%%%%%%%%

\NEW{Sign}{\sTest}{
  \s\MYname	{A Room}
  \s\MYloc	{10-250}
  \s\MYtext	{A lecture hall with large, sliding blackboards.}
  }


%%%%%%%%%%%%%%%%%%%%%%%%%%%%%%%%%%%%%%%%%%%%%%%%%%%%%%%%%%%%%%%%%%


%%GM HQ
\NEW{Sign}{\sMurderConsequences}{
  \s\MYname	{If your character succeeds in murdering someone, read this sign}
  \s\MYloc	{GM HQ}
  \s\MYtext	{Take a copy of the greensheet in the packet attached to this sign. It has instructions on exactly how losing your memory works, and what you have access.}
  \s\MYgreens {}
}
  
\NEW{Sign}{\sSignCOne}{
  \s\MYname	{Sign C}
  \s\MYloc	{GM HQ}
  \s\MYtext	{You may not interact with this sign unless instructed to do so. If you are instructed to do so, you will lift this sign and read the sign under it.}
}

\NEW{Sign}{\sSignCTwo}{
  \s\MYname	{Sign C -2 Dreaming of the Past}
  \s\MYloc	{GM HQ}
  \s\MYtext	{After successfully completing the ritual to use \iMirror{} to look into the past, you went to sleep. Overnight, you had the following dream://
  \emph{You see \iMirror{} in your mind. In it, you see your reflection. The mirror drifts close enough to you for your breath to fog the surface. Instead of fading, the mist begins to swirl slowly across the darkening surface of the mirror. You feel compelled to touch it. And in doing so, you fall forward into the mirror, tumbling into the vortex. Back through the swirling mist of time you are drawn.}

\emph{You find yourself in a fancy restaurant in \pTech{}. Two figures are speaking quietly in a corner booth. \cEvil{} says ``I can buy you more time to figure out how to stop the storms. But you must do something for me in return'' \cHeadScientist{} nods eagerly, and the scene fades as \cEvil{} passes \cHeadScientist{} a small satchel.}

\emph{A flash of \cEvil{} handing a small bottle to someone dressed as a \pSchool{} employee, on the edge of the Lake beneath the \pSc{}. \cEvil{} says ``make sure only the students drink the wine. More bodies than that and people might find their way back to me.'' The person nods.}

\emph{Suddenly you’re in a forest clearing, somewhere in \pFarm{}. \cDiplomat{} is meeting with \cEvil{}. \cEvil{} says ``I have secured our ability to manipulate the vote at the Time of Deciding. My agent already has the fake stones.'' \cDiplomat{} replies ``Good, now make sure you clean up any loose ends.'' The grin \cEvil{} has as the scene fades sends chills down your spine.}

\emph{The \pSchool{}, the Ritual to Control the Storm just completed. The students stand or sit, panting. One student is clearly a \cHeir{\formal}. Another looks very much like \cAssistantScientist{} might have looked 6 years ago.  The Teachers and Advisors rush in from the edges of the room to support the students. Somehow in the chaos glasses of wine are produced. A toast is proposed. All of the students drink, except \cAssistantScientist{}’s lookalike. \cAssistantScientist{\They} sniff the glass, make a dubious face, and put it down. As \cAssitantScientist{\their} classmates start to collapse, \cAssistantScientist{\they} slip out a side door, unnoticed.
}
  
The Dream reverberates with magic, and you know it to be the truth - it may not be the `textbf{whole} truth, but everything you saw happened. When you speak of what the dream showed you, if you speak truly, your voice will ring with the same magic, carrying the weight of truth to convince your listeners. \emph{(OOC: You should tell other players that your voice carries the weight of magical truth when you speak of what you’ve seen. Keep in mind however that a character may choose to pretend not to believe you for any number of reasons.}

If one or more \iStoneFlower{\MYname} (\iStoneFlower{\MYnumber}) were substituted in the ritual for other flowers, you also have a pounding headache for the next 30 minutes or so.}
}
  

%%%%%%%%% The Gardens %%%%%%%%%%

\NEW{Sign}{\sGardenGeneral}{
  \s\MYname	{The College Gardens}
  \s\MYloc	{Garden}
  \s\MYtext	{The botanical gardens of the College of the Gods are said to rival the finest estate gardens in The Children of the Sun. Rare flowers, trees, and shrubs from all the known corners of the world, and some few found nowhere else, thrive in the carefully tended gardens. Something always seems to be in bloom, and those versed in the subtle language of flowers can tell the time of year, even the time of day, by what is blooming. A lovely place for a stroll, a secret rendezvous, or simply to meditate in solitude. To explore the gardens and see what is blooming, spend one uninterrupted minute in front of this sign without taking any other actions (conversation is fine), then pull one item from the appropriate envelope below (see the smaller signs: ``\sGardenDay{}'' and ``\sGardenNight{}''.), depending on the time of day. You may then either put the item back for others to enjoy, or pick the flower (against College policy) and keep the item.}
}

\NEW{SignMed}{\sGardenDay}{
  \s\MYname	{The College Gardens By Day}
  \s\MYloc	{Garden}
  \s\MYtext	{If you choose to explore the gardens during the day (7:00 am to 5:00 pm), you will pull one item from the envelope attached to this sign.}
  \s\MYitems {multi{2}{\iFlameOrchid{}\iMorningGlory{}}}
}

\NEW{SignMed}{\sGardenNight}{
  \s\MYname	{The College Gardens By Night}
  \s\MYloc	{Garden}
  \s\MYtext	{If you choose to explore the gardens during the night (5:00 pm to 7:00 am), you will pull one item from the envelope attached to this sign.}
  \s\MYitems {multi{2}{\iNightshade{}\iMoonflower{}}}
}

\NEW{Sign}{\sGardenFountain}{
  \s\MYname	{The Garden Fountain}
  \s\MYloc	{Garden}
  \s\MYtext	{A lively fountain of bright green malachite burbles peacefully in this quiet corner of the garden. Carved in the shape of vines and tendrils, the fountain blends seamlessly with the surrounding vegetation, the water flowing like rain over leaves.

\emph{OOC: If you need a source of freshwater for a mechanic, this fountain is an appropriate source.}}  
}


%%%%%%%%% The Graveyard %%%%%%%%%%

\NEW{Sign}{\sGraveyardGeneral}{
  \s\MYname	{The College Graveyard}
  \s\MYloc	{Graveyard}
  \s\MYtext	{The small graveyard of the College of the Gods receives few visitors, and has even fewer graves. Most people who meet their demise on campus are returned to their home nations. Nonetheless, more than a few important figures from the College’s history, and that of the world as a whole, are buried here. The imposing, wrought iron gates are flanked by bronze statues of the avatars of the Gods - a scarab beetle, a sea serpent, and a giant hummingbird. Both statues and gates are badly corroded, perhaps due to the fog which seems to perpetually cover this place. Grand mausoleums and humble tombstones alike lie shrouded ahead, their dim silhouettes like ghosts in the mist. Walk among the graves if you dare, and see what the dead can teach.}  
}

\NEW{Sign}{\sFadedTombstone}{
  \s\MYname	{A Faded Tombstone}
  \s\MYloc	{Graveyard}
  \s\MYtext	{This humble grave is clearly one of the oldest in the graveyard - overgrown with moss and weeds, its tombstone chipped, worn, and half sunken into the ground. Whatever inscription it once held is now illegible. The tombstone does not seem to match the style of any of the three nations. To search among the weeds around the grave, spend 1 uninterrupted minute in front of this sign without taking any other actions (conversation is fine), then pull one item from the envelope.}  
  \s\MYitems {\iStoneFlower{}}
}

\NEW{Sign}{\sMemorialToFallenStudents}{
  \s\MYname	{Memorial to Fallen Students}
  \s\MYloc	{Graveyard}
  \s\MYtext	{A somber memorial stands here, carved from a single block of obsidian. The symbols of all three nations are carved into it, and the inscription reads as follows:

\begin{center}
\emph{In memory of the 12 students who perished on campus at the end of The Time of Deciding, year 1,816 PF. May they be remembered not simply for their final act, but for the lives they lived, and might have lived, had tragedy not struck them down in their prime. These are the names of the fallen:}

\vspace{12pt}
\emph{Edon Faleden}
\emph{Chadwick Rexford}
\emph{Clarissa Reinhardt}
\emph{Jonathan Gloaming}
\emph{Henrietta Noonshadow}
\emph{Lily Dewsberry}
\emph{Luca Belleville}
\emph{River Dawning}
\emph{Diana of 8th Fleet, Osprey Archipelago}
\emph{Jethro of 1st Fleet, Guillemot’s Landing Harbor}
\emph{Luna of 4th Fleet, Sheerwater Ship}
\emph{Tiburon of 3rd Fleet, Black Albatross Ship}
\end{center}
}
}

\NEW{Sign}{\sTombOfFirstPrincipal}{
  \s\MYname	{Tomb of the First Principal}
  \s\MYloc	{Graveyard}
  \s\MYtext	{From a distance, you would not have guessed that this is the grave of so important a personage as the first Principal of the College of the Gods - a small, stone urn with few adornments and a simple tombstone engraved with the symbols of the four Deities. The inscription reads as follows:

\begin{center}
\emph{Here rests Katharina Friedrich, First Principal of the College of the Gods from the year 0 PF to 181 PF.}

\emph{Her steady hand guided the College through its earliest days and established the sacred traditions which govern this institution.}

\emph{Our First Principal both coined and lived by the College’s First Principle:}

\emph{“For all of Cengea.”}

\emph{She will be missed.}
\end{center}

}  
}

%%%%%%%%% The Training Field %%%%%%%%%%

\NEW{Sign}{\sTrainingFieldGeneral}{
  \s\MYname	{The Training Field}
  \s\MYloc	{Training Field}
  \s\MYtext	{The training ground of the College of the Gods stretches before you, littered with the detritus of centuries of magical training, despite the best efforts of the groundskeepers. The wide, open fields are pockmarked with smoldering craters, bizarre plants, shattered fragments of magitech devices, persistent localized weather systems, and other oddities. To search among the detritus, spend one uninterrupted minute in front of this sign without taking any other actions (conversation is fine), then pull one item from the envelope. Tread carefully - this place is decidedly unsafe.
}
  \s\MYitems {multi{3}{\iLily{}}}
}

\NEW{Sign}{\sCombatCircle}{
  \s\MYname	{The Combat Circle}
  \s\MYloc	{Training Field}
  \s\MYtext	{At the center of the Training Ground stands the Combat Circle, a squat pillar of red marble looming above the surrounding landscape. Carved into the stone, a spiral staircase winds round its flanks before reaching the summit. It is here that students at the College of the Gods are trained in the art of magical combat and self defense. The Circle has seen everything from friendly sparring to grueling combat drills and winner-take-all duels; the College prides itself on the fact that no one has ever died within its bounds. The Circle’s pockmarked surface is covered with glowing magical sigils, powering the energy fields which prevent anyone swept from the summit from falling to their death.}
}

