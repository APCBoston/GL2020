\documentclass[char]{GL2020}
\parindent=0pt
\begin{document}
\name{\cHistory{}}

%% Michael AKA COS Teacher History

You are \cHistory{} (\cHistory{\they}/\cHistory{\them}) and comfortably middle aged. You teach History at the \pSchool{}, helping to shape new minds while simultaneously secretly managing \cQueen{\Their} Majesty’s spy network. It’s not \emph{exactly} that your loyalties are split, but it would be wrong to say that after over fifteen years at the \pSchool{} you have managed to keep yourself from being attached to the place. The students here need guidance, the \pSc{} needs maintenance, and sometimes international politics needs a swift kick in the rear. It is a busy life, but you are good at it, you enjoy it, and the duplicity feels so natural at this point that you hardly notice the qualms.

You are the firstborn \cHistory{\child} of a middlingly powerful noble family in \pFarm{}. You wanted for very little growing up, and navigated the viper’s nest of Court with aplomb. Eventually this brought you to the attention of the most powerful people in the country, including \cQueen{\Majesty} \cQueen{} \cQueen{\themself}. Your previous life’s ambition -- to teach at a \pFarm{} university for a few years before marrying well and settling down on your family’s estate to raise a family -- was swept aside in favor of the needs of the nation.

A position at the \pSchool{} was quickly arranged for you. At first, it was simply in exchange for small reports on the activities of your fellow teachers, sent home clandestinely. Then came a request for a verbal report, made to \cQueen{\Their} Majesty directly. And then it was private conferences with the \cQueen{\Majesty}, always spent advising \cQueen{\them} on matters of national importance. It was always in private and unofficial, of course -- but it occurred with rapidly increasing frequency. Before you really knew had happened, you were your \cQueen{\Majesty}’s personal spymaster. Sure, there is an official state spymaster -- but you answer to the \cQueen{\Majesty} alone. If this became known, at best you would simply lose your job. Teachers are supposed to be neutral entities, teaching all students equally, and to benefit of all three nations. At worst, you would be struck down in secret by the nobles of your own nation, or by jealous rivals in other nations who could finally strike at you with your mask removed.

In service to your cover as a teacher, and in honor of your first love (History), you do your best with the students. Not all young minds start out eager to learn, but every class has one or two who really take to it. While your expertise is in the history of \pFarm{}, you teach the history of all of \pEarth{}. You have some hope for \cLibAssist{}, but otherwise the current crop of students is a little disappointing.

The \pSc{}, though, is a marvel unto itself. The floating island on which it is built is unique -- it is sustained by the ley lines, which need to be maintained with the utmost care. You’ve watched that ritual before, always led by the principal, and have contemplated participating in the past -- but always decided against it, due to the price. It is obvious to you, and anyone else who has studied the phenomenon, that the ley lines need the island to stay together as much as the island needs the leylines to float. If something were to \emph{happen} to the island -- if it were destroyed -- the proximity of so much intense magical energy, unshielded by the rocks and dirt of the island, would shred the fragile leylines in a matter of hours. With the leylines severed, all magic would stagnate, trapped like an insect in amber. Only the Gods themselves could repair the damage - but without magic, how would the people of \pEarth{} even appeal to the Gods for help?

The Library at the \pSc{} is also a wonder of architecture and magic. In it are stored some of the most rare and valuable tomes in all of \pEarth{}, which you count yourself lucky for the opportunity to study whenever you want, as well as many magical artifacts. These priceless pieces are stored here because, while the storm rages around the island as it’s power builds, the storm never strikes the school -- so things stored here are protected in a way that no other place on \pEarth{} is. For this weekend, the most important are likely the Relics. The Relics are attuned to one of the three nations, and used in the ritual to control the Storm. The integrity of the Relics is the purview of the priests, but the construction and execution of the Ritual generally falls to the teachers. You’ve helped \cLibrarian{} with this in the past, and found \cLibrarian{\them} quite pleasant to work with. You look forward to helping with this duty again this weekend, and understand the importance of keeping the Relics that are normally stored in the Library on the island. Relics that were brought by Advisors should go home with those advisors obviously, but the relics entrusted to the Library must remain at the Library when not in use for the ritual.

The third noteworthy part of the school are the Bunkers. These marvels of engineering provide a way to protect the people who stay at the school during the Storm. An advisor named \cBunker{} built them several decades ago, and comes by every six years to look in on them. Again, claiming a historian’s curiosity, you’ve sat in on \cBunker{\their} work before -- while you aren’t a great hand at repairs, you can certainly pass tools. You look forward to working with \cBunker{} again this year. While you haven’t met \cBunker{\them} in person too many times, you’ve maintained a casual correspondence, and found quite the kindred spirit. \cBunker{} is a most gifted architect, and has a keen mind for current affairs. 

But with a war on, maybe anyone with any sense of self preservation has a keen interest in current affairs. Six years ago, the \pFarm{} and \pTech{} signed a treaty that completely screwed over the \pShippies{}. The storm was directed to the \pShip{}, instead of the \pTech{} that year in the most momentous stab in the back of the modern era. And then the children who were voting to direct that storm were killed. That detail irked you. How could you have not been told that was part of the plan? Obviously you would have tried to persuade people to find another way -- but that was not sufficient grounds to cut you out of that particular piece of intelligence. This incident convinced you that you needed to turn a little more attention to what was going on at home. You recalled some of your agents and began to place them within the \pFarm{} in hopes of never being caught unaware like that again. One need not be a diligent student of history to predict the war that followed the betrayal. And you were confronted with the question: is there ever such a thing as a just war?

At first the question seemed moot. Your country was at war. The \cQueen{\Majesty} was for it. Therefore, you were obligated to be too. You passed what information you could, although a small but not insignificant number of reports from your birdies started to look a little odd. Still, you had a war to win. Until one day, only a few weeks ago, you got a letter from \cQueen{} \cQueen{\themself}. It was written in your own personal code that only the two of you know, and in it was the most unexpected news. \cQueen{} is actually against continuing the war! \cQueen{\They} want a treaty, ideally one without too many painful reparations, but as long as it involves protection from retaliation by the \pTech{}, \cQueen{\they} support it. Of course, \cQueen{} can’t publicly declare \cQueen{\their} support for a treaty without putting \pFarm{} in a really awkward position, so \cQueen{\they} \cQueen{\have} turned to you to influence the proceedings without being detected. \cEvil{} carries the official public responsibility for negotiating at the treaty table, but if things come down to it, you have a letter from the \cQueen{\Majesty}. This would allow you to reveal your authority and rank and speak on \cQueen{\their} Majesty’s behalf -- but only if you cannot accomplish your ends in a more subtle way.

Of course, ending the war with minimal negative backlash to the \pFarm{} would be easier if you understood the Warlord who leads the \pShip{} better. Your sources suggest that \cLoud{\full} was not particularly bloodthirsty a decade ago. So what changed? \cWarlordDaughter{} is the Warlord’s \cWarlordDaughter{\child}, and you’ve tried to coax the story out of \cWarlordDaughter{\them} -- but to no avail so far. Perhaps you can find other sources? One way or another, though, you are determined to find out the history of the Warlord, both in case it assists in neutralizing \cLoud{\them}, and on general principle as a historian.

This weekend you expect to be quite busy doing all the things you do best. You have truths to uncover, both for an accurate historical record, and to the advantage of your \cQueen{\Majesty}. You have children, relics, and priceless tomes to protect. And you have a war to end without being noticed. Good thing you are up to the challenge.

\begin{itemz}[Goals]
	\item Work with \cLibrarian{} to prepare the ritual to control the storm. Encourage the students to take an active role in the whole process -- it’s a learning opportunity!
	\item Be the \cQueen{\Majesty}’s voice in the treaty negotiations, advocating for a treaty and end to the war, without ever being recognized as interfering.
	\item Uncover what happened to \cLoud{} that turned them into such an effective Warlord.
	\item Act as the \pFarm{} spymaster and collect as much information as you can that might benefit your country and it’s interests.
	\item Work with \cBunker{} to repair the normal wear and tear on the bunkers.
	\item Ensure that the relics that are stored on the Island are returned to their proper place when they are not in use for the ritual.
\end{itemz}

\begin{itemz}[Notes]
	\item 
\end{itemz}

\begin{contacts}
	\contact{\cLibrarian{}} The ultimate authority on the Ritual to Control the Storm. \cLibrarian{\They} \cLibrarian{\are} very pleasant to work with.
	\contact{\cBunker{}} The genius behind the bunkers. You enjoy chatting with \cBunker{\them} while helping inspect the bunkers.
	\contact{\cEvil{}} \cEvil{} is the public voice of the \cQueen{\Majesty} at the treaty talks. Unfortunately \cEvil{\They} seem to be very strongly pro war. Manipulating them may prove challenging.
	\contact{\cLibAssist{}} Your most promising student. You might make a historian of \cLibAssist{\them} yet, if \cLibAssist{\they} \cLibAssist{\does}n’t decide to become a librarian.
\end{contacts}

\end{document}

%\textbf{Plots: Primary}
%(4)Preparing the Ritual
%https://docs.google.com/document/d/1xCN7BLfzBB3Tt237YSzGKusQj7cmi6nQxATP6aCLgEQ/edit?usp=sharing
%Library has 3 more relics to find among the stacks somewhere. Michael is one of the best resources for navigating the library.
%
%(6)The Bunkers
%https://docs.google.com/document/d/1b60m8irZphM3_9f5FWSMUcdZrBoPz9K_Y0rcXGIDqLM/edit?usp=sharing
%Helping to fix the physical layer of the bunkers
%
%(24)Form a treaty/Ceasefire
%https://docs.google.com/document/d/1t-0IfPUyv6w-RkDMiytG7EWiUynflrxyMFz4B1tRkG8/edit?usp=sharing
%Pro treaty. Feels bad about screwing the WR, and wants people to work together. Michael was privately assigned to be the Queen’s voice unofficially, as she can’t publicly support ending the war.
%
%\textbf{Plots: Secondary}
%(3)The Net of Two Phases
%https://docs.google.com/document/d/1rv5safAChTNCpSK-80O9Q7a3R7tQ5Mwfe89AwtFA0bs/edit?usp=sharing
%Michael will not let anyone take any of the relics off the island in case they are needed for the ritual
%
%(36)The Ley Lines
%https://docs.google.com/document/d/1Eea2ldlPpnp_LxW0KkHEjFFKn6ZceE_Q6JtAJgek0No/edit?usp=sharing
%Knows they need maintenance.
%
%(44)Researching the war of the gods
%https://docs.google.com/document/d/1DFStUjq-cH95rfnQNWdMzA0K1vPof8MEQs0QmwRWdH8/edit#
%Not explicitly stated in document, presumably helping with research?
%
%(25)Why did the warlord start the war?
%https://docs.google.com/document/d/11K_59HPsCmn3bV0UQHoDBrnTp2MyFn6Ax8DJNrtItcU/edit?usp=sharing
%Advisors who want treaty to stop the war need to neutralize the warlord somehow.  (Michael is pro treaty) Michael is looking to solve historical records lack of info about the warlord, get the whole story, both for history and queen.
%
%(26)Trading on the black market
%https://docs.google.com/document/d/1YSPBFxKp7VyLE-c0Emw3tWsl2Ua-Nw-LwUEftAxS8fg/edit#
%Eleanor wrote to some L’eau NPC friend/contact that Michael is supposed to be a mostly neutral teacher, but he is secretly besties with the queen and spying/doing her bidding in the school, where her influence is supposed to be limited to the Advisors.  He could lose his job over this, and the L’eau knowing it would be an advantage to them.
%
%Michael is a Spymaster and colleagues with \cBunker
%
%

