\documentclass[blue]{GL2020}
\parindent=0pt
\begin{document}
\name{\bTech{}}

The \pTech{} is a nation of dreamers, inventors, and artists.  Located in the northern portion of \pEarth{}, and separated from the \pFarm{} by a mountain range known as the \pSpine{}, the \pTechies{} live in a vibrant and beautiful nation whose natural vistas and vast urban centers (built in massive cave systems) inspire creativity in all who see them.  The \pTech{} is seen both nationally and internationally as a beacon of technology and the arts, but struggles with the inherent economic class and income differences imposed by capitalism and competition.  To counter this, they have also adopted many social programs and regulations to try to ensure that everyone can afford a good life.  

Eight years ago, the \pTechies{} reached the brink of perfecting a technological solution to the problem of the Storms.  With the promise of this new technology, the Council saw fit to form a secret alliance with the \pFarm{}, with whom they agreed to send the Storms exclusively towards the \pShippies{}, so that the other two nations could have the available resources to pursue this invention for the good of all nations in the long term.  In the aftermath of this betrayal, the \pShip{} declared war on the allied nations in hopes of claiming safe land and punishing those who had wronged them.

\section*{Government}
The \pTech{} is run by a group of officials called the Council, which is led by the Arbiter.  Most Council members are appointed, but some are elected to represent various self governing groups like trade unions, and only one member is elected by the general public.  This Council makes the laws and everyday decisions needed to maintain a thriving culture and economy, but bows to the authority of the Temple where the governance of magic is concerned.  The Council appeals to the Temple on behalf of the people to further tech, magic, and the arts.  They also orchestrate the provision of social welfare programs such as public health clinics, childcare, primary education, and career placement. While well intended, in practice these programs are riddled with bureaucratic inefficiencies and red tape, and many people slip through the cracks.

The current Council has 12 members, who are notable in many different fields believed to represent the best and brightest of the nation.  The members of the Council most relevant to game are as follows:
\begin{itemize}
	\item The High Priest: \cAntiChup{\full}, known as “\cTechGod{}'s Grace,” is the leader of the Temple and speaks on behalf of the Temple's interests.
	\item Tech Star: Annually, a National Tech Star Competition is held, and anyone who has invented a new piece of technology may compete for the position of Tech Star.  The current Tech Star, \cTechStar{\full}, is young enough to also be a student at the \pSchool{}.  The Tech Star only sits on the council for a year before the next one is chosen.
	\item The Faledon Family: \cFaledonParent{\full} is the head of the Faledon family, who was the original philanthropic family who helped design the structure of the current government centuries ago.  The Faledons are an incredibly wealthy family (and corporation), with numerous patented inventions.  While Faledon Industries occasionally draws criticism for its monopolistic practices and exploitation of cheap labor, the company has a robust PR department that is swift to quash negative scrutiny, drawing attention instead to their philanthropic foundations which, among other things, offer scholarships for promising but underprivileged students seeking higher education.  Such students usually go on to work for Faledon Industries.  
\end{itemize}

\section*{The War Effort}
The war with the \pShip{} was not the predicted outcome of the alliance with the \pFarm{} made six years ago, as the \pShippies{} have historically been slow to adopt great changes. Unfortunately, a warmonger rose up among the \pShippies{} and roused them to attack viciously, stirring sea serpents to ravage coastal lands.  This Warlord, \cLoud{\full}, is reviled by the \pTechies{} and the \pFarm{} alike, and every encounter with the \pShippies{} has been brutal.  And if that were not bad enough, the \pShip{} have also unleashed their ``pirates,'' who seem bent not on war, but on looting, and have no qualms about killing to take what they want, amnesia or no.  As such, the \pTech{} has suffered terrible losses, and national morale is at an all time low for those who live in coastal regions.  Even away from the coast, many families have lost loved ones to the fighting, and with much of the economy redirected towards the war effort, rationing of food and other basic necessities has affected everyone's quality of life.

Thankfully, the alliance has a new weapon – four years ago, the \pTech{} engineers invented trebuchets designed to launch curses made by the \pFarm{}.  Mass production was begun immediately, and one year later, these potent weapons could be found on every battlefront, driving back L'eau marauders and sea serpents alike.  The trebuchets continue to hold the \pShip{} mostly at bay -- hopefully long enough to finish the research on ending the Storms, and reach a peaceful resolution to the war.  Most citizens of the Free People's Federation would welcome such a resolution – as long as it doesn't come at the price of the next Storm (hopefully the last ever) being directed at their homeland.  The nation has made no preparations to receive the Storm, so the consequences of it striking would be devastating.
	
\section*{Economy}
The Free People's Federation is an industrialized society thanks to their plethora of magitech inventions.  Their economic system can be described as heavily regulated capitalism.  Technological innovations go through a lengthy (and many complain byzantine) process in order to be mass produced.  Tech must be prototyped by an inventor, approved by the Temple, enabled by a Magician, licensed by the Council's Office, and finally manufactured and sold in either the free or subsidized markets.  Scientists are researchers who study magic as it applies to tech and the world, often in esoteric fields.  Such abstract science often lacks immediate economic value, so they usually work for the Temple or Council and are barely paid enough to make ends meet.  

The war with the \pShip{} has cost both lives and resources as well as inciting civil unrest, since, as usual, the poorest have been hit hardest by the war, and have the fewest resources to replace the dried up imports from the \pShippies{}.  Both the Temple and the Council are concerned at the state of the economy, which is showing signs of instability.

\subsection*{Trade}
Art, certain technology, and fashion are the \pTech{}'s primary exports.  This includes everything from novels to fine clothing to traveling circuses to advanced farming equipment.  While most technological devices, like artificial lights and industrial machinery, are forbidden from being exported, in the last 20 years, and especially since the alliance was forged, there has been a marked increase in farming tech sold to the \pFarm{}.

Historically, much of the nation's food is imported from neighboring countries, both from the \pShippies{} who fish the seas and from the \pFarmers{} who work the fertile lands to the south.  Food has been scarcer as a result of the war, and the poor are struggling to keep their families fed.  All official trade with \pShip{} stopped when the war began, although there is still some black market trade occurring that officials have not yet been able to thwart.  This has become a major problem, since the \pShippies{} have been able to more easily react to new \pTech{} innovations during wartime than they would have otherwise been able to.

\section*{Technology}
Which tech is greenlit for sale and manufacture is regulated by the Temple, and must be activated by a Temple appointed magician.  Only the simplest of tools do not run on magic, so the Temple has completely constructed day-to-day life in the \pTech{}.  Tech that hasn't been activated by a magician will not function, and non-Temple appointed mages are forbidden on penalty of divine punishment from activating tech. Inventors can temporarily activate a prototype (for approx. 30 min.) for the sole purpose of demonstrating it's functionality, but only properly trained Clerics are able to permanently activate it, and then, only with written approval from the Church. The Temple ostensibly maintains its tight regulation of technology in order to ensure that any new tech benefits the people and encourages peace, stability, equality, and innovation.  Dissidents sometimes question the Temple's decisions on which technology to approve, however, claiming its decisions are all too often motivated by corruption, fear, and an overbearing need to preserve of the status quo.

Many modern amenities have an approximation produced in the \pTech{}, including refrigerators, toasters, and showers.  Technology such as surgical equipment and industrial machinery is common as well.  One of the most prized pieces of technology is tech clothing, which is clothing that can display video or changing patterns, configurable by the wearer.  VidCom devices, which are communication technology that approximates a modern day walkie talkie with an added video screen, are the most high profile invention to come out of the annual Tech Star competition this year.  However, by a narrow one vote margin, the Council decided to ban mass production of the devices and limit their use to military applications only, out of fear that they could fall into the hands of the \pShippies{}.

\section*{Religion}
The \pTechies{} follow \cTechGod{}, the \cTechGod{\God} of Knowledge, Inspiration, Engineering, and Science. \cTechGod{}'s Avatars are the Celestial Beetles, special scarab beetles that walk through ink in order to leave the \cTechGod{\God}'s runic messages on paper for Priests to interpret.  The Circle of Five, the highest officers of the Temple, govern according to a written body of religious laws.  The Temple was founded at the birth of the Nation, by the Grace of \cTechGod{}.

\cTechGod{} punishes the following crimes among \cTechGod{\their} followers: Murder, kidnapping, major Intellectual Property theft, and scam artistry are all punished with memory loss.  Unregulated magical activation of technology, on the other hand, is punished by a lifetime inability to use magic or technology.  If the crime is not as severe, it may lead to temporary memory loss or a temporary mark on the forehead that brings social shame.  While it does not elicit divine punishment except in the most extreme cases, lying in general is considered anathema to Kero, whereas truth is held sacred.  The temple is expected to keep an eye out for all transgressions, and draw \cTechGod{}'s attention when they occur.

	
\section*{Education}
Education is free through general education schools that keep kids until they are 15.  After that, there are private and Council run schools that teach trades, art, performance, and engineering -- but these come with a hefty tuition.  Children of poorer families can petition wealthier families for adoption to get access to higher education.  This raises the prestige of both families because both are seen to be prioritizing the education of the child, which is paramount.  A few truly exceptional students may also be granted scholarships -- but this is a prize dangled in front of students far more often than it is realized. 

\subsection*{Tech Stars and Tech Fairs}
Each university holds an annual Tech Fair, where both students and non-students may apply to present their inventions.  The best of these inventions will be sent on to compete in the National Tech Star Competition to determine which inventors win the prestige that year, and who gets to take the Tech Star's seat on the Council.

\section*{Art and Clothing}
The \pTech{} is enchanted by art of all kinds, but especially performances.  Performances of all kinds are treasured and made available to all at almost any price point. Anywhere you walk down the street, you will see art installations on street corners, walls, and atop buildings, and hear music floating from all directions.  It is truly a haven for artistry.

Clothes are a status symbol in the \pTech{}. The \pTechies{} tend to dress primarily in reds, oranges, whites, and gold, with a steampunk or solarpunk style, with the meta-irony that the steam engine has not been invented in this world (who needs an engine when magic can run anything you want perpetually?) The \pTechies{} are also the most likely to wear shiny or synthetic fabrics like satin, crepe, and velvet, as well as lacework in flowing or fitted designs, accented by wearable or integrated tech.  All classes and genders like to wear flashy cosmetics and/or accessories, as they can afford.

\section*{Norms and Taboos}
Every member of the \pTech{} wants to raise themselves up. They wish to stand out, achieve wealth, fame, greatness, and make a name for themselves.  It's an unspoken shame not to be able to afford higher education for all your children, and so poorer families do everything in their power to get their children education -- including giving them up for adoption to wealthier families.  

Marriages in the Free People's Federation are a social contract strictly between two people – sex and romance are generally, but not exclusively, related. Extramarital affairs happen but are considered scandalous. To the extent that anyone knows about polyamory, it is seen as an oddity that some members of other nations engage in.

\section*{Opinions about People from the Other Nations}
The \pFarmers{} are a coarse people, stuck in the past and more than a little bit backward.  The feudal structure of their country stifles genius that can come from the lower classes, and elevates the incompetent to high positions.  Their idea of ``adoption'' is also backwards -- while families should be free to give children better lives somewhere else, that should be with the consent of both parties.  Their denial of music from most people is also a shame -- living without music would be horrid.  Still, they are reliable allies and solid trading partners. 

The \pShippies{} are a warlike people.  They spend most of their lives fighting something or other, whether it be the other nations, sea serpents, or each other.  It's no wonders that odious pirates have sprung from their ranks!  While not as coarse as the \pFarmers{}, they too lack refinement, and do not put enough stock in the effort and life saving power of technology.   The two strangest things about the \pShippies{} are their form of government and one of their Goddesses, Ebb.  Why would you let children as young as 14 make ANY kind of decision more impactful than what to eat for breakfast?  And why on \pEarth{} would you worship a \cEbb{\Deity} dedicated to destruction, when there is so much to build in life?
					
\end{document}

