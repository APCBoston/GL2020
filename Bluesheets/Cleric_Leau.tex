\documentclass[blue]{GL2020}
\parindent=0pt
\begin{document}
\name{\bLeauCleric{}}

Being a Cleric of \cEbb{\full} and \cFlow{\full} is beautiful, wonderful, and all encompassing. You are the \cEbb{\God}ess's eyes, ears, and mouth to their people. You are expected to live your life according to the strictest standards of the \cEbb{\God}es's edicts. If you cannot follow the tenants, how will you convince the faithful that it is not only possible, but meaningful and fulfilling to do so?

\section{A Cleric's Powers and Responsibilities}
The Goddesses grant their Clerics power in many areas. They grant the clarity of mind to guide decisions, the steadiness of hand to repair things in need of it, and the force of will to destroy that which has run its course. \pShip{} Clerics, like all Clerics, can provide physical and mental healing, though often not as well as a \pFarm{} Cleric, well versed in the musical healing arts.

The greatest power a Cleric of \cEbb{} and \cFlow{} have is in listening without judgement to a supplicant's plight, and offering guidance and suggestions. It is important to keep in mind that the \pShip{} religion believes that no true peace can be achieved without balance. If someone has been done wrong, the most likely advice from a Cleric is ``demand restitution or get even.'' The understanding is that nothing short of reciprocating in kind will allow someone to put down their hurts. If someone has done a wrong, a Cleric will likely advise ``make material amends as soon as possible. An apology that is just words is hollow.''

\section{Becoming a Cleric}
To become a Cleric of \cEbb{} and \cFlow{} is not easy. Most are not even allowed to try. The Goddesses mark those with the aptitude for it before their 12th birthday. It is exceedingly rare for them to appear later (but not completely unheard of). The twinned marks may appear anywhere on the child's body, but the back of both wrists is the most common. Once the marks appear, the child may be enrolled at a Monastery exclusively, or continue at their Academy, and take supplemental classes at the Monastery. Either way, a minimum of 4 years of study is typical before anyone will sponsor the initiate (as they are now called) to the full priesthood.

The trial for priesthood is fairly straightforward to begin -- all you need is a sponsor, at least one witness, and a Path. The sponsor must be a full Cleric of \cEbb{} and \cFlow{}, and have been one for at least 2 years. The Witness can be anyone of the \pShippie{} religion, although another cleric or fellow initiate is the most common. While only one official Witness is required, the elevation of a new Cleric is a very important part of their journey, so Initiates often wish to share the experience with a wide community of family and friends, both in and out of the priesthood. The ritual itself involves several steps, including communing with the Goddesses and with the support of your sponsor and your witness, hopefully proving your worth.

Before the Ritual can begin, the Initiate must pick a Path. Once you are brought into the priesthood on a Path, you cannot change. Therefore the decision is incredibly important, and initiates often spend months agonizing over it. Most Initiates ultimately choose to follow \cEbb{} or \cFlow{} primarily -- but some attempt to walk the difficult path of true Balance. 
\begin{itemize}
\item \cEbb{\full} is the keeper of ships, and the caretaker of things lost or taken away. Clerics of \cEbb{} know the value in loss, destruction, and removal. One cuts away at the wood to reveal the ship within. The end of one dream opens new and better opportunities. Sacrifice is honored highly. Loss and grief are comforted. Cleaning and clearing make way for the growth of \cFlow{}. 
\item \cFlow{\full} is the keeper of the rain, and caretaker of things created or given. Clerics of \cFlow{} know the value of starting, building, nurturing, and improving. One starts a new project because they believe in the outcome. One gives things away to start ripples, which become waves of sharing and growth. Creation and generosity are revered. Quality, and the patience to create it are cultivated. Uninhibited growth in preparation for the culling of \cEbb{}.
\item To walk the path of balance between these two Goddesses is the most challenging of all. Balance is about the moment of turning. When is it time to stop cutting things? When is it time to take a break from creating? It's in the fullness at the top of the breath, and the emptiness at its end. The Path of Balance is all about turning points and change. It is about reigning in the excess of either extreme, and bringing things into harmony. There are no more than two dozen Clerics of Balance among all of the \pShippies{}. Those rare few for whom marks appear late are often particularly good candidates for the Path of Balance.
\end{itemize}

\section*{The Creation of the \pShippies{}}
The most pervasive creation story in \pShip{} is as follows:\\
\emph{When the world was complete, \cEbb{} and \cFlow{} wanted beings in their own image on \pEarth{}. They came down to the earth on the Western shore. They walked out across the waves, each in a different direction -- \cEbb{} to the north, and \cFlow{} to the south. Only one sliver of the ocean, \pWod{}, did not feel the trod of the \cEbb{\God}es' feet. In each place the \cEbb{\God}es stepped, up rose an island -- land that was to be sacred to their people. They walked for three days and three nights, and finally met on the Eastern coast of \pEarth{}. There they sat to rest on an Island that grew up from the sea.}

\emph{\cFlow{} began to build with the sand, but the dry sand would simply collapse -- refusing to hold it's shape, even for a \cFlow{\God}. \cEbb{} observed this, and drew the waves up the beach until the sand \cFlow{} was working with was soaked. The two of them knelt and built statues of a small group of humans together. \cFlow{} raised the Sun, and it dried the sand, creating firm bodies that would not be washed away by the water. \cEbb{} raised the Moon, and it's soft light brought peace and wisdom into the statues. Both \cEbb{\God}es called the wind, and it blew wanderlust into the heart of the \pShippies{}. Though the \cEbb{\God}es were tired they looked upon their creation with joy.}

\emph{The \cEbb{\God}es took each other's hands and prepared to return to the place of the Gods, but the humans they had brought to life cried out, and begged the \cEbb{\God}es not to abandon them. The \pShippies{}'s plea moved the \cEbb{\God}es, and their soft hearts ached. It seemed that nothing could be done, as the \cEbb{\God}es could not stay on \pEarth{}, and the humans could not live in the realm of the Gods. Very little stops a \cEbb{\God} for long though, and \cEbb{} and \cFlow{} soon had a solution. They took up a fistful of sand, and threw it to the night sky. There the grains lodged as the first stars. If a \pShippies{} lives their life right by the \cEbb{\God}es, and by each other, then they will have a place among the stars, alongside the Moon and Sun, and \cEbb{} and \cFlow{}.}

\end{document}
