\documentclass[char]{GL2020}
\parindent=0pt
\begin{document}
\name{\cTechStar{}}

You are \cTechStar{\full} (\cTechStar{\they}/\cTechStar{\them}), 19 years old and a technological prodigy.  You are a first year student at \pSchool{}, and come from the fair but rigid \pTech{}, ruled by \cTechGod{} who both inspires creativity and (in your own privately blasphemous opinion) severely limits the benefits of progress to all. The one thing you want more than anything else is to follow your dreams of spreading your technology to the world, and you won't let anything, God or mortal, stop you.  An idealist to your core, you know you could use the tremendous opportunities of your new position as Tech Star, and the spectacular invention that won you the title, to help those less able to help themselves.

Growing up as the beloved only \cTechStar{\child} of a loving and moderately well to do family in the \pTech{}, you always knew you wanted to be an inventor.  You began in the traditional way, dissecting the family suction cleaner, to your parents' delight.  They dutifully provided you with materials, tools, and the education you would need to succeed, and you have exceeded their wildest expectations by inventing VidCom (Video Communication) devices, which allow two people to communicate remotely \emph{(OOC Note: Think of these like long distance walkie talkies with a video screen rather than a phone; they only work in pairs)}.  This concept is mind blowing in a world of letter writing, and your invention easily won you the annual TechStar competition. You almost felt sorry for the other competitors, but you knew your invention could do the world the most good.  

Upon winning the competition you were granted the TechStar Council position in the government for a year, and admission to \pSchool{} due to your innate magical ability, as well as a number of prestigious job offers that will be open to you once your education is complete. It's hard to keep up with the Council's work alongside all your schoolwork, but you do your best, attending Council meetings when absolutely necessary, voting by proxy when possible, and abstaining when the matters at hand are not in your area of expertise. You've been lucky to have the help, and friendship, of \cScholarship{\full}, who's always happy to chat or help you with your schoolwork.

Unfortunately, despite your lofty achievements and position, you feel enormously disappointed with the way the \pTech{} has handled the development and distribution of your VidCom devices.  You'd assumed that they would immediately enter mass production and be given to people around the country, selling for a reasonable price, but this did not occur. First, the Temple refused to magically enable more than a handful of the devices, stating the masses were not ready for such technology. You appealed to the Council, but by a mere one vote margin, your fellow Council members rejected your appeal and sided with the Temple. The Head of the \cHeir{\formal} family, \cFaledonParent{\full}, was the swing vote. You don't trust \cFaledonParent{\them}, or anyone associated with that family, including \cHeir{\full} and \cDiplomat{\full}, to deal fairly with, or have the best interest of the people at heart – they only want what benefits them and their precious family name. The Council claimed that your VidCom devices were ``too important'’ to the war effort, and that handing them to the public would inevitably lead to them getting into the hands of the enemy. It's ridiculous! You know there's a war on, but surely people being able to communicate with their loved ones would boost morale and national pride? 

And the war seems like a mistake in the first place. Why can't the nations work things out peacefully? Why shouldn't people of all nations come together through your Vidcom Devices, and see that they are not three separate nations, but one people, sharing the planet? You feel that all three governments have mishandled the war just like the council mishandled your invention. You are going to fix it - by yourself if that's what it takes.

Your entrance to \pSchool{} has provided you many opportunities, and some of them are directly related to your goal of spreading your tech to the world. Most importantly, you've heard credible rumors about the existence of a black market that sells tech from the \pTech{} at the \pSc{} – and a black market that sells tech MUST be able to activate it without reprisal from \cTechGod{}! Rumor has it that \cLibAssist{\full} is a runner for the black market, so \cLibAssist{\they} will be your point of contact. Hypothetically, if you can set up a stable supply chain of production for your VidCom devices for the black market, you'll be one step closer to your dream. You do not plan to part with the blueprints for your device this weekend, though. Whoever has the blueprint is the only one able to make the VidCom devices, and you're not ready to release your monopoly yet, so you haven't even made a second copy of them. Pursuant to manufacturing more VidCom devices, you've made arrangements with a stellar engineer from the \pShip{} named \cBunker{\full} to bring you supplies with which to build VidCom devices here (``Misc. Magitech Parts'’). You're not entirely sure how much \cBunker{\they}'ll be able to bring you, but you hope \cBunker{\they} is bringing at least some. 

While that plan may get things rolling, you have higher aspirations. If you could get \cTechGod{}'s blessing to activate your technology yourself, you wouldn't need the black market in the long run. \cTechGod{} has granted the power to oversee the activation of tech to the Temple, but they're clearly not acting in a manner that honors \cTechGod{} or your people any more, because they are keeping your tech out of the hands of the people whose lives it could greatly improve. Like most successful inventors, you have the magical power it would take to activate VidCom devices yourself, but not the know how. Further, the laws of \cTechGod{} decree that doing so would have unimaginable consequences. If only you could plead your case directly\ldots{} Just this morning, you overheard your teacher, \cFlowPriest{\full}, and one of the advisors, \cCurse{\full}, talking over coffee about some enchanted beans that could create a portal to the Divine Plane, and this immediately set your mind spinning. Is that really possible? You're not sure, but at the end of the day, it's too tempting to pass up. You are going to help those two, and if this isn't some kind of pipe dream, get them to take you with them to the realm of the Gods - then beseech \cTechGod{} to favor you and your people by changing this sacred law - or at least grant you an exemption. Surely, \cTechGod{} will see that the Temple is mistaken, and correct the situation.

When you're not busy working out how to change the fundamentals of your religion and country, you've also got a couple of personal projects on your plate. Your best friend \cDisney{\full}'s birthday is coming up on Sunday, and you want to surprise \cDisney{\them}. Since \cDisney{\them} \cDisney{\were} awakened recently from 200 years of magical sleep, \cDisney{\they} \cDisney{\have} some gaps in \cDisney{\their} memory that \cDisney{\they} complain to you regularly about. You're sure that with a little research and innovation you could figure out a way to recover these lost memories. It's a perfect gift, and you know \cDisney{\they}'ll be happy. Maybe you and \cDisney{}'s other friends, \cAdopted{\full} and \cPirateChild{\full}, could plan a surprise party for \cDisney{\them} too? Unfortunately, you and \cPirateChild{} don't get along – you don't see eye to eye on anything. You aren't sure what \cDisney{} sees in \cPirateChild{\them}.

Speaking of \cDisney{}, \cDisney{\theyare} far too close with \cDisney{\their} shady mentor, \cWildCard{}. While \cWildCard{} might have been the one to wake \cDisney{} up and \cDisney{\they} is justifiably grateful for that, you're pretty certain that \cWildCard{\their} motives were (and are) more than \cWildCard{\they} let\cWildCard{\plural} on. \cWildCard{} is just\ldots{} shifty, and you'd really prefer that \cDisney{} distance \cDisney{\them}self from \cWildCard{\them}. But \cDisney{} is so attached, you will probably need proof of some kind to convince \cDisney{\them}. This is yet another place you don't see eye to eye with \cPirateChild{}. \cPirateChild{} actually likes \cWildCard{}. Maybe you shouldn't be surprised. They are kind of like two peas in a pod. But why does \cDisney{} have to associate with them? But maybe that's unfair to \cPirateChild{} – maybe \cWildCard{} got to \cPirateChild{\them} somehow, and is forcing \cPirateChild{\them} to pretend to like \cWildCard{\them}? If it turns out that \cPirateChild{} is under \cWildCard{}'s thumb, you should probably get \cPirateChild{\them} out of that so you can actually evaluate who \cPirateChild{\they} is and decide if you like \cPirateChild{\them}.

In other happenings going on this weekend, you're also very, VERY interested in learning all you can about the inner technological and magical workings of the all important Bunkers at the school, and you'll get just this one chance to gain access. You've volunteered to help with the planned maintenance and repairs that will be happening soon, and you're thrilled at the chance to work with \cBunker{} – the star engineer who actually built the bunkers, and the very person you contracted to bring the parts you need to build more VidCom devices! You look forward to the honor of working with \cBunker{} on maintaining the bunkers.

Also of note this weekend, the current principal, \cPrincipal{\full}, is retiring, and your favorite teacher, \cBeetle{\full}, is one of the two finalists for the position in a heated competition with \cMusic{\full}, another teacher. You think \cBeetle{} would make a great principal and you hope \cBeetle{\they} win\cBeetle{\plural}!

Another event you are interested in this weekend is the Ceremony of Excellence. You know that this ceremony traditionally involve a demonstration of power. As one of the most magically gifted and creative students, you believe that honor rightfully belongs to you. You intend to do whatever it takes to convince \cMusic{}, the teacher in charge of the ceremony, to pick you for the student combat demonstration.

At the end of the day, while there are a lot of things going on that deserve your attention, if you can secure a route to activate and distribute your VidCom devices safely, you'll leave happy - there's simply too much at stake there to not. If you can make sure that \cDisney{} remains safe and happy, so much the better. This will be your weekend - you can feel it.

\begin{itemz}[Goals (in roughly descending order of importance)]
	\item Secure a solid path to activate and distribute your VidCom devices to as many people as possible, preferably without accumulating debt to the black market or anyone else.
	\item Help \cBunker{} maintain the Bunkers, and learn as much as you can from \cBunker{\them}.
	\item Determine whether the portal to the Realm of the Gods is feasible, and if it is, help make it a reality so you can speak to \cTechGod{}.
	\item Keep \cDisney{} safe from whatever \cWildCard{} is really planning.
	\item Prepare a great birthday present for \cDisney{} – restoring their lost memories.
	\item Show your stuff as one of the combat demonstrators at the Ceremony of Excellence.
\end{itemz}

\begin{itemz}[Notes]
	\item You are a first year student.
\end{itemz}

\begin{contacts}
	\contact{\cDisney{}} Your friend and confidant who magically slept for a few centuries. Sweet and innocent, with some memory issues.
	\contact{\cBunker{}} The brilliant engineer behind the college's bunkers. Also bringing you supplies to manufacture more VidCom devices.
	\contact{\cWildCard{}} A shady \cWildCard{\person} who woke \cDisney{} from \cDisney{\their} enchanted slumber. Clearly has ulterior motives. Why can't \cDisney{} and \cPirateChild{} see that?
	\contact{\cPirateChild{}} A friend of \cDisney{}'s who you don't like one bit.
	\contact{\cScholarship{}}A friend you often study with, who you feel like you can rely on.
	\contact{\cLibAssist{}} A fellow student who helps out in the Library, and, more importantly, is a runner for the black market. Maybe \cLibAssist{\they} can help you distribute your VidCom devices?
	\contact{\cCurse{}} You don't know much about \cCurse{\them}, beyond \cCurse{\them} being a bit weird, but \cCurse{\they} \cCurse{\have} the key to something you really want – access to \cTechGod{}.  
	\contact{\cFlowPriest{}} A decent enough teacher, but mostly interesting since \cFlowPriest{\they} \cFlowPriest{\have} the key to something you really want – access to \cTechGod{}. 
	\contact{\cBeetle{}} Your favorite teacher, even though Religion isn't your favorite subject. Competing to be the next Principal of the school.
\end{contacts}

\end{document}
