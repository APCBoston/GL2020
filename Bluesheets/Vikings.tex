\documentclass[blue]{GL2020}
\parindent=0pt
\begin{document}
\name{\bVikings{}}

\section*{Basic History}
The people of \pShip{} call themselves \pShippies{}

\subsection*{The Storms}
Despite the \pShip{} \cEbb{\God}es being connected with rain and the sea, the Storms are still devastating. \cEbb{} and \cFlow{} are powerless to substantially impact the storm. It's all part of the great balance. Magic necessitates Storms. Fighting off the serpents and uniting the \pShip{} nation requires magic. So does building ships that avoid running around all on their own, and harpoons that always hit their target, and nets that attract fish. In order to preserve the \pShip{} way of life, there must be Storms. The present trouble is caused by the imbalance in who endures the punishment from the Storm.

Six years ago, it was \pTech{}'s turn to have the storm directed at them, but the storm was directed at the \pShip{} instead. The damage was tremendous. Unprepared for the storm to hit them, the \pShippies{} were about their regular business. More than half the fleet was destroyed in the initial 48 hours. Reports from \pSchool{} about what happened were frustratingly vague. Despite the consequences of murder in Cengea, somehow the twelve students involved in controlling the storm that year ended up dead. If that wasn't fishy enough, subsequent investigation revealed that the vote was apparently unanimous. All the students, including the ones from \pShip{}, supposedly voted to send the storm there. The prevailing theory is not a comforting one. Someone, or some\emph{thing} sabotaged the voting and changed the votes after they had been cast, but before the actual ritual to control the storm began.


In other nations, the people can hunker down and wait out the storm. In \pShip{}, that isn't really possible. Ships are as much at risk of damage in port as they are at sea. Coastal towns can be torn from the cliffside by the massive waves, or washed away in the floods from the rain. The \pShippies{} are still somewhat safer on land though, and so most ships will dock at the nearest port before the Storm hits if they can. The storms stir up mud from the bottom that drive fish from their normal migrating patterns, making food much harder to come by. The storms also seem to attract sea serpents, with the largest (and most dangerous) specimens entering shallow waters in the wake of the Storm. With fishing patterns disrupted, their fleet in tatters, and the sea serpents agitated, the \pShip{} struggled to feed themselves those first few months after.


%%How publically known is it why storm was sent at the L'eau?  Each nation might have slightly different story on it - gov'ts will have to make statements to their people.  Free People maybe said "`A new treaty has been established' .  Requires additional discussion...What if the leau kid was in on it and didn't die, but is living in a different country?  What if the free people brought the idea, and the Agrarian brought the assassin.  All three countries are trying to figure out what happened.  For most of nation, it's a mystery.  Maybe an advisor knows?  Determine exactly what happened and how solved it later.  All three kids are believed dead, but leau kid survived.  all three kids theoretically agreed to do the thing / send the storm at leau
  %%How long does the storm last? three days to brew, three days raging, three days to calm.  How long does it take to travel from place to place?  Cluster of fast-moving tornados or similar (multiple spouts) - travels together in a cluster.  Like a tornado hurricane


\subsection*{Politics/International Relations}
Up until 6 years ago, the \pShip{} were on good terms with both the \pFarm{} and \pTech{}. The \pShippies{} were available for hire to transport goods, or to chase off sea serpents. For the most part they kept to their own and more or less lived and let live. That all changed with the breaking of the treaty. Calls for isolationism reached the Council of Stormwatchers from almost every Fleet within 4 months of the Storm. One voice in particular, \cLoud{\full}, enjoyed a metoric rise to prominence on unifying these cries and answering the call for revenge. Very few \pShippies{} doubted that ultimately \pFarm{} and \pTech{} do not value \pShip{} lives as they should, and that the only recourse is to protect themselves by withdrawing from diplomatic relations and instead seeking safety in territory on the mainland that they have occupied. The scant few voices that argued otherwise fell silent 3 years ago, when the storm was sent to \pShip{} territory again.

\subsection*{Immigration and Emigration}
Up until 6 years ago, attempts at immigration to \pShip{} is not too uncommon. The appeal of a seemingly carefree life is strong. But joining the \pShippies{} is a tall order. Immigrants are expected to work three years on whichever ship they can convince to take them, learning \pShip{} ways and  following \pShip{} rules, but without a voice in debates regarding rules and policy. For those not born on a ship, the life is harsh and particularly dangerous. Fully 25\% of immigrants don't survive the three year initiation period. Emigration is fairly common too. The often austere life of the \pShip{}, coupled with the dangers of the sea, drive many to seek safety on dry land.

\section*{Geography}
The \pShip{} are mostly a waterfaring folk. They travel Cengea's oceans, as far out as the continental shelf extends. The waters beyond the shelf harbor sea serpents beyond the ken of even the most experienced \pShippies{} crew. No ship that has ventured out beyond the shelf has returned to tell the tale. \emph{Exploration beyond the edge of the continental shelf is beyond the scope of this game.} The \pShippies{} also float Cengea's rivers and lakes, using them as highways for travel, moving goods and people from the other two nations as well as their own.

Most \pShip{} towns are ports, and cling to the coastline, no matter how rugged. Only on the eastern side of Cengea do the \pShippies{} claim any land in the interior, and most of it is a massive flood-plain. Around the rest of the continent, the \pShip{} control small stretches of coastline, and almost every island large enough to support a human settlement. The only area of ocean that \pShippies{} fear to sail is \pWod{}. \pWod{} is due west of Cengea. It is an area of particularly turbulent waters, with unpredictable currents and jagged rock pillars looming out of the ever-present mist. The source of the unsettled waters is a massive whirlpool at the heart of \pWod{}. A few of the bravest - or most desperate - crews will work \pWod{}, scavenging from the shipwrecks. Ships are driven into the \pWod{} by sea serpents on a regular basis, leading to the prevailing theory among the \pShippies{} that it is a place fundamentally opposed to their dieties.  %%some ships may not wait for the ship to shipwreck and attack sooner

\section*{Economics and Industry}
The economy of \pShip{} is primarily one of trade. Within the country, trade is always good between ships and port, and ship to ship, as folks trade food and supplies from all around the continent. Economic interactions with the other two countries were primarily the charging and collecting of delivery fees as the \pShippies{} transported goods and materials. In the years since the isolationism began, the economy has suffered some, as there is no longer an influx of outside wealth, but compared to the damage done by two storms in a row, the loss is negligible.

By leaps and bounds, the \pShippies{} can build ships like no other. Whether by lack of necessity on the part of other countries, or some influence from \cEbbFull{} a ship built by the \pShippies{} will always be higher quality, no matter it's intended purpose. Ropes, nets, and other sea-faring gear is much the same. \pShippies{} are also the only nation meaningfully engaged in fishing. Hunting the nomadic shoals of fishes, and dodging or fighting off the sea serpents requires strength, skill, and luck that none beyond the \pShippies{} possess.

\pShippies{}s don't go in for many pure luxuries. If it isn't functional, it's dead weight, and dead weight could be the difference between running around and escaping an angry sea serpent. There is a particular art and skill to making something that is both functional and beautiful, and minimalism is prized highly.

Both \pFarm{} and \pTech{} control the majority of the coastline that borders their interior holdings, and \pFarm{} have many rivers and lakes. All are infested with sea serpents. The other two countries regularly hire \pShippies{} to clear out lakes, defend a port, or to escort travelers. A talented crew can make good money doing this - as long as they don't get killed on the job.

\section*{Form of Government} The \pShippies{} have a consensus based representational democracy. The government consists of four levels:

\begin{enumerate}	
	\item The {\bf``Ships'':} The ships are the smallest unit within the \pShip{} government. Each ``Ship'' is comprised, literally, of the crew of each ship. The term also applies to the residents of a house in a town or city (most \pShip{} houses are communal living complexes, with strong internal communities and shared crafts - e.g.: a house made up of a dozen or more weavers). Each ship is in a way the law unto itself. They may make their own rules as long as they do not violate edicts from the larger governing bodies (e.g.: The ``Raft''). A ship may also elevate an issue that goes beyond their own rails to the ``Raft'' and call a meeting of those representatives to discuss it. When the Storm-Watchers declare something in need of discussion, they send messages (via dolphin or gull in modern times.) All members of the crew, 14 years and older are allowed to participate in the discussion, which continues until consensus has been reached. Once it is reached, a representative (traditionally the first mate) is sent to the ``Raft'' to carry on the next round of discussion and consensus building.
		\item The {\bf ``Rafts'':} A ``Raft'' is composed of all of the ``Ships'' that call the same port home. Houses belong to their nearest port that is in the same ``Fleet''; this is usually the city the house is built in, as the \pShippies{} have very few inland towns. A representative from each ``Ship'' will be sent to a predetermined meeting point for the ``Raft'' when it is necessary to discuss matters of policy, either brought up from a ``Ship'', or handed down from the ``Council of Storm-Watchers''.
		\item The {\bf``Fleets:'':} A ``Fleet'' is composed of all of the ``Rafts'' from a geographic area. Each geographic area is identified by a 1/12th arc of a clock, centered in the middle of the continent. The sections are numbered 1-12, starting N to NE Whichever slice the home port is in, the ``Raft'' belongs to that ``Fleet'', regardless of where the ``Ship'' is currently. A representative from each ``Raft'' will be sent to a predetermined meeting point for the ``Fleet'' when it is necessary to discuss matters of policy from either above or below. When necessary, a representative from the ``Fleet'' will be sent to the ``Council of Storm-Watchers'' when consensus is reached.
		\item The {\bf``Council of Storm-Watchers'':} The standing high council of the \pShip{} nation. This group of 12 elected officials (one for each ``Fleet'') holds office for 3 years. Council-members may not hold consecutive terms, and may not hold more than 3 terms in their life. One position is up for election each season of the year, meaning that in any 3 years, the whole council will have turned over. Storm-Watchers are tasked with identifying topics that must be discussed by the nation as a whole so that policy may be developed. When such policy is identified and disseminated, the expanded council is eventually convened. It is made up of the 12 ``Storm-Watchers'' and the 12 representatives from the ``Fleets'', and convenes only to discuss matters of policy. The most recent decision made by the extended Council of Storm-Watchers was the vote for isolationism, \pIsolation{} ago.
\end{enumerate}

When not discussing matters of policy, most ships run under mild hierarchy. The exception is when they are hunting. During an active hunt for Sea Serpents, a strict militaristic hierarchy is observed, and most captains won't hesitate to throw a mutineer overboard.

\section*{Family Structure}
Chosen family is far more important to \pShippies{} than blood family. Children are raised communally on the ship or in the port, and children as young as 7 are regularly allowed to change ships if they so desire, as long as the other ship will accept them. There is no particular pressure to have children or maintain ties with parents, partners, or children. The most important grouping to \pShippies{} is which ship they call home.

\section*{Religion and Morality}
The \pShippies{} primarily serve the dual \cEbb{\God}, \cEbb{} and \cFlow{}. \cEbbFull{\MYname}, is also primary keeper of ships, and the caretaker of things lost or taken away. \cFlowFull{\MYname}, is also primary keeper of the rain, and the caretaker of things created or given. They preside jointly over the sea and their people.


\subsection*{Morals}
The twin \cEbb{\God}es demand a morality primarily of balance, with Faithfulness and Loyalty as secondary considerations. Give and take is fundamental to the \pShip{} viewpoint on the world. Whatever you get, you are expected to give back in equal or greater measure. In theory, this ought to create a fair and equitable society, but in practice, it doesn't. Those who have, give to each other, containing their wealth and indeed multiplying it by passing it amongst each other. The rest struggle to make ends meet. 

There is also an ebb and flow among the \pShippies{} of good and ill will. If someone does you a bad turn, you are expected to harm them in return. This creates localized escalations, and they can spread from these points of conflict, like ripples on a pond, to encompass vast swaths of \pShip{} territory. Eventually they die down. After all, the ocean also washes everything clean, given enough time, but great violence has been done under the banner of ``an eye for an eye.'' There are even folk tales of murder done in revenge for another murder not carrying the same consequences. But that's quite a stretch, even for \cEbb{} to tolerate.

%update!
Loyalty to your ship has become the bedrock of society, and while it has brought many ships through hard times, it has also left abusive captains in power for far too long. A true follower of the twin \cEbb{\God}es should know the difference between loyalty to the current captain, and faithfulness to the proper way of things between a captain and their crew. But in reality, it's harder to spot than anyone cares to admit.

<insert the third tenant>

\subsection*{Seeking a Goddess' Approval}
As \cEbb{} and \cFlow{} are Patrons of \pShip{}, they are called up on regularly by their people to judge actions and consult in decisions. A simple invocation of the \cEbb{\God}' name, followed by an out loud statement of the problem is generally considered sufficient. A small offering, such as the burning of a length of cord, or tossing a bit of fish overboard is included for especially important matters. A breeze stirring to blow upon your face is a sign of approval. A cloud over the sun or the moon is considered disapproval. Anything else, including apparently nothing changing, means you are on your own. Still, you've done your due diligence in the matter, and you should proceed according to your best judgment.

\subsection*{Creation Myth}
The most pervasive creation myth in \pShip{} is as follows:\\
\emph{When the world was complete, \cEbb{} and \cFlow{} wanted beings in their own image on Cengea. They came down to the earth on the Western shore of Cengea. They walked out across the waves, each in a different direction - \cEbb{} to the north, and \cFlow{} to the south. Only one sliver of the ocean around Cengea, \pWod{} did not feel the trod of the \cEbb{\God}es' feet. In each place the \cEbb{\God}es stepped, up rose an island - land that was to be sacred to their people. They walked for three days and three nights, and finally met on the Eastern coast of Cengea. There they sat to rest on an Island that grew up from the sea.}

\emph{\cFlow{} began to build with the sand, but the dry sand would simply collapse - refusing to hold it's shape, even for a \cFlow{\God}. \cEbb{} observed this, and drew the waves up the beach until the sand \cFlow{} was working with was soaked. The two of them knelt and built statues of a small group of humans together. \cFlow{} raised the Sun, and it dried the sand, creating firm bodies that would not be washed away by the water. \cEbb{} raised the Moon, and it's soft light brought peace and wisdom into the statues. Both \cEbb{\God}es called the wind, and it blew wanderlust into the heart of the \pShippies{}. Though the \cEbb{\God}es were tired they looked upon their creation with joy.}

\emph{The \cEbb{\God}es took each other's hands and prepared to return to the place of the Gods, but the Human's they had brought to life cried out, and begged the \cEbb{\God}es not to abandon them. The \pShippies{}'s plea moved the \cEbb{\God}es, and their soft hearts ached. It seemed that nothing could be done, as the \cEbb{\God}es could not stay on on Cengea, and the humans could not return to the place of the Gods en masse. Very little stops a \cEbb{\God} for long though, and \cEbb{} and \cFlow{} soon had a solution. They took up a fistful of sand, and threw it to the night sky. There the grains lodged as the first stars. If a \pShippies{} lives their life right by the \cEbb{\God}es, and by each other, then they will have a place among the stars, alongside the moon and sun, and \cEbb{} and \cFlow{}.}

\subsection*{Magic}
The magic of \pShip{} is a subtle one. It lives in the hands of the shipwright, the fisherfolk, the harpoon-maker, and the cooks. Their magic is tied to the sea, and tied to the intentions and outcomes of the items made. So subtle is their magic that there are some in \pFarm{} and \pTech{} that look down their nose at the \pShippies{}, and claim their magic of no consequence. But for those with eyes open, the magic of the \pShippies{} makes daily life possible in Cengea.

\section*{Culture}
The \pShip{} culture is one of wanderlust. Very few \pShippies{} stay in the same place, or on the same ship, where they were born past their teens. Adventurous spirits pervade, and detailed planning is eschewed in favor of following where the wind takes you. Camaraderie among a crew is paramount, with fellowship within a chosen family holding much stronger sway than any blood tie.

\pShippies{} on the whole make very little visual art. They tend toward song, dance, and storytelling as forms of expression.

\subsection*{Taboos and Cultural Norms}
Naming among the \pShip{} is a simple enough affair. Each person is given a single name by their biological parents, which they may change at any time, assuming the \cEbb{\God}es approve the change, and there are none others by the same name on their ship. Each \pShippies{} then takes the name of their ship, and the fleet to which it belongs, as a suffix. I.e.: \cLoud{\full}.

%old ships -> houses; uise mre durable matereals
\pShippies{} do not do well with being confined indoors. Most buildings made by \pShip{} hands have huge windows, patios, and courtyards. They use light and airy building material wherever the weather allows, including walls that fold away, opening entire rooms to the weather outside.

\section*{Important Figures}
The Current Council of Storm-Watchers:
\begin{itemize}
	\item \cLoud{}
\end{itemize}

\emph{Players are encouraged to invent new NPCs as needed, for example, very few of the Storm-Watchers are pre-defined.}

\section*{Unusual Biological Features (if any)}
Most \pShippies{} can hold their breath for upwards of 15 minutes.

\section{Educational system:}
%%Ideal to study in 2 academies, both in different fleets from where you were born? Academies as trade schools
Education up through the 6th grade is handled aboard the ship or in the housing complex where the child was born or is currently living. Two additional years of schooling are required, in one of the 12 academies (one per fleet). Ten of the academies have acceptance criteria, of various stringencies. The Academies in Fleets 8 and 9, which share a border defined by \pWod{} have no acceptance criteria, and so primarily teach the dregs of the barrel. The academies offer advanced study courses beyond the 8th year, but attendance dwindles dramatically with each additional year.

\subsection*{College attendees:}
The \pShip{} \pSchool{} students are elected. Starting in their 7th year of schooling, students may petition their academy to sponsor them for the election. If selected, the student(s) will be assisted by the school in running a campaign across the nation. The Council of Stormwatchers will hand down a decision on how many students should be sent that year, after consulting the portents, and often taking council with the \cEbb{\God}s themselves. A vote is held across the nation to select the incoming class. As many as 20 students, and as few as 1 have been sent to the \pSc{} historically.

\section{Technology:}
%%Ship sail powered
%%Navigation
%%weapons
%%food preservation

\section{Costuming Related Items:}
Nautical themed costuming.



\end{document}
