\documentclass[blue]{GL2020}
\begin{document}
\name{\bAgrarians{}}
%% Vary sentence length to improve readability
\section*{Form of government:}

\pFarm{} are ruled by a monarchy and are currently under the governance of \cQueen{}, considered by most to be a wise and benevolent leader.  With \cQueen{\they} Consort at \cQueen{\their} side and advised by a panel of Councilors, she presides over the noble Lords and Ladies of the country - wealthy land owners and mages.  She passes down the laws of the land and ensures that the nobility uphold their obligations and responsibilities toward their people and their lands.

The system of government is such that the eldest child of the reigning monarch inherits, but whoever they marry doesn't automatically take on the title of ruler, but instead becomes a Consort of the current ruler.  Upon the reigning monarch's death, the next closest kin inherits.  Should the partnership be dissolved, the Consort does not retain any titles of King or Queen, and any future children they have are not considered eligible for the throne.  For the purposes of inheritance, gender is irrelevant.

The Nobility preside over their lands and the people who live on and work them, none of whom own any land themselves.  For the purposes of day to day management, the Nobility typically designate one or more trusted individuals to oversee their lands during their absences.

%KELSEY: Add in the power of the nobility to rule their own spaces as long as they don't contradict the queen's edicts

\section*{Names of major players:}

\cQueen{}:\\
\cConsort{}:\\
Queen's Children:\\
Notable Councilors\\
Notable Nobility\\
Notable Overseers\\

\section*{School attendees:}

The majority of the students sent to the \pSchool{} are the children of the reigning monarch or those of the Nobility. Occasionally, however, a child with magic will be born to a working class family.  When these children are discovered, they are often taken in by the nobility for tutelage.  On the rare occasion that they show especially promising talents, they sometimes will be sponsored to go to the \pSchool{}.

%Representatives?

\section*{Geography:}

\pSunCh{} are a people gifted with plenty.  Bounded by mountains to the North and ocean on every other side, they live on a large watershed ripe with farmlands and teeming with birds and wildlife.  

%Insert more information once map is finalized. Also mention weather

\section*{Major industries, Economy, and Trade:}

The major industry of \pFarm{} is agriculture - with their vast, rich farmlands, enhanced every spring by the mages amongst the nobility and aided by the technology produced by \pTech{}, the \pFarmers{} produce enough food to feed not only themselves but also  \pCreators{} to the North.  Until the recent breakdown in trade with the \pVikings{}, they produced enough surplus to supply to their nation as well.  In return for the traded food, they receive technology from \pCreators{} that allow them to continue their mass production of food. %add in the relationship wit the Vikings from before the isolation. Fish for fertilizer, protection from serpents, trading and escort.

It is a time of plenty for \pSunCh{}, but the farmlands which were once fertile have become weary from decades of constant, unsustainably exuberant growth.  Without the technology of \pTech{} and the flood of magics from the nobility of \pFarm{} every year, the land would collapse into dustbowl and famine.

\section*{Politics, International relations, and Immigration:}

The alliance between the \pSun{} and \pTech{} has lead to an unprecedented period of growth for the \pSunCh{} - it has been nine years since they were last hit by the storms, and the guaranteed safety from the storms has meant that they have been able to dedicate themselves to agricultural expansion and improvements, rather than focusing on preparations and fortifications for the next coming storm.  Should things continue as the leaders of \pTech{} suggest, with the magic and technology of \pTech{}, soon the storms will hopefully be rectified altogether.  It is unfortunate for the \pShip{} that they will have to bear the brunt of the storms until then, but they don't have the responsibility of providing food for multiple nations or producing lifesaving technology, and aren't as dependent on magic in the first place.

Relations with the \pShip{} have been rocky since the Alliance was severed.  The leaders of the \pShip{} are threatening to pull out from the small settlements and stop protecting them from the sea dragons if not appeased.  A few of the Noble families whose lands border the coast and the \pShip{} settlements have attempted to speak for them to \cQueen{}, but so far \cQueen{\they} has decided that the benefits of the continued alliance with \pTechies{} and continued safety of the \pFarm{} from the storms outweighs the cost to the \pShip{} and potential loss of protection.

%Wordsmith a little the sentence that starts ``It is frowned upon''.
Immigration to \pFarm{} is uncommon, but welcomed by them.  Unlike the stringent requirements of the other two nations, the \pFarm{} will take you as you are, so long as you are willing to work and work hard.  Unless the person migrating is possessed of a great skill or talent that is considered of cultural use, they will typically find themselves working the lands.  Immigration away from \pFarm{} is more rare.  It is frowned upon by the \pFarm{} to travel to \pTech{} for the regular auditions - gifts of magecraft are needed to ensure the seeds sprout every spring, and gifts of music are something that is improper and vaguely sacrilegious for those outside of the priesthood to use.  As for immigration to \pShip{} - it is a fantasy worthy of children to want to run away to a \pShippies{} ship, but the realities of \pShip{} life are hard and dangerous, and who would want to trade a peaceful life on solid ground under the gifts of the Sun for the unpredictability and danger of the sea?  Those who fail at both endeavors must return home to the lost standing in the eyes of their families and neighbors.

\section*{Brief History of the country:}

%Current history: Things to potentially include?
 - Oldest child of the Queen is widely disliked - doesn't do much good for the people, generally considered to be a waste of space.  Youngest daughter is widely beloved, but she is too far away in the succession path to inherit.  There are whispers amongst the people of a plot to put the youngest daughter on the throne. %%What is the precedent? has this happened before?
 - Uprising at one of the farms and resulting dust bowl - Queen ruled against the farm workers.  She ruled that the overlord was in the wrong for mistreating their people so, but that it still does not warrant murder.  She does not grant the lands to a new mage / noble, but lets them lie dead as a reminder of what happens when the finely tuned balance is disrupted
 - The largest of the hummingbirds have gone missing and have not been seen since T-6 years, the first year the \pShip{} were hit out of turn.  There are whispers in the Monastery that the God is signaling disapproval for the new pattern of the storms / the focus on the \pShip{}. %Has this ever happened before.

%History of when the storm has hit your nation, and what that changed for your nation each time. Separate document for the whole game that just has the timeline

Storm Hit These Locations starting with present year:\\
Present year - happening, don’t know where\\
T-3: Vikings\\
T-6: Vikings (was Technocracy’s turn)\\
T-9: Agrarian\\
T-12:Vikings\\
T-15:Technocracy\\
T-18: Agrarian\\
T-21: Vikings\\
T-24: Technocracy\\
T-27: Agrarian\\
T-30: Vikings\\
T-33: Technocracy\\
Previous - random or whatever politicking at the time, don’t know yet, or even same cycle but broken occasionally\\

\section*{Educational system:}

The wealthy offspring of the Nobility are educated by private tutors, hired to give them a thorough grounding in a variety of subjects, including magic, religion, reading, writing, geography, history, mathematics, and governance.  %Where do the tutors come from? Federation? priesthood? younger children of nobility.

Many overseers see themselves as the caretakers of the people who live and work their lands.  As the priests say, the working class are the closest to the earth and are therefore closest to the gods, even if their work is full of hardship.  The farmworkers are educated by members of the priesthood.  Lessons are given to children until the age of twelve, when the majority join their parents in learning their trades.  A select few join the priesthood and continue their lessons - because they feel called to the work, because they are scholastically inclined, they don't want to work in manual labor all their lives, have a disability that prevents such work, or to fulfill a love of art or music by channeling it through service to the Gods.

\section*{Cultural foods:}

The food of the \pFarm{} is a food rich in herbs and spices.  While they may preserve some herbs and spices for sale and shipment to the wealthier people in \pTech{}, the goods do not maintain their full flavor after the preservation process.  Generally one has to travel to \pFarm{} to experience the full depth of flavor that can be found in the dishes of even the most humble of homes.  The food of the \pFarmies{} is also mostly vegetarian, due to the prohibition of their God against the killing or harming of other nonhuman beings.  The people who live along the coasts may trade locally for fish with the \pShip{} settlements, but otherwise none of the \pFarm{} consume meat.

\section*{Gender, Sexuality, Relationships, and Family Structure:}

For \pFarm{}, gender is largely irrelevant when it comes to social standing or societal roles.  For the noble classes, children are expected to pair with at least one partner who is compatible for procreation, said partner also expected to be gifted in magecraft.  Though this partly stems from desires to continue on the family lines, it also stems from a need to provide for the futures of their lands and their peoples - without mages, the seeds will not sprout in the spring and the land will collapse into famine.   %%Older strong magic to inherit, younger strong magic married out. 2 low magics married join one or the other estate.

For the working classes, family structures are more flexible.  Children are considered a communal responsibility and resource - adults are free to partner with whomever they wish, but it is frowned upon to choose to be childless.  With the majority of adults working on the farm during the daylight hours, caregiving responsibilities are shared throughout the community.  Single, double, and multiple adult households are all common.       

\section*{Cultural taboos / social norms:}

%Leave parts of it up to players
%Frameworks for things that players should know

%Did I cover enough of this in other sections that I don't need to add more to this??
Vegetarian
don't hurt animals
have lots of children

\section*{Costuming Related Items:}

The \pFarm{} amongst all social strata are a people in love with color - all members of society, from the poorest farm worker to the wealthiest among the nobility, dress in a multitude of gem-like hues.  Dye-making is almost an art amongst the working class, and special washes are brewed from local plants to make colorful garments difficult to stain.  Embroidery and needlework are pastimes for winter months, and garments are often embellished.  Common motifs include flowers, leaves, and the sacred hummingbird.

\section*{Magic:}

In \pFarm{}, magic is the gift and responsibility of the nobility. Generally only members of the noble classes have magic, and care is taken by each family to ensure that this gift continues amongst their children.  It is common for courtship rituals amongst the nobility to include demonstrations and displays of magical talent.  On the occasion, a child with magical abilities will be found amongst the working class - perhaps from an extraordinary ability to coax life into stubborn seeds or from a demonstration of healing when one of the farm animals or other farm workers becomes injured or sick.  In this case, the child is usually taken into the household of the overlord or local nobility for education, and if they are in very good standing with the landholder, may be sponsored to go to \pSchool{}.

This gift is one that comes with a heavy responsibility - that of ensuring that all people in multiple countries remain fed.  Every spring, the mages of each fiefdom go down to the farmlands to push their magic into the lands and seeds prepared by the farm workers.  Each seed is imbued with magic, allowing them to sprout and causing them to grow two to three times as quickly and as bountifully as they would otherwise do.  Every plant is pest free, every fruit and vegetable free from fungi or rot that would blight the harvest.  In times of severe injury or illness, the overlord and mages in the keep will also be sent for to provide healing.   

Before the fertilizers produced by \pTech{}, the lands were allowed to fallow and rest between seasons to rebuild the nutritional value of the land.  Since the trade relationship was struck, however, that has no longer been necessary - the fertilizers replenish what nutrients are removed from the soil each year, allowing an increase in farming capacity that has caused \pFarm{} to comfortably be able to provide for the agricultural needs of all three nations.

%Can we soften this just a little bit? maybe indicate that we theorize that this will happen.
However, their farmlands are now dependent on this magic - the land is exhausted, and only the magic continuing to flow into the land keeps it alive.  Were the magic to cease to flow into the land, all plant life would die and the dirt would turn to dust.  There are stories told of when this happened once before, when an overlord neglected his people and they rose up and killed him.  Without mages to return magic to the land, the land turned to dust, and those who could not find homes in other fiefdoms starved.
 
\section*{Religion:}

Hummingbirds are considered sacred to the \pFarm{}, for they are the oracles of the many-faceted God of \pFarm{}.  Ranging in size from the nail of one's thumb to the size of small houses, their song contains the messages of the God.  Monasteries plant special flowers and sweet-smelling herbs known to be especially pleasing to hummingbirds, and many in the Priesthood study music their entire lives to commune with the oracles and please the God with their songs.

All living creatures are under the care and protection of the God, from the lowliest pill bug to the most majestic stag.  None are permitted to be harmed or killed by \pFarm{} without facing severe punishment from the God.  Because of this, the people of \pFarm{} work diligently to maintain the ecological balance of their farmlands, ensuring that aphids and other living pests are kept under control by their own predators such as ladybugs and birds.  %prop up this part with magic too.

\section*{Crime and Punishment:}

While small crimes and misdemeanors are punished by society, there are some crimes which are considered so blasphemous by the God of \pFarm{} that, if they are committed, the world itself will enact punishment regardless of whether the crime was observed or not.

Although all Gods punish murder of humans with the removal of one's conscious memories, the God of \pFarm{} considers all living creatures to be worthy of such protection.  Birds, fish, mammals, bugs - all creatures which walk, crawl, fly, or swim are precious to the God.  Those \pFarm{} who live near the settlements of the \pShip{} and eat fish are careful not to trade for live fish, but to only accept fish post-death.

Abuse to animals is also strongly punished.  Should a \pFarmer{} starve their animals, kick their pet, beat their livestock, they will receive the same as they laid upon their animals but tenfold.  Whispers are told of people who squashed a beetle on their plants, only to have a giant bolder hurtle from the sky and squash them where they stood.  %%Add an intention clause. Age clause; mild consequences 5-10.

\end{document}

%%Hedge-witches sell curses and blessings - weak magical talent among the farmers.
%%Can you have a God that picks and chooses who you can kill? can you make someone lose the protection/favor of the god
