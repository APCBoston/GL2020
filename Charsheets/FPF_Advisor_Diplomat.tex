\documentclass[char]{GL2020}
\parindent=0pt
\begin{document}
\name{\cDiplomat{}}

You are \cDiplomat{\full} (\cDiplomat{\they}/\cDiplomat{\them}), foremost Diplomat of the \pTech{}, attending the \pSchool{} on this important weekend as an Advisor for the second time at age 52.  At the core of your success is your single guiding principle -- loyalty to your nation and its interests above all else. When you speak, others listen, because the weight of the governing Council and your Faledon connection grants power to your words. This weekend there is nobody attending with a more solid position than your own -- or at least that's the impression you'd like to give off. Privately though, it's beginning to be a bit of a stretch to keep everything from falling to pieces. People just can't seem to be satisfied leaving important matters in your capable hands, and just have to make a nuisance of themselves. But you aren't the most powerful diplomat in the three nations for nothing, and there's no way you're going to let people wrest control from your grip.

You've accomplished great things over your storied career, but the accomplishment – as well as the horror – of six years ago looms largest in your mind. In the defining moment of your career and the deal that made your name an international one, you established an unprecedented treaty with the \pFarm{} and cemented an alliance, turning your two nations temporarily against the \pShip{} for the ultimate benefit of all.  This new treaty dictated that the Storm shall henceforth always be directed towards the \pShip{}. The \pFarm{} will grant significant deals to your own nation on food and goods, and in exchange \pTech{} has devoted its resources and best scientists, including \cHeadScientist{\full}, to developing technology that will permanently eradicate the Storm itself. The treaty came too late for the Time of Deciding six years ago, however, and children are notorious for not following directions, so you conspired with \cEvil{\full}, a ruthless courtier from the \pFarm{}, to ensure that the Storm would be sent to the \pShip{} that year, no matter how the children voted. To this day, you do not know exactly how that was accomplished – \cEvil{} handled the specifics – but \cEvil{\they} did tell you that \cHeadScientist{}, a junior scientist at the time, present as an advisor that year, was a crucial part of the plan (one advisor slot is always given over to an ``up and coming'' person, to reinforce the dream of upward mobility within the \pTech{}). Afterwards, you rewarded \cHeadScientist{} with a promotion to head scientist of the project to end the Storms; this had the added bonus of placing \cHeadScientist{\them} in your debt, and somewhere you could keep an eye on \cHeadScientist{\them} in case \cHeadScientist{\they} started talking too freely.

But no silver lining is without its cloud. After the ritual to control the Storm was completed that year, the 12 students adjourned from the Great Hall to the Student Lounge for a private toast without the teachers and advisors breathing down their necks. After nearly an hour of waiting, you, Principal \cPrincipal{\full}, and \cMusic{\full} (one of the teachers) decided that you had better go check on them. What you found was 11 dead bodies. But where was the twelfth student, \cKidScientist{\full}? A quick inspection indicated that it was probably the wine that killed the children. While your two companions stared in horror, your mind worked furiously. This must have been \cEvil{}'s doing, to tie up loose ends and make sure the students weren't around to protest that their votes had been altered. Damn \cEvil{\them} for taking such a drastic action without consulting you! Even worse, \cHeirSibling{\full}, heir to the powerful Faledon family, was among the dead. This would require serious damage control, and no one could ever be allowed to find out that you were involved. You were shaken from your reverie when \cPrincipal{} and \cMusic{} proposed keeping the fact that \cKidScientist{} was not among the dead a secret. They believed that \cKidScientist{} must have survived and, not knowing who to trust, fled for \cKidScientist{\their} life. Keeping it out of public knowledge that \cKidScientist{\they} escaped would avoid anyone continuing to hunt \cKidScientist{\them}. Little did they realize that one of those would-be hunters stood in their very midst. At your earliest convenience, you told \cEvil{} that \cKidScientist{} had survived, and after you chewed \cEvil{\them} out for the huge mess \cEvil{\they} made, the two of you agreed that \cKidScientist{\they} must be found and silenced. Unfortunately, all your efforts to find \cKidScientist{\them} have been fruitless. Perhaps whatever \cEvil{} did just took longer to affect \cKidScientist{}, and \cKidScientist{\they} didn't survive after all.

Regardless of the lack of proof, the timing of the events of six years ago, and your presence on the scene, has since caused you to fall under suspicion. Not that anyone would dare show their suspicion to your face, but\ldots{} you swear you can \emph{feel} them thinking it. You'll need to keep a sharp eye out for anyone looking into the deaths of the students six years ago, since it's very possible that you would be implicated by whatever such an investigation uncovered. Of course, you'll want to seemingly support any investigation, as it's the ``just'' thing to do -- but behind the scenes, you'll need to see that it goes nowhere. If the worst should happen, you can always throw \cEvil{} and \cHeadScientist{} under the cart, but they are both useful tools whom you would prefer not to have to expend. You'll also have to stay vigilant for the possibility of either of them betraying you to cover up their own guilt if the plot is uncovered. What a mess.

On a much more cheerful note, you are about to see something come to fruition that you've been working on for a long time -- a union between \cHeir{\full}, current heir to the \cHeir{\formal} household and younger \cHeir{\sibling} of the late \cHeirSibling{}, to \cChupStudent{\full}, \cChupStudent{\child} of a highly regarded noble family from the \pFarm{}. \cChupStudent{}'s family is close to the crown and holds a crucial territory adjoining the border with the \pTech{}. Getting them to agree to marrying off their second born \cChupStudent{\child} to the \cHeir{\formal}s was one of your first strokes of political genius some 17 years ago. With this connection forged, the \cHeir{\formal}s can finally get a serious foothold in the \pFarm{} and start expanding their influence -- and by extension, yours -- internationally. \cHeir{} has, as usual, been rebellious about the whole thing, but that's not particularly new.

You've known \cHeir{} all \cHeir{\their} life, and \cHeir{\they} \cHeir{\are} constantly complaining about having no control over \cHeir{\their} choices or dragging \cHeir{\their} feet around family business. It's exasperating, to say the least! That's why this time you've come prepared. You've gotten both of their families to organize a surprise wedding, and have delivered letters just this morning to both students, informing them of such. This closing of the trap all at once was a well-played move on your end -- but something about the way \cHeir{} reacted to the news left you with a nagging feeling that something wasn't quite right. You had better make sure that neither \cHeir{} nor \cChupStudent{} try anything funny to weasel out of the wedding, and that \cHeir{} upholds the \cHeir{\formal} family's dignity. A wedding should also help keep \cHeir{} distracted from digging up any uncomfortable truths about the death of \cHeirSibling{}. 

With the wedding imminent, \cHeir{}'s \cFaledonParent{\parent} \cFaledonParent{\full} has asked you, in your frequent role as \cFaledonParent{\their} representative, to perform a brief ritual this weekend. Now that \cHeir{} has come of age and will be graduating soon, you are to pass on the Heir's Seal signet ring so that \cHeir{} can begin conducting family business under \cFaledonParent{}'s guidance once school is finished. When to perform this ritual is a balancing act. On the one hand, getting it out of the way as soon as possible would allow you to focus on other important matters about the Time of Deciding. On the other hand, the longer you wait, the more of \cHeir{}'s accomplishments you'll have to celebrate. Assuming of course \cHeir{} can be convinced to partake in any tasks that might be construed as accomplishments. Lazy child.

Speaking of other important matters, you have been honored with bearing one of your nation's Relics, \iMirror{}. While the Mirror may or may not ultimately be used for the ritual, as there is an alternate Relic from your nation, the \iLariat{}, in the library, it is still your task to see that the mirror doesn't fall into the wrong hands or be misused. Even more importantly, it is your responsibility as an Advisor to ensure that the right students get the most votes to direct the Storm, and that everyone votes correctly. Naturally, you'll see to it that \cHeir{} holds the premier position among the \pTech{} students in the ritual and votes to direct the Storm to the \pShip{}. This will uphold your treaty with \pFarm{}. Although \cHeir{} has always been somewhat of a lazy, disinterested child, \cHeir{\they} \cHeir{\have} generally done as you directed in the past, given enough prodding. Still, you haven't seen much of \cHeir{\them} since \cHeir{\they} started at \pSchool{} a couple of years ago, and you hope for the sake of the Faledon family's business interests that \cHeir{\they} \cHeir{\have} matured some in the interim. In order to make sure \cHeir{} gets the most votes, you may need to have a \emph{discussion} with your fellow advisors, most particularly \cAntiChup{\full}, who has a tendency to favor \cAntiChup{\their} own protégé, even at the expense of what is best for the nation.  

Speaking of treaties, you've been informed recently that certain parties, specifically \cHeadDiplomat{\full}, premier Diplomat of the \pShip{}, are trying to broker a ceasefire to end the war between the nations. While this effort is commendable in principle, in reality this is likely to result in disaster, as any new treaty ending the hostilities would inevitably be bought at the price of your alliance with the \pFarm{} being abolished -- which would mean that the Storm could again be pointed at \pTech{}. Such an occurrence would be devastating for your nation, as there have been no preparations made to receive the Storm. Such expensive and time consuming work was stopped as part of the alliance you brokered, so that the resources could be diverted towards the more permanent solution of ending the Storm for good. The \pFarm{} is similarly unprepared to receive the devastation of the Storm cutting a swath through significant portions of their vital crop land. Such a thing would inevitably cause them to hoard their supplies, and this would in turn bring famine to your own people. It has been your life's work to protect your beloved nation from the Storm, and you have succeeded. The short sighted fools around you cannot see that you are ultimately bringing that protection to all of \pEarth{}; it just takes a little sacrifice along the way. Your people have thrived in the safety you have earned them, and the casualties of war are paltry in comparison. There is no way you are going to let \cHeadDiplomat{} or anyone else broker a ceasefire that would ultimately threaten your people. You will still have to sit in on any treaty negotiations to make sure the \pTech{} comes out on top of anything that \textbf{is} discussed. If the other negotiators insist on a new treaty, and you cannot prevent it, you can at least ensure that you are named as the Mediator of any such treaty, or are known as the primary author. Ultimately, though, you hope to end the war in other ways. 

One such way is supposed to come to fruition this weekend, when your scientist, \cHeadScientist{\full}, and \cHeadScientist{\their} assistant \cAssistantScientist{\full}, are scheduled to present their research into ending the Storms. You scheduled this presentation months ago, to provide the team a little extra motivation finalize this project that has already dragged on for more years than you hoped. A permanent end to the Storms, and the revelation that the sacrifice of the \pShip{} was justified, will make the need for a new treaty about wher to send the storm obsolete. You intend to do everything you can to make sure the presentation goes flawlessly. 

But it pays to take a multi-pronged approach, and you suspect that the actually ending this war will require eliminating the leader of the \pShip{} war efforts. \cLoud{\full}, probably seen as a hero by the \pShippies{}, is a warmongering murderer bent on the destruction of your homeland and the \pFarm{} alike, and must be disposed of immediately. You've learned that your old co-conspirator \cEvil{} is interested in facilitating such a\ldots{} removal. As much as you distrust \cEvil{} and are wary of \cEvil{\their} methods, you make it a policy never to set aside a useful tool -- and \cEvil{} is definitely that. Your nation comes first, whatever the cost, and you'll do whatever you can to aid \cEvil{} in this task and ensure the success of this mission. Although you wouldn't mind doing it in such a way as to keep your own hands clean, should the plot be discovered.

They say there is no rest for the weary, and that will certainly be true for you this weekend. So much to do! But you are used to juggling numerous political and social intrigues with ease, and have no doubt that your experience and guile will win out in the end.  You'll ensure \cHeir{} is married off and voting according to the treaty, arrange the assassination of \cLoud{} and more\ldots{} all while ensuring that nobody links you with the death of the students six years ago.

\begin{itemz}[Goals (in roughly descending order of importance)]
	\item Support the work of your scientists, \cHeadScientist{} and \cAssistantScientist{}, in whatever ways they need. Their research into ending the Storms is the result of everyone's sacrifice, and you want to make sure their presentation goes off without a hitch. 
	\item Prevent people, and especially \cHeir{}, from digging too deeply into the deaths of the students six years ago, while making it seem like you support such investigations.
	\item Guide \cHeir{} in \cHeir{\their} development into the next Faledon head of house. Arrange for \cHeir{\them} win the most voting authority and that \cHeir{\they} and the other students vote for the Storm to go to the \pShip{}. Ensure \cHeir{} doesn't do anything to interfere with \cHeir{\their} wedding to \cChupStudent{} (you'd better keep an eye on \cChupStudent{} too while you're at it). And lastly, conduct the ritual to pass down the Heir's Seal signet ring to the Faledon Heir.
	\item Arrange to assassinate Warlord \cLoud{} with \cEvil{}'s help.
	\item Either ensure the treaty negotiations fail, or position yourself as the designated Mediator (or primary author) in any treaty likely to be ratified by the governments of the three nations.
	\item Ensure that \iMirror{} remains safe, attuned to the \pTech{}, and under your control.
\end{itemz}

\begin{itemz}[Notes]
	\item International espionage is a thriving trade these days, and given the pivotal events happening this weekend, you would not be at all surprised if spies from all three nations are present. 
	\item The Order of the Black Crocus is, or rather was, a secret international law enforcement entity with jurisdiction across all three nations. As far as you know, it splintered into factions when the war started, each faction subsumed by the intelligence services of its corresponding nation. Good riddance, as far as you're concerned. They were a meddlesome bunch.
\end{itemz}

\begin{contacts}
	\contact{\cEvil{}} A fellow diplomat from the \pFarm{}, and your ostensible ally -- talented at what \cEvil{\they} \cEvil{\does}, but with highly questionable morals and methods. Your interests align on certain goals, but you trust \cEvil{\them} about as far as you can throw \cEvil{\them}.
	\contact{\cHeir{}} The current, lazy, unmotivated, troublesome Heir of the Faledon family. If only \cHeir{\they} had the drive of \cHeir{\their} older sibling, \cHeirSibling{}, one of the students killed six years ago.
	\contact{\cChupStudent{}} \cHeir{}'s future spouse. The child of a powerful \pFarm{} family along the border with \pTech{}, \cChupStudent{\they} \cChupStudent{\are} an excellent addition to the Faledon family, and arranging the match was a major feather in your cap.
	\contact{\cAntiChup{}} Your advisor counterpart at the negotiating table. Where you represent the government, \cAntiChup{\they} represent\cAntiChup{\plural} the Temple. A \cAntiChup{\clergy} of considerable reputation and an inconvenient tendency to favor \cScholarship{\full} over \cHeir{}.
	\contact{\cPrincipal{}} Was your favorite teacher at the College of the Gods when you were a student, and has since become a friend. Currently Principal of the school.
	\contact{\cHeadScientist{}} The lead scientist on the project to end the Storms forever — capable, brilliant, and reliable. Played a role in the covert operation that sealed the alliance with the Children of the Sun six years ago that sent the Storm at the \pShip{} out of turn.
	\contact{\cAssistantScientist{}} \cHeadScientist{}'s research assistant. Seems competent enough.
\end{contacts}

\end{document} 


