\documentclass[blue]{GL2020}
\parindent=0pt
\begin{document}
\name{\bAgrarians{}}

The \pFarmers{} are a people gifted with plenty.  Bounded by mountains to the North and ocean on every other side, they live on a large watershed ripe with farmlands and teeming with birds and wildlife.  The Mediterranean sun blesses the land and those who work it with \cFarmGod{}’s light.   

\section*{Form of Government}
The \pFarm{} are ruled by a monarchy and are currently under the governance of \cQueen{\full}, considered by many to be a wise if not strictly benevolent leader.  Those who think otherwise have an unsettling tendency to disappear or meet with fatal ``accidents.''  With \cQueen{\their} Consort at \cQueen{\their} side and advised by a panel of Councilors, \cQueen{\they} preside\cQueen{\plural} over the Nobility of the country -- the wealthy landowners and mages.  \cQueen{\They} pass\cQueen{\pluralC} down the laws of the land and ensures that the Nobility uphold their obligations and responsibilities toward their people and their lands.

The rules of succession dictate that the eldest child of the reigning monarch inherits.  Whoever they marry doesn't automatically take on the title of ruler, but instead becomes a Consort of the current ruler.  Should the partnership be dissolved, the Consort does not retain any titles they did not hold independently, and any future children they have are not considered eligible for the throne.  For the purposes of inheritance, gender is irrelevant.  In recorded history, there have been a few times when a younger child of the reigning monarch is deemed the most fit to be Heir and the eldest has been passed over, but such events are rare and tend to cause unrest.

The Nobility preside over their lands and the people who live and work there, none of whom own any land themselves.  For the purposes of day to day management, the Nobility typically designate one or more trusted individuals to oversee their lands during their absences.  On their own lands, the Nobility's words are law – so long as they don't contradict the \cQueen{\Majesty}'s edicts.  Nobility are even in charge of how the Priesthood interacts with the commoners to some extent, so services such as education and healing provided by the Priesthood vary from one region to the next.

\section*{Major Industries, Economy, and Trade}
The major industry of the \pFarm{} is agriculture -- with their vast, rich farmlands, enhanced every spring by the fertility mages amongst the Nobility, the \pFarmers{} produce enough food to feed not only themselves but also the \pTech{} to the North.  In return for the traded food, they receive technology from the \pTech{} that allows them to continue their mass production of food.  Until the war with the \pShippies{} began, the \pFarm{} produced enough surplus food to supply their nation as well.  In return for food, they were provided with fish and guano for fertilizers, protection from sea serpents, and transportation of their goods.  While the newest innovations in farming technology from the \pTech{} make up some of the difference, the fertilizers from the \pShip{} are sorely missed.  Times have been particularly difficult for the seaside holdings between serpent attacks and \pShip{} raids.  And across all of the \pFarm{}, the farmlands which were once natively fertile have been depleted from decades of constant, unsustainably exuberant growth.  Without the technology of the \pTech{} and the flood of fertility magics from the Nobility every year, the land would promptly collapse into dustbowl and famine.

\section*{Politics, International Relations, and Immigration}
Eight years ago, diplomats from the \pTech{} came to the \pFarm{} in secret with the offer of a new alliance.  On the brink of inventing tech that would allow for the permanent removal of the devastating triennial storms, the scientists of the \pTech{} needed a little more time to finish their research – time they wouldn’t have if they were occupied by preparations for the upcoming Storm.  With the promise of this new technology, \cQueen{\Majesty} \cQueen{} agreed to form a new alliance with the \pTech{}, wherein they agreed to send the Storms exclusively towards the \pShip{} so that the two nations could have the available resources to pursue this invention for the good of all nations in the long term.  In the aftermath of the betrayal six years ago, the \pShip{} declared war on both of the allied nations in hopes of claiming safe land on the continent proper and punishing those who had wronged them.

The war has come at great cost for the \pFarm{}. The \pShippies{} were far better equipped for war, since their ships were already equipped for hunting the monsters of the sea, and they are no longer protecting the \pFarm{} from sea serpents.  Coastal provinces and lands bordering major rivers have been particularly hard hit by sea serpent attacks and \pShip{} raids, with the common folk bearing the brunt of the death and destruction.  And if that were not bad enough, the \pShip{} have also unleashed their ``pirates,’’’ who seem bent not on war, but on looting, and have no qualms about killing to take what they want, amnesia or no.  A few of the noble families whose lands border the coast and the \pShip{} settlements have appealed to \cQueen{\Majesty} \cQueen{} for peace, but so far \cQueen{\they} \cQueen{\have} maintained that the benefits of the continued alliance with the \pTechies{}, and continued safety of the \pFarm{} from the Storms, outweighs the costs of the war.  Nobles who continued to speak out against the war in spite of the Queen’s decree have found themselves experiencing dwindling allies and an alarming rate of fatal ``accidents.'' 

Over time, the \pFarm{} have adapted offensive spells to defend their lands and people – using ill luck curses, charms to attract dangerous sea creatures, and other indirect means of targeting the warships to avoid incurring \cFarmGod{}’s wrath.  These curses are launched by advanced trebuchets designed by the \pTech{}.

Despite the war, the alliance between the \pFarm{} and the \pTech{} has lead to an unprecedented period of growth for the \pFarmers{} -- it has been nine years since they were last hit by the Storms, allowing them to dedicate themselves to agricultural expansion and improvements, rather than focusing on preparations and fortifications for the next Storm.  And as long as research continues as the leaders of the \pTech{} have promised, soon the Storms will be rectified altogether and everyone will be able to enjoy this prosperity. 

Given all of this, there is no immediate end in sight for the ongoing hostilities with the \pShip{}.  Though there is talk of a new treaty to end the conflict, it is hard for most to imagine what terms would be acceptable to all sides.  The \pFarm{} have made no preparations to receive the Storm, so any treaty which sent the Storm to that nation would have devastating consequences, both for the inhabitants and for the global food supply.  Nationalist fervor and hostility towards the \pShippies{} are commonplace, so potential treaty components meant to placate them, such as reparations, are controversial at best.  That said, many, particularly among the commoners, are weary of war and willing to make compromises to see it come to an end.  Above all, people hold out hope for the permanent end to the Storms that the \pTech{} has promised, believing that it will smooth the way to a peace treaty.

Immigration to the \pFarm{} is uncommon, but welcome.  Unlike the stringent requirements of the other two nations, the \pFarm{} will take you as you are, so long as you are willing to work hard.  Unless the person migrating is possessed of a great skill or talent that is considered of cultural use, they will typically find themselves working the lands.  Emigration away from the \pFarm{} is rarer.  The \pFarmers{} strongly discourage members of their society from emigrating. It is considered borderline blasphemous to travel to the \pTech{} for their regular magical auditions -- gifts of magecraft are needed at home to ensure the seeds sprout every spring, and gifts of music are sacred and contained within the Priesthood.  As for immigration to the \pShip{} -- it is a fantasy only worthy of children to want to run away to a \pShippie{} ship, but the realities of \pShip{} life are hard and dangerous. Who would trade a peaceful life on solid ground under the gifts of the Sun for the unpredictability and danger of the sea?  Those who fail at such endeavors must return home to the lost standing in the eyes of their families and neighbors.

\section*{Educational System}
The wealthy offspring of the Nobility are educated by private tutors, hired to give them a thorough grounding in a variety of subjects, including magic, religion, literature, history, mathematics, and governance.  Their tutors include members of the Priesthood and the younger children of Nobility without strong magical talents of their own.  The children of the wealthiest families are even sometimes tutored by renowned teachers from the \pTech{}.

Many members of the Nobility see themselves as the caretakers of the people who live and work their lands.  As the clerics say, the working class are the closest to the earth and are therefore closest to \cFarmGod{}, even if their work is full of hardship.  The farm workers are educated by members of the Priesthood.  Lessons are given to children until the age of twelve, when the majority join their parents in learning their trades.  A select few join the Priesthood and continue their lessons -- whether because they feel called to the work, are scholastically inclined, have a disability that prevents grueling manual labor, or to fulfill a love of music by channeling it through service to \cFarmGod{}.

\subsection*{College Attendees}
The majority of the students sent to the \pSchool{} are the children of the reigning monarch or those of the Nobility by blood.  Occasionally, however, a child with strong magical talent will be born to a peasant family.  When these children are discovered, they are taken in by the Nobility for tutelage.  On the rare occasion that they show especially promising talents, they can be sponsored to go to the \pSchool{}.

\section*{Cultural Taboos and Social Norms}
The \pFarm{} have a number of cultural norms, most of which derive from their religion.  Almost everyone is vegetarian for religious reasons, though people who live along the coasts used to eat fish caught and traded by their \pShip{} neighbors.  The food of the \pFarm{} is a food rich in herbs and spices.  Children are cherished and it is expected that if you are able to have them, you will.  If you are unable or unwilling to have children, you are still expected to participate in child rearing, and there is a lot of social pressure to adopt.  Hard work, especially of a physical nature, is also highly valued, and everyone who can is expected to do their share.  Helping and caring for others is also considered a sacred responsibility.

Every culture has taboos.  For the \pFarmers{}, one of the most important is cruelty or physical harm to animals and sentient living beings.  It is also a grave offense to create music or art outside of the Priesthood.  The classes at the College of the Gods are a rare exception, under an obscure clause about understanding the resources the students will have at their disposal as future leaders.  The music teacher, \cMusic{\full}, has a special dispensation from the Church to teach music.  Lesser taboos include choosing not to have children or be involved in child rearing, and choosing to emigrate away and abandon your family and responsibilities.

\section*{Gender, Sexuality, Relationships, and Family Structure}
For the \pFarm{}, gender is largely irrelevant when it comes to social standing or societal roles.  In some families, the firstborn inherits regardless of magical ability; for others, a sufficiently powerful younger child may have a chance of superseding their older sibling.  For the Nobility, children are normally expected to pair with one partner who is compatible for procreation, who is  also expected to be gifted in magecraft.  Sexual monogamy with said partner is socially expected, unless  the couple is unable to produce offspring, in which case discrete ``arrangements'' are made with a suitable partner. Such expectations are held most strongly for the heir, and is somewhat more relaxed for other children. This approach partly stems from desires to continue on the family lines, and partially from a need to provide for the future of their lands and their peoples -- without mages, the seeds will not sprout in the spring and the land will collapse into famine.  Unfortunately, this kind of strong magical talent is not common in children even through careful selection of partners, so the Nobility tend to have large, extended family structures.  Older strong magical talents inherit, younger strong magical talents are married to Nobility in other lands in a complex diplomatic dance.  Children without magical talents of their own find a place for themselves as a caretaker of the estate, an educator, or join the Priesthood.

For the working classes, family structures are more flexible.  Children are considered a communal responsibility – adults are free to partner with whomever they wish, but it is frowned upon to choose to be childless.  With the majority of adults working on the farm during the daylight hours, caregiving responsibilities are shared throughout the community.  Polyamory is common, and single, double, and multiple adult households are all considered normal.   

\section*{Religion}
The \pFarm{} worship \cFarmGod{}, \cFarmGod{\God} of the sun, fertility, agriculture, music, healing, and life.  \cFarmGod{}'s precepts include the sacredness of music and all life, the value of hard work and serving the community, and the importance of rearing children.  The Priesthood, keepers of the faith, are the spiritual guides, healers, and educators of the nation.  Sacred music is used to heal, and zealously guarded.  The use or creation of music by others is forbidden by the Priesthood, though the traveling circuses from the \pTech{} technically stand outside these rules, hence some of their popularity.

Hummingbirds are considered sacred to the \pFarmers{}, for they are the Avatars of \cFarmGod{}.  Ranging in size from the nail of one's thumb to the size of small carts, their song contains the messages of \cFarmGod{}.  Monasteries plant special flowers and sweet-smelling herbs known to be especially pleasing to hummingbirds, and many in the Priesthood study music their entire lives to commune with the Avatars and please \cFarmGod{} with their songs.

All living creatures are under the care and protection of \cFarmGod{}, from the lowliest pill bug to the most majestic stag.  None are permitted to be harmed or killed by a \pFarmer{} without facing severe punishment from \cFarmGod{}.  Because of this, the \pFarm{} work diligently to maintain the ecological balance of their farmlands, ensuring that aphids and other pests are kept under control by their own predators such as ladybugs and birds, and discouraged through the use of curses and substances meant to make the crops taste bad to pests without actually killing them. 

\subsection*{Magic}
In the \pFarm{}, apart from the healing magic of the Priesthood, magic is considered the gift and responsibility of the Nobility.  Publically, only members of the noble classes and the Priesthood have ``true'' magic, and care is taken by each family to ensure that this gift continues amongst their children.  It is common for courtship rituals amongst the Nobility to include demonstrations of magical talent.  When a child with strong magical abilities is found amongst the farm workers, they are brought into the household of the local noble family for education.  In order to avoid any \emph{distractions} from their studies, the children are strongly discouraged from contacting their birth family.  If they are particularly magically gifted, the child may even be sponsored to go to the \pSchool{}.  Should the child be discovered early enough, they are often formally adopted into the household and may not even be told the truth of their heritage.  It is considered dangerous to allow such potent magical talents to remain among commoners, who do not know how to help and support them.  

The gift of magic is one that comes with a heavy responsibility – that of ensuring that all people in multiple countries remain fed.  Every spring, the mages of each fiefdom go down to the farmlands to push their magic into the lands and seeds prepared by the farm workers, allowing the crops to grow much more quickly and bountifully than they would otherwise.  Plants are made resistant to pests, fungi, or rot that would blight the harvest.  It is exhausting but vital work.  However, the farmlands have come to be dependent on this magic -- the land is expended (and has been for centuries).  It is only the magic continuing to flow into the land that returns strength to it each year.  Were the magic to cease to flow into the land, it is believed that most plant life would die as the soil itself turned to dust.

In recent years, the Nobility amongst the \pFarm{} assisting with the war efforts have turned to a type of magic previously (and often still) scorned -- curses.  Previously the domain of those hedgemages amongst the commoners who grow up without attracting the attention of the Nobility, the Nobility have long considered curseworking to be plebeian and unsophisticated.  While some insist on calling curses providing beneficial effects to be blessings, all small magics both beneficial and negative fall under the umbrella term of curses.  With the power of the \pFarm{} mages behind them, however, these ``small'' magics have been able to sink ships and cause \pShip{} weaponry to malfunction and break.

\subsection*{Crime and Punishment}
While small crimes and misdemeanors are punished by society, there are some crimes which are considered so blasphemous by \cFarmGod{} that, if they are committed, the world itself will enact punishment regardless of whether the crime was observed by mortal eyes or not.  Although all Gods punish murder of humans with the removal of one's conscious memories, the \cFarmGod{\God} of the \pFarm{} considers all living creatures to be worthy of such protection.  Birds, fish, mammals, bugs – all creatures which walk, crawl, fly, or swim are precious to \cFarmGod{}. 

Abuse of animals is also strongly punished.  Should someone intentionally starve their animals, kick their pets, or beat their livestock, they will receive tenfold what they laid upon their animals.  Whispers are told of people who intentionally squashed a beetle on their plants, only to have a giant boulder hurtle from the sky and squash them where they stood.  While families work hard to ensure that their children learn this lesson early, it is known that \cFarmGod{} is more lenient towards youth.  Should a six year old accidentally crush an earthworm in their enthusiastic grasp, they are likely to get no more than a peck and a scold from a nearby hummingbird. 

\section*{Costuming}
The \pFarm{} amongst all social strata are a people in love with color – all members of society, from the poorest farm worker to the wealthiest among the Nobility, dress in a multitude of warm, rich earth tones.  Greens, yellows, browns, and copper are particularly favored.  Dye-making is an art amongst the working class and special washes are brewed from local plants to make garments difficult to stain.  Embroidery and needlework are pastimes for winter months and garments are often embellished.  Common motifs include flowers, leaves, and the sacred hummingbird.  Natural fabrics such as linen, cotton, and wool, are used.  Social status is indicated by the elaborateness of the garment, the richness of the fabric, and the presence of jewelry and other accessories. \emph{OOC Note: Clothing styles tend toward what you might see at a real life renaissance fair.}

\section*{Opinions about People from the Other Nations}
Uncouth and mercenary, the \pShippies{} are a warlike people -- unstructured and with no respect for tradition, they go wherever the wind takes them.  They have no respect for their elders either -- they let \emph{14-year-olds} help with decision making!  These spring green sprouts need to buckle down and learn the practicalities of the world before they start thinking about voicing their opinions about politics.

The \pTech{} are a nation of grand ideas and even grander technology – but they are scandalously free with music and the traveling circuses skirt blasphemy with their performances.  One mustn’t let young, impressionable children get too close to the performances, or their blasphemy may be catching.  The \pTech{} are also arrogant, haughtily looking down on the \pFarm{} and the labors and products of the earth.  Only the poorest among them deign to wear wool, linen, or cotton.  Despite all that, they are reliable allies and solid trading partners.  \end{document}
