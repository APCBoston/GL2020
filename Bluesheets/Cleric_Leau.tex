\documentclass[blue]{GL2020}
\parindent=0pt
\begin{document}
\name{\bLeauCleric{}}

Being a Cleric of \cEbbFull{\full} and \cFlowFull{\full} is beautiful, wonderful, and all encompassing. You are the Goddess's eyes, ears, and mouth to their people. You are expected to live your life according to the strictest standards of the their edicts. If you cannot follow the tenets, how will you convince the faithful that it is not only possible, but meaningful and fulfilling to do so?

\section{A Cleric's Powers and Responsibilities}
The Goddesses grant their Clerics power in many areas. They grant the clarity of mind to guide decisions, the steadiness of hand to repair things in need of it, and the force of will to destroy that which has run its course. They can also call out to the deities dirctly, although the form taken by the response is not always one the asker expected.

The greatest power a Cleric of \cEbb{} and \cFlow{} has is in listening without judgment to a supplicant's plight, and offering guidance and suggestions. It is important to keep in mind that the \pShip{} religion believes that no true peace can be achieved without Balance. If someone has been wronged, the most likely advice from a Cleric is, ``demand restitution or get even.'' The understanding is that nothing short of reciprocating in kind will allow someone to put down their hurts. If someone has wronged another, a Cleric will likely advise, ``make material amends as soon as possible. An apology that is just words is hollow.''

\section{Becoming a Cleric}
To become a Cleric of \cEbb{} and \cFlow{} is not easy. Most are not even allowed to try. The Goddesses mark those with the aptitude for it before their 15th birthday. It is exceedingly rare for them to appear later (but not completely unheard of). The mark may appear anywhere on the child's body, but the back of one or both wrists is the most common. Once the mark appears, the child may be enrolled at a Monastery exclusively, or continue at their Academy, and take supplemental classes at the Monastery. Either way, a minimum of 4 years of study is typical before anyone will sponsor the initiate (as they are now called) to the full priesthood. For details on the ritual to ordain a new Cleric, see ``Promoting someone to a Full Cleric'' in your  ``Being a Cleric'' greensheet.

Before the ritual can begin, the Initiate must pick a Path. Once you are brought into the priesthood on a Path, you cannot change. Therefore, the decision is incredibly important, and initiates often spend months, or even years, agonizing over it. Most Initiates ultimately choose to follow \cEbb{} or \cFlow{} primarily -- but some few attempt to walk the difficult path of true Balance. 
\begin{itemize}
  \item \cEbbFull{\full} is the keeper of ships, and the caretaker of things lost or taken away. Clerics of \cEbb{} know the value in loss, destruction, and removal. One cuts away at the wood to reveal the ship within. The end of one dream opens new and better opportunities. Sacrifice is honored highly. Loss and grief are comforted. Cleaning and clearing make way for the growth of \cFlow{}. 
  \item \cFlowFull{\full} is the keeper of the rain, and caretaker of things created or given. Clerics of \cFlow{} know the value of starting, building, and nurturing. One starts a new project because they believe in the outcome. One gives things away to start ripples, which become waves of sharing and growth. Creation and generosity are revered. Quality, and the patience to create it are cultivated. Unbridled growth prepares for the shaping and cultivating of \cEbb{}.
  \item To walk the path of Balance between these two Goddesses is the most challenging of all. Balance is about the moment of turning. When is it time to stop cutting things? When is it time to take a break from creating? How much iterating is too much? It's in the fullness at the top of the breath, and the emptiness at its end. The Path of Balance is all about turning points and change. It is about reigning in the excess of either extreme, and bringing things into harmony. There are no more than two dozen Clerics of Balance among all of the \pShippies{}. Those rare few for whom marks appear late are often particularly good candidates for the Path of Balance. If someone insists upon the path of Balance, both Avatars must be summmoned as part of the ritual.
\end{itemize}

\end{document}

