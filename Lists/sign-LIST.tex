
%%%%
%%
%% This file sets up the Sign and Label datatypes and creates Sign and
%% Label macros.
%%
%% Signs generally represent interesting parts of game area, usually
%% as things posted on walls.  Labels represent other things, often on
%% or inside envelopes, that are part of complex mechanics.
%%
%% The default value for \MYloc will inherit location from the Place
%% or Sign most immediately up the ownership tree.  Override this by
%% setting \MYloc to anything (even blank).
%%
%% Sign is for full-sized signs that would cover most of a large
%% manila envelope; SignMedium is for signs sized to half-sized manila
%% envelopes; SignSmall is for signs sized for small manila envelopes
%% (the same size as item cards).  Label, LabelMedium, and LabelSmall
%% are analagous, but they don't have a \takedownby note at the
%% bottom.  You can always use a sign or label without an envelope or
%% with a differently-sized envelope.  Choose which based on
%% visibility and content.
%%
%% SignTiny is for signs you want to be hard to find; it is small and
%% does not have a \takedownby note.  SignDot is for a very small
%% "dot" which only has a title.
%%
%% SignStrip produces a strip of paper (without a \takedownby note)
%% with labels on the outside that show on both sides if you fold it
%% in half.  These are a convenient alternative to sub-envelopes. They
%% can also be used for "s-packets" taped to walls (see
%% Extras/README-s-packets).
%%
%% LabelCover produces a label similar to the cover to a research
%% notebook.  LabelPage, likewise, produces a page.
%%
%% EOG is for full-sized end-of-game signs.
%%
%%%%%

\DECLARESUBTYPE{Sign}{Element}
\PRESETS{Sign}{
  \FD\MYloc	{\mylocation} %% real-space location
  \FD\MYtext	{} %% text of sign
  }
\POSTSETS{Sign}{
  \edef\mylocation{\MYloc}
  \protected@edef\@ownerstring{%
    \MYname%
    \ifx\mylocation\empty\else\ (\mylocation)\fi%
    }
  }
\def\mylocation{}

\def\loc#1{\rs\MYloc{#1}}

\DECLARESUBTYPE{SignMedium}{Sign}
\DECLARESUBTYPE{SignSmall}{Sign}
\DECLARESUBTYPE{SignTiny}{Sign}
\DECLARESUBTYPE{SignDot}{Sign}
\PRESETS{SignDot}{\s\MYtext{}}

\DECLARESUBTYPE{Label}{Sign}
\PRESETS{Label}{\s\MYloc{}}
\DECLARESUBTYPE{LabelMedium}{Label}
\DECLARESUBTYPE{LabelSmall}{Label}

\DECLARESUBTYPE{SignStrip}{Sign}
\DECLARESUBTYPE{LabelCover}{Label}
\DECLARESUBTYPE{LabelPage}{Label}

\DECLARESUBTYPE{EOG}{Sign}
\PRESETS{EOG}{%
  \s\MYname	{End Of Game}
  \s\MYtext	{{\bf\Huge You may not pass through here.}}
  }


%%%%%
%% \signbig[<location>]{<name>}{<text>}
%% \eog[<location>]
%%
%% \signmdeium[<location>]{<name>}{<text>}
%% \signsmall[<location>]{<name>}{<text>}
%% \signtiny[<location>]{<name>}{<text>}
%% \signdot[<location>]{<name>}
%%
%% \labelbig{<name>}{<text>}
%% \labelmedium{<name>}{<text>}
%% \labelsmall{<name>}{<text>}
%%
%% \signstrip[<location>]{<name>}{<text>}
%% \labelcover{<name>}{<text>}
%% \labelpage{<name>}{<text>}
\newinstance{Sign}{\signbig[3][\mylocation]}{
  \s\MYloc{#1}\s\MYname{#2}\s\MYtext{#3}}
\newinstance{EOG}{\eog[1][\mylocation]}{\s\MYloc{#1}}

\newinstance{SignMedium}{\signmedium[3][\mylocation]}{
  \s\MYloc{#1}\s\MYname{#2}\s\MYtext{#3}}
\newinstance{SignSmall}{\signsmall[3][\mylocation]}{
  \s\MYloc{#1}\s\MYname{#2}\s\MYtext{#3}}
\newinstance{SignTiny}{\signtiny[3][\mylocation]}{
  \s\MYloc{#1}\s\MYname{#2}\s\MYtext{#3}}
\newinstance{SignDot}{\signdot[2][\mylocation]}{
  \s\MYloc{#1}\s\MYname{#2}}

\newinstance{Label}{\labelbig[2]}{
  \s\MYname{#1}\s\MYtext{#2}}
\newinstance{LabelMedium}{\labelmedium[2]}{
  \s\MYname{#1}\s\MYtext{#2}}
\newinstance{LabelSmall}{\labelsmall[2]}{
  \s\MYname{#1}\s\MYtext{#2}}

\newinstance{SignStrip}{\signstrip[3][\mylocation]}{
  \s\MYloc{#1}\s\MYname{#2}\s\MYtext{#3}}
\newinstance{LabelCover}{\labelcover[2]}{
  \s\MYname{#1}\s\MYtext{#2}}
\newinstance{LabelPage}{\labelpage[2]}{
  \s\MYname{#1}\s\MYtext{#2}}


%%%%%
%% \sEOG{}
%% use \sEOg[\loc{<location>}]{} for EOG sign at a specific place
\NEW{EOG}{\sEOG}{
  }


%%%%%%%%%%%%%%%%%%%%%%%%%%%%%%%%%%%%%%%%%%%%%%%%%%%%%%%%%%%%%%%%%%

\NEW{Sign}{\sTest}{
  \s\MYname	{A Room}
  \s\MYloc	{10-250}
  \s\MYtext	{A lecture hall with large, sliding blackboards.}
  }


%%%%%%%%%%%%%%%%%%%%%%%%%%%%%%%%%%%%%%%%%%%%%%%%%%%%%%%%%%%%%%%%%%


%%GM HQ
\NEW{Sign}{\sMurderConsequences}{
  \s\MYname	{If your character succeeds in murdering someone, read this sign}
  \s\MYloc	{}
  \s\MYtext	{Take a copy of the greensheet in the packet attached to this sign. It has instructions on exactly how losing your memory works, and what you have access.}
}
  
\NEW{Sign}{\sSignCOne}{
  \s\MYname	{Sign C}
  \s\MYloc	{}
  \s\MYtext	{You may not interact with this sign unless instructed to do so. If you are instructed to do so, you will lift this sign and read the sign under it.}
}

\NEW{Sign}{\sSignCTwo}{
  \s\MYname	{Sign C -2 Dreaming of the Past}
  \s\MYloc	{}
  \s\MYtext	{After successfully completing the ritual to use \iMirror{} to look into the past, you went to sleep. Overnight, you had the following dream://
  \emph{You see \iMirror{} in your mind. In it, you see your reflection. The mirror drifts close enough to you for your breath to fog the surface. Instead of fading, the mist begins to swirl slowly across the darkening surface of the mirror. You feel compelled to touch it. And in doing so, you fall forward into the mirror, tumbling into the vortex. Back through the swirling mist of time you are drawn.}

\emph{You find yourself in a fancy restaurant in \pTech{}. Two figures are speaking quietly in a corner booth. \cEvil{} says “I can buy you more time to figure out how to stop the storms. But you must do something for me in return.” \cHeadScientist{} nods eagerly, and the scene fades as \cEvil{} passes \cHeadScientist{} a small satchel.}

\emph{A flash of \cEvil{} handing a small bottle to someone dressed as a \pSchool{} employee, on the edge of the Lake beneath the \pSc{}. \cEvil{} says “make sure only the students drink the wine. More bodies than that and people might find their way back to me.” The person nods.}

\emph{Suddenly you’re in a forest clearing, somewhere in \pFarm{}. \cDiplomat{} is meeting with \cEvil{}. \cEvil{} says “I have procured the fake Voting Stones you asked for,” and hands over a small satchel to \cDiplomat{}. \cDiplomat{} replies “Good, now make sure you clean up any loose ends.” The grin \cEvil{} has as the scene fades sends chills down your spine.}

\emph{The \pSchool{}, the Ritual to Control the Storm just completed. The students stand or sit, panting. One student is clearly a \cHeir{\formal}. Another looks very much like \cAssistantScientist{} might have looked 6 years ago.  The Teachers and Advisors rush in from the edges of the room to support the students. Somehow in the chaos glasses of wine are produced. A toast is proposed. All of the students drink, except \cAssistantScientist{}’s lookalike. \cAssistantScientist{\They} sniff the glass, make a dubious face, and put it down. As \cAssitantScientist{\their} classmates start to collapse, \cAssistantScientist{\they} slip out a side door, unnoticed.
}
  
The Dream reverberates with magic, and you know it to be the truth - it may not be the *whole* truth, but everything you saw happened. When you speak of what the dream showed you, if you speak truly, your voice will ring with the same magic, carrying the weight of truth to convince your listeners. (OOC: You should tell other players that your voice carries the weight of magical truth when you speak of what you’ve seen. Keep in mind however that a character may choose to pretend not to believe you for any number of reasons.}
}
  
  
  
  
  
  
